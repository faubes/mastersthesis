EDITS: We must always clearly state when an action or consequence occurs: at the end of a turn
or the end of a round. (A round being composed of two turns).

Add Theorem and proof environments.

Another theorem on worst-case capture time? Diameter of grid?

The Grid graph $G_{m,n}$ is a rectangular arrangement of $mn$ vertices in
$m$ rows and $n$ columns. Vertices are joined by an edge if they are on the same
row or column. The goal of this section is to prove

\begin{theorem} The zombie number of the Grid is 2 for any $m, n$.
\end{theorem}

\begin{figure}
  \centering
  \includegraphics[width=0.5\textwidth]{grid/grid4by5}
  \caption{$G_{4,5}$ \label{fig:grid4by5}}
\end{figure}

As a quick aside, note that the grid graph can be constructed by taking the
Cartesian product of two paths $P_m \square P_n$.
The Cartesian product of $G_1$ and $G_2$, denoted $G_1 \square G_2$ is a graph $G$ whose
 vertices are all pairs of vertices of the two graphs.
Two vertex pairs are connected by an edge if they are equal in one component and
the other is joined in the original graph. In set notation, that is

\[V(G) = \{ (u, v) \mid u \in V(G_1), v \in V(G_2) \} \]
\[E(G) = \{ \{(u_1, v_1), (u_2, v_2) \} \mid (u_1 \sim_{G_1} u_2 \land v_1 = v_2) \lor (u_1 = u_2 \land v_1 \sim_{G_2} v_2) \} \]

We claim that two zombies suffice to win on this family of graphs since they can execute
a guarding strategy. To demonstrate this, we need the following observations about shortest
paths on the grid.

First, if the zombie and the survivor are on the same row (or column) then there is a
single shortest path and hence only one valid zombie move: the zombie moves
closer to the survivor along the row (or column).

\begin{figure}
  \centering
  \includegraphics[width=0.5\textwidth]{grid/samerow}
  \caption{Zombie and Survivor on Same Row \label{fig:samerow}}
\end{figure}

Second, if the zombie and survivor are on different rows and columns then
 there are at least two shortest paths joining them and exactly
 two possible zombie moves: horizontal or vertical.
The survivor and zombie are assumed on different rows and columns
so the zombie can make progress in one or the other direction.

 \begin{figure}
   \centering
   \includegraphics[width=0.25\textwidth]{grid/diffrowcol}
   \caption{Zombie and Survivor on Different Row and Column\label{fig:diffrowcol}}
 \end{figure}

We now show that two zombies can play a shadowing strategy which is guaranteed to
capture the survivor. We mimic a proof strategy from Cops and Robbers in which we show
that the Robber Territory is shrinking at every round.
We analoguously define the Survivor Territory $S_j \subset V(G)$  as
the set of vertices to which the survivor may move on turn $j$ without being eaten
by a zombie.

To enact the shadowing strategy, the zombies may choose any starting position
(and so the set of winning zombie starts $Z_W(G_{m,n}) = V(G_{m,n})$ in Fitzpatrick's notation -- so the grid
belongs to the family of graphs for which any zombie starting position will work -- Fitzpatrick
asks if there is a characterization of graphs for which any start will win. We do not have an answer
but observe that the grid belongs to this family for 2 zombies).
Each zombie will shadow the survivor's position along an axis: one horizontal,
the other vertical.

\subsection{The Shadowing Strategy}

\begin{proof}
Let us consider one zombie at a time, say the zombie which will shadow the
survivor's vertical shadow (i.e., its column), since the other zombie's behaviour is
symmetric. We show that the zombie will eventually
capture the vertical shadow -- that the zombie will close the horizontal distance between the
survivor and the zombie -- and that, once it does so, the zombie can always recapture the survivor's shadow
after the survivor moves.

Assuming that the zombie and survivor are on different columns, then the zombie
may move one column closer since the vertex on the same row but one column closer
 lies on a shortest path to the survivor. If they are on the same row, then that is
 the only possible zombie move. In response, the survivor may:

\begin{itemize}
  \item Remain in place in which case the zombie has closed the horizontal distance by 1.
  \item Move vertically (up or down) in which case again the zombie has closed the horizontal distance by 1.
  \item Move horizontally towards the zombie in which case the horizontal distance is reduced by 2.
  \item Move horizontally away from the zombie in which case the horizontal distance is preserved.
\end{itemize}

In the first three scenarios, the horizontal distance between the two adversaries has been reduced.
The fourth scenario in which the survivor moves away cannot occur indefinitely since the
grid is finite.  Thus, in at most $n$ rounds (the number of columns)
the zombie will capture the survivor's vertical shadow.

Suppose now that the zombie has captured the survivor's vertical shadow; that they are now
on the same column. It is
clear that the zombie can recapture the survivor's shadow no matter how the survivor moves:

\begin{itemize}
  \item If the survivor moves vertically or remains in place, then the zombie must move vertically
   and the survivor's vertical shadow remains captured.
  \item If the survivor moves horizontally, then the zombie may choose to mimic the move and thereby
  recapture the vertical shadow.
\end{itemize}

This argument shows that after a finite number of moves a zombie may capture the
survivor's vertical shadow. Now observe that once the survivor's
vertical shadow has been captured, the survivor can never enter the zombie-occupied
row: any attempts to go around the zombie are immediately blocked. (Expand and clarify?)

\subsection{Shrinking the Survivor Territory}

By the previous discussion, after $\max \{ m, n \}$ rounds, both the survivor's vertical
and horizontal shadows have been captured. Suppose we have reached a point in the
game where the zombies have moved and captured both horizontal and vertical shadows.

\begin{figure}
  \centering
  \includegraphics[width=0.25\textwidth]{grid/shadowscaptured}
  \caption{Once Both Shadows Are Captured\label{fig:shadowscaptured}}
\end{figure}

The survivor has five possible moves:

\begin{itemize}
  \item Stay in place, in which case both zombies have a single shortest path and
  move closer along their current row/column. So the Survivor Territory has shrunk by a column and a row.
  \item Move vertically, in which case the zombie capturing the vertical shadow
  has no choice but to move closer, while the other zombie recaptures the horizontal shadow.
  Here the Survivor Territory has shrunk by one row.
  \item Move horizontally, in which case the zombie capturing the horizontal shadow
  has not choice but to move closer, while the other zombie recaptures the vertical shadow.
  Now the Survivor Territory has shrunk by one column.
\end{itemize}

\begin{figure}[h]
  \centering
  \includegraphics[width=0.25\textwidth]{grid/survivorterritory}
  \caption{Shrinking Survivor Territory: The survivor moves up when shadowed; the zombie
  guarding the vertical shadow must move up, thereby eliminating the row below. The
  zombie guarding the horizontal shadow also moves up, so its column remains blocked.\label{fig:survivorterritory}}
\end{figure}

In every scenario, at least one of the zombies is forced to move one
step closer to the survivor. Since the survivor can never enter the rows and columns
occupied by the shadowing zombies, this means that the Survivor Territory
 shrinks by at least one row or one column at every round.
Since the grid is finite, the Survivor Territory is eventually empty and hence
the survivor is captured.

\end{proof}

\subsection{Defining Shadowing Rigorously}

Label the vertices of the grid using the integer points of the first quadrant
of the plane and consider $z_1, z_2, s \in V(G_{m,n})$ as points in $[0, m-1] \times [0, n-1]$.
Rows and columns are then first and second coordinates of points in a finite lattice.
Say $s^j = (x^j_0, y^j_0), z^j_1 = (x^j_1, y^j_1), z^j_2 = (x^j_2, y^j_2)$ are the
positions of the players $s, z_1$ and $z_2$ on round $j \geq 0$.
The players cannot escape the bounds of the grid and so
 $0 \leq x^j_i \leq n -1$ and $0 \leq y^j_i \leq m-1$ for $i \in \{0, 1, 2\}$ and for $j \geq 0$.

Let's show that -- after a finite number of turns -- a zombie can mirror
the survivor's $x$-coordinate. Formally, there exists a round $k\geq 0$ such that $x^k_0 = x^{k+1}_1$
and that for all $j > k$, $x^j_0 = x^{j+1}_1$. That is to say, from turn $k$ onwards, the zombie can
always move onto the $x$-projection of the survivor on its turn.

Note that have $N[z^j_i] \subseteq \{ (x^j_i, y^j_i), (x^j_i \pm 1, y^j_i), (x^j_i, y^j_i \pm 1) \}$
and the inclusion is strict when $z^j_i$ is on the boundary of the grid.

Suppose we already have $x^k_0 = x^k_1$. As mentioned above, in this case the
 zombie has a single shortest path to $(x^k_1, y^k_1 \pm 1)$ where for simplicity
 we will assume that the move is ``upwards'' to $(x^k_1, y^k_1 + 1)$. The zombie
 has moved onto the survivor's $x$-coordinate. The survivor has five possible responses.

Now, we can assume that $x^j_0 > x^j_1$ (the opposite being symmetric). The zombie
may follow two shortest paths, one of which is to $(x^j_1 + 1, y^j_1)$.
...
