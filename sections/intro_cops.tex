We start with an explanation of the game, then give some key results from C \& R since much of our thesis is a comparison between the two games. We investigate ``the cost of being undead'', as Fitzpatrick \cite{fitzpatrick2016deterministic} would call it.

\subsection{How to Play Cops and Robbers}

C \& R is a two player game: one player controls the cops, the other the robber.
The cops begin the game by choosing start vertices. Next, the robber choooses a start position.
On each following round the cops may move along an edge to a neighbouring vertex
or remain in position.
Here a move is an instantaneous jump between adjacent vertices.
If the robber remains uncaught after the cops have had a chance to move, the
robber then gets the opportunity to move along an edge.

In this game, the players have complete information of the graph and the positions of the players.
The cops move, the robber responds and these two \textit{turns} make one \textit{round}.

The game is decided when either:
\begin{itemize}
\item A cop captures the robber. That is, the cop player wins if one of the cops move onto the vertex
occupied by the robber.
\item The robber wins if it can evade the cops indefinitely.
\end{itemize}

\subsection{Cops and robbers, Cop-Number}

Study of vertex-pursuit games is first attributed to Quilliot \cite{quilliot1978jeux, quilliot1983problemes}, and Nowakowski and Winkler \cite{nowakowski1983vertex}.
These researchers independently consider games of C \& R with a single cop and characterize by way of a relation those graphs where the cop always wins. These are now known as \textit{cop-win} graphs
and can be recognized by the existence of an ordering of the vertices called a \textit{dismantling};
so-called because it is the successive deletion of \textit{corners} resulting in a single vertex (see Subsection~\ref{intro dismantlings}).

The \textit{cop-number} of a graph, denoted $c(G)$, is introduced by Aigner and Fromme  \cite{aigner1984game} and defined as the minimum number of cops required
to guarantee they win on a graph $G$. Later, Berarducci et al. and Hahn et al.  generalized the characterization of cop-win graphs
into \textit{$k$-cop win} graphs \cite{berarducci1993cop, hahn2003characterisation}.

A graph is $k$-cop win if and only if there exists a function (on a $k$-product of the graph to represent the position of the cops)
which satisfies certain properties; essentially it is a function which takes as input a position $C$ of cops and returns the next position for the cops that guarantees a win (see \cite{bonato2011game}[p. 119]).
There exists a polynomial-time algorithm for deciding whether a graph is $k$-cop-win by iteratively
solving for this function, so we can decide if $c(G) \leq k$ for any graph in polynomial time as long as $k$ is fixed and not a function of $\lvert V(G) \rvert$.

Another important line of inquiry relating to the cop-number is the investigation of Meyniel's conjecture, which posits that $\bigO(\sqrt{\lvert V(G)\rvert })$ is an upper bound on the cop-number \cite{frankl1987cops}.
Incremental progress has been made on special classes of graphs as well as for graphs in general  \cite{gera2016graph}[p. 31].

\subsection{The Cop-Number and the Genus of the Graph}

One of the most surprising results about the C \& R is owed to Aigner and Fromme \cite{aigner1984game}, who showed that the cop number of a planar graph is at most 3.
Basically, a graph is planar if it can be drawn in the plane (say, on a piece of paper) without crossing any edges. Aigner and Fromme describe a 3-cop strategy which uses \textit{isometric} paths of the graph to encircle and entrap the robber.

Outerplanar graphs are planar graphs which can be drawn such that all vertices belong to a common face
(called the \textit{outerface}). Clarke \cite{clarke2002constrained} showed that the cop number of outerplanar graphs is 2 by considering
two possible cases: those with and without cut vertices. The 2 cops have a winning strategy on outerplanar graphs without cut vertices, and this strategy can be used to cordon off sections (blocks)
of the outerplanar graph.

The game has also been studied for graphs embeddable in surfaces of higher order.
In 2001, Schroeder conjectured \cite{bonato2017topological} that for a graph of genus $g$,
the cop-number is at most $g+3$. [TODO: Include here the best known bound].

\subsection{Relation to the Girth and Minimum Degree of a Graph}

Aigner and Fromme also show a relationship between the cop-number, the girth of a graph and
its minimum degree \cite{aigner1984game}. More precisely, if $G$ has girth at least 5, then $c(G)\geq \delta(G)$ where $\delta(G)$ is the minimum degree of $G$.

This result has since been refined \cite{frankl1987cops}: if $G$ has girth at least $8t-3$ and $\delta(G) = d$, then more than $d^t$ cops are needed to win. In a recent
seminar by B. Mohar (Graph Searching Online Seminar, held May 1, 2020) it was
argued that a graph with girth $g$ and $\delta(G)=d$ will require at least $\tfrac{1}{g}(d-1)^{\lfloor \frac{g-1}{4} \rfloor}$.

\subsection{Dismantlings, Cop-win Trees \label{intro dismantlings}}

Quilliot and Nowakowski both indepedently characterized cop-win graphs as those which
admit a \textit{dismantling}.

Let $G$ be a reflexive graph with $x\in V(G)$ a fixed vertex. A (one-point) \textit{retract} is an edge preserving function $f : G \rightarrow H = G \setminus v$
(aka a homomorphism) such that $f(v) = x$ for some $x \neq v \in V(G)$ and $f$ restricted on $H$ is the identity.

Formally,

\[ f(v) = x \qquad f(u) = u \qquad \forall u \in V(G)\setminus \{ v \} \]

and

\[ xy \in E(G) \implies f(x)f(y) \in E(G \setminus \{ v \}) \].

Since the graphs studied herein are reflexive, a one-point retract can be seen as the
absorption of one vertex into another. The edge between two adjacent vertices becomes another loop.
The retract maps a graph $G$ to graph $G'$ with one less vertex.

A vertex $u$ is a \textit{corner} if its closed neighbourhood
is a subset of one of of its neighbours' closed neighbourhood, i.e.,

\[u \in V(G) \qquad \text{and} \qquad \exists v\in V(G) : N[u] \subseteq N[v] \].

It is possible to define a retract on corner $u$: if $u$ is a corner, then it is
dominated by some $v \in V(G)$. So if $x \in V(G)$, $x \neq u$ and
$xu \in E(G)$ then $xv \in E(G)$ since $u$ is a corner. Therefore the map

\[ f(x) = \begin{cases}
v & \text{if } x = u \\
x & \text{otherwise}
\end{cases} \]

is edge-preserving since $f(x)f(u) = xv$ and $xv \in E(G)$, so $xv \in E(H) = E(G - u)$.
For other vertices $x,y \not\in \{u,v\}$, $f(x)f(y) = xy \in E(G)$ so $f(x)f(y) \in E(G- v)$ also.
This shows that $f$ is a homomorphism as required and hence a retract.
This is a formal way of saying that a corner of a graph can be folded into a
dominating vertex.

A \textit{dismantling} is a sequence of retracts $f_1, f_2, \dots, f_{n-1}$ such that the
composition $F_{n-1} = f_{n-1} \circ f_{n-2} \circ \cdots f_2 \circ f_1$ gives a
function for which $F_{n-1} (G) = K_1$. That is, there is a sequence of retracts
which maps the graph to a single vertex.

Not all vertices of a graph need to be corners in order for there
to exist a dismantling: it suffices to have an ordering where each $v_i$ is a corner in
$G[\{v_i, v_{i+1}, \dots, v_n\}]$.
Such a sequence of $f_i$'s defines a copwin ordering
\[ \mathcal{O} = ( v_1, v_2, \dots, v_n ) \]
 where $v_1$ is a corner in $G_1 = G$, $v_2$ is a corner in $G - v_1$, and so on.
A fundamental result in C \& R is that cop-win graphs -- graphs for which a single
cop is guaranteed to win --  are characterized by the existence of such dismantlings.
A graph is copwin if and only if it is dismantlable.

A cop-win spanning tree combines the idea of a dismantling with a spanning tree
and was first proposed in \cite{clarke2002constrained}. A cop-win spanning tree $S$ is defined as a tree where $x,y\in V(G)$ and $xy \in E(S)$ if there exists a retract $f_j$ in the dismantling
$F_n = f_{n} \circ f_{n-1} \circ \cdots \circ f_{2} \circ f_1$
such that $f_j (x) = y$ or $f_j (y) = x$ in $G[\{j, j+1, \dots, n \}]$.
Cop-win spanning trees give a strategy for the cops to follow: start at the root
(the last vertex in the ordering) and descend the tree in the branch containing the robber.
Lemmas 2.1.2 and 2.1.3 from \cite{clarke2002constrained} show that the cop
can always stay in the same branch (and above) the robber in the tree. So the
robber is eventually stuck in a leaf and caught.
