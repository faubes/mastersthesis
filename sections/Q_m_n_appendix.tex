\appendix
\section{Simplifying $z_2$'s inequalities for Case II.A.}
We have
\begin{align*}
 n - \Delta + \ell \leq & \Delta + (m - \ell) & \text{and} \\
 n - \Delta + \ell \leq & \Delta + 1 + \ell
\end{align*}
\begin{center}or\end{center}
\begin{align*}
 n - \Delta + 1 + m - \ell \leq & \Delta + (m - \ell) & \text{and} \\
 n - \Delta + 1 + m - \ell \leq & \Delta + 1 + \ell
\end{align*}
These can be simplified further with a bit of algebra:
\begin{align*}
 n-m +2\ell \leq & 2 \Delta & \text{and} \\
 n-1 \leq        & 2\Delta
\end{align*}
or
\begin{align*}
 n + 1 \leq         & 2 \Delta & \text{and} \\
 n + m - 2\ell \leq & 2 \Delta
\end{align*}

These inequalites are of the form
\begin{align*}
 n-x \leq & 2 \Delta & \text{and} \\
 n-1 \leq & 2\Delta
\end{align*}
\begin{center}or\end{center}
\begin{align*}
 n + x \leq & 2 \Delta & \text{and} \\
 n + 1 \leq & 2 \Delta
\end{align*}

Where $x = m -2\ell$.

Supposing $x\geq 0$, we have
\begin{align*}
 n-x \leq & n+x \leq 2 \Delta & \text{and} \\
 n-1 \leq & n+1 \leq 2\Delta
\end{align*}

Whereas if $x <0$, then from assumption A we must have $m-2\ell = -1$, so that
our constraints reduce to
\begin{align*}
 n+1 \leq & 2 \Delta & \text{and} \\
 n-1 \leq & 2 \Delta
\end{align*}

\newpage
\section{Simplifying $z_2$'s inequalities for Case II.B.}

We have
\begin{align*}
 n - \Delta + \ell \leq & \Delta + (m - \ell) & \text{and} \\
 n - \Delta + \ell \leq & \Delta + 1 + \ell
\end{align*}
\begin{center}or\end{center}
\begin{align*}
 n - \Delta + 1 + m - \ell \leq & \Delta + (m - \ell) & \text{and} \\
 n - \Delta + 1 + m - \ell \leq & \Delta + 1 + \ell
\end{align*}
These can be simplified further with a bit of algebra:
\begin{align*}
 n-m+2\ell \leq & 2 \Delta & \text{and} \\
 n-1 \leq       & 2\Delta
\end{align*}
or
\begin{align*}
 n+1 \leq        & 2 \Delta & \text{and} \\
 n+m-2\ell  \leq & 2 \Delta
\end{align*}

These inequalites are of the form
\begin{align*}
 n-x \leq & 2 \Delta & \text{and} \\
 n-1 \leq & 2\Delta
\end{align*}
\begin{center}or\end{center}
\begin{align*}
 n + 1 \leq & 2 \Delta & \text{and} \\
 n + x \leq & 2 \Delta
\end{align*}

Where $x = m -2\ell$. Now since assumption B gives $m - 2\ell \leq -1$, we
see that
\begin{align*}
 n-1 \leq & n-x \leq 2 \Delta \\
          & \text{or}         \\
 n+x \leq & n+1 \leq 2 \Delta
\end{align*}

\newpage
\section{Alternate Strategy}

We wish to show -- in a simpler way -- that $\Delta = \lfloor \frac{n}{2} \rfloor$
also works for the cornering strategy.

In order to be in Case I. A, we need

\[ 4 \leq 2 \ell \leq m+1 \]
and

\[ 2\Delta \leq n - m + 2\ell \]
or
\[ 2\Delta \leq n-1 \]

Negating these conditions gives

\[ 2\Delta \geq n - m + 2\ell +1 \]
and
\[ 2 \Delta \geq n - 1 + 1 \]

or

\[ m+1 \leq 2 \ell \leq 2m -2 \]

Suppose we set $\Delta = \lfloor \frac{m}{2} \rfloor$ and we assume that
we are not in Case 1. A. Since $2\Delta \geq n$,
\[
2\Delta = 2 \left\lfloor \frac{m}{2} \right\rfloor =
\begin{cases}
m & \text{if $m$ is even} \\
m -1 & \text{if $m$ is odd}
\end{cases}
\]

Assuming $m$ is odd leads to a contradiction since
\[ n-1 \geq m-1 = 2\Delta \geq n \]

Since $n \geq m = 2 \Delta \geq n$, we must have $m = n$ and $m$ even.

\begin{align*}
  2 \Delta & \geq n - m + 2\ell + 1 \\
  m & \geq m - m + 2 \ell + 1 \\
  m & \geq 2 \ell + 1 \\
  2 & \ell \leq m -1
\end{align*}

So, if $m = n$ and they are even, then we are in Case 1. A unless $2 \ell \leq m -1$.

To recap: If we set $\Delta = \lfloor \frac{m}{2} \rfloor$, we are in Case 1.A unless
\[ m = n \qquad \text{and they are even} \]
\[ \Delta = \lfloor \frac{m}{2} \rfloor = \frac{m}{2} \]
\[ 4 \leq 2 \ell \leq m -1 \]

Now, can we be in Case 1. B? Case 1. B is described by the following constraints:

\[ m+1 \leq 2 \ell \leq 2m -2 \]
and
\[2 \Delta \leq n +1 \]
or
\[2\Delta \leq n + m - 2 \ell \]

The negation of which is:
\[2 \Delta \geq n +1 +1 \]
and
\[2\Delta \geq n + m - 2 \ell + 1\]
or
\[4 \leq 2 \ell \leq m +1 \]

But this leads to the contradiction:
\[ n \geq m \geq 2 \Delta \geq n +2 \]

In remains to check if we win in Case 2. A.

Assuming still that
\[ m = n \qquad \text{they are even} \]
\[ \Delta = \frac{m}{2} \]
\[ 4 \leq 2 \ell \leq m -1 \]

The win conditions require
\begin{align*}
2n - 2m + 2\ell \leq 2 \Delta \leq 2n + 2\ell -m -1 \\
2m - 2m + m - 1 \leq 2 \Delta \leq 2m + 4 - m - 1 \\
m - 1 \leq 2 \Delta \leq m + 3
\end{align*}

Which holds for $\Delta = \frac{m}{2}$.
