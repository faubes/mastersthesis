We analyze the Game of Zombies \& Survivors on a cycle with a single chord.

\begin{definition}
 Take a cycle of length $m+n$ and add a chord which
 divides the cycle into paths $P_m$ and $P_n$ of lengths $m$ and $n$.
 Without loss of generality $m \leq n$. We denote such a cycle as $Q_{m,n}$.
\end{definition}

\begin{figure}
  \centering
 \includegraphics[scale=0.15]{q_m_n/Q_m_n_basic}
 \caption{A cycle with one chord \label{fig:Q_m_n_basic}}
\end{figure}

\begin{theorem}
 The zombie number of a cycle $Q_{m,n}$  $(3 \leq m \leq n)$ with
 a chord dividing the cycle into paths of lengths $m$ and $n$ is 2.
\end{theorem}

\begin{proof}
  Denote as $P_m$ and $P_n$ the paths of lengths $m$ and $n$ respectively.
 We think of $Q_{m,n}$ as embbeded in the plane with $P_m$
 -- the shortest side -- on the left.
 This does not limit the generality of the following and allows us to define
 (counter-)clockwise distance: the length of the path
 along a cycle with respect to this embedding.

  Setting $m=n=1$ gives $K_2$ with two added loops, which is zombie-win.

  With $m=n=2$ we have two adjacent cliques $K_3$ which are dominated by a single vertex,
  so it is also zombie-win.

  For $m=2$ and $n\geq 4$, 2 zombies win by starting on diametrically
  opposed vertices on the cycle $C_{n+2}$.

  If $m=n=3$ the zombie number is 2 since two zombies on the chord enpoints dominate the graph.

  For $m=3$, $n=4$, the zombie number is also 2: placing the zombies on the endpoints
  of the chord divides the graph into $C_4$ and $C_5$ and the zombies clearly win from this
  position.

  The same strategy works for $Q_{3,6}$, $Q_{4.4}$, $Q_{4,5}$ and $Q_{5,5}$ but it does
  not work for $Q_{3,7}$, $Q_{4,6}$ nor indeed for any $Q_{m,n}$ for $m \geq 3$ and $n \geq 6$.

  We seek a winning zombie strategy (that is, a zombie start) for $m \geq 3$, $n \geq 6$.
  The chord is the crux of the game, so first we assume that one zombie is on the chord
  and another at some distance $\Delta$ while the survivor is somewhere on $P_m$.
  We know the first zombie chases the survivor around the cycle, so
  we need to control the arrival of the second zombie so that the survivor cannot escape,
  nor can it trick the second zombie into spinning the same direction as the first.

  Second, we show how to position the zombies at the start of the game so that --
  no matter where the survivor starts -- a losing position is guaranteed.
  Either the survivor is stuck on a path between the two zombies (so that capture
  is obviously inevitable) or the survivor will be pushed into the carefully
  orchestrated scenario described in the first part of the proof.

  Lastly, we show that such a starting position is always available to the zombies
  for any $m \geq 3$, $n \geq 6$.

 Note that if $P_1$ and $P_2$ are two possible $zs$-paths with distinct next moves and
 \[ |P_1| \leq |P_2| \]
 then in the following argument we suppose that the zombie follows $|P_1|$
 since that is a valid move.

 \newpage

 \subsection{Cornering the Survivor on the Smallest Cycle}

\begin{proofpart}
   Suppose that the game has reached the following state:

  \begin{figure}
    \centering
    \includegraphics[scale=0.20]{q_m_n/diagram1}
    \caption{One zombie on the chord \label{fig:onthechord}}
  \end{figure}

  \begin{itemize}
   \item the first zombie is on an endpoint of the chord, say $v$
   \item there are $\Delta$ vertices counting clockwise from $u$ to $z_2$.
   \item the survivor is on $P_m$ at a distance of $\ell$ vertices counting
   clockwise from $v$.
  \end{itemize}

  By comparing the lengths of different paths, we calculate the values of $\Delta$
  which guarantee that the survivor will be cornered on $P_m$. That is to say,
  the survivor will be intercepted by $z_2$ before it can reach any vertex
  in $Q_{m,n} \setminus P_m$.

  Denote as $\ell$ the length of the clockwise
  path from $v$ to $s$. Note that we must have $2 \leq \ell \leq m-1$ else
  $z_1$ captures the survivor on the next round.

  We can assume that once $z_1$ chooses a direction from $v$
  that it will continue in that direction:
  either the zombie has no choice or both directions around
  the cycle are of the same length (and so
  may continue in the same direction).

  We can also assume that on its turn the survivor will move away from
  $z_1$ and maintain a distance of $\ell$ (or $m-\ell +1$, if they are moving counter-clockwise)
  since a winning survivor strategy which involves waiting a turn
  or moving backwards is equivalent to a survivor strategy
  which always moves but starts with a smaller (or larger) value of $\ell$.

  These two assumptions allow us to ``fast-forward'' the game by $\Delta$ rounds
  and determine when the survivor is captured.

  Since $z_1$ is already on the same cycle as the survivor, it has two options:

  \begin{itemize}
   \item[A.] $z_1$ goes clockwise if $\ell \leq 1 + m - \ell$.
         Combined with the bounds on $\ell$, this gives $4 \leq 2 \ell \leq m + 1$

   \item[B.] $z_1$ goes counter-clockwise if $1 + m - \ell \leq \ell$.
         Combined with the bounds on $\ell$, we obtain $m + 1 \leq 2 \ell \leq 2m - 2$
  \end{itemize}

  There are four possible shortest paths for $z_2$ to the survivor:

  \begin{figure}
    \centering
    \includegraphics[scale=0.45]{q_m_n/z2_path_Pa}
    \includegraphics[scale=0.45]{q_m_n/z2_path_Pb}

    \includegraphics[scale=0.45]{q_m_n/z2_path_Pc}
    \includegraphics[scale=0.45]{q_m_n/z2_path_Pd}
    \caption{Four possible outcomes \label{fig:different_paths}}
  \end{figure}

  \begin{itemize}
   \item $P_a$ of length $\Delta + (m - \ell)$
   \item $P_b$ of length $\Delta + 1 + \ell$
   \item $P_c$ of length $(n-\Delta) + 1 + (m-\ell)$
   \item $P_d$ of length $(n-\Delta) + \ell$
  \end{itemize}

  Comparing path lengths we see that:

  \begin{itemize}
   \item[I.] $z_2$ moves counter-clockwise if either $|P_a| \leq \min \{ |P_c|, |P_d| \}$ or $|P_b| \leq \min \{ |P_c|, |P_d| \}$.

   \item[II.] $z_2$ goes clockwise if either $|P_c| \leq \min \{ |P_a|, |P_b| \}$ or $|P_d| \leq \min \{ |P_a|, |P_b| \}$.
  \end{itemize}

  We will examine all combinations of the possible decisions
  made by the zombies from this configuration:

  \begin{itemize}
   \item I. $z_2$ goes counter-clockwise
   \item II. $z_2$ goes clockwise.
   \item A. $z_1$ goes clockwise
   \item B. $z_1$ goes counter-clockwise
  \end{itemize}

  \textit{Case I.A} We have the following constraint on $\ell$ from
  assumption A.
  \begin{align*}
   4 \leq 2 \ell \leq m + 1
  \end{align*}
  and the following constraints on $\Delta$ from assumption I.
  \begin{align*}
   \Delta + (m - \ell) \leq & n - \Delta + 1 + m - \ell & \text{and} \\
   \Delta + (m - \ell) \leq & n - \Delta + \ell
  \end{align*}
  \begin{center}or\end{center}
  \begin{align*}
   \Delta + 1 + \ell \leq & n - \Delta + 1 + m - \ell & \text{and} \\
   \Delta + 1 + \ell \leq & n - \Delta + \ell
  \end{align*}
  These can be simplified with a bit of algebra and assumption A:
  \begin{align*}
   2 \Delta \leq & n+1                    & \text{and} \\
   2 \Delta \leq & n - m + 2\ell \leq n+1
  \end{align*}
  \begin{center}or\end{center}
  \begin{align*}
   2 \Delta \leq & n+m -2 \ell             & \text{and} \\
   2 \Delta \leq & n -1 \leq n + m - 2\ell
  \end{align*}
  So for $z_2$ to follow either $P_a$ or $P_b$ and go counter-clockwise we must have
  \begin{align*}
   2 \Delta \leq & n - m + 2\ell & \text{or} \\
   2 \Delta \leq & n - 1                     \\
  \end{align*}

  Next we consider: which of $s$ or $z_2$ reaches $u$ first?
  If $\Delta = m - \ell$ both $z_2$ and $s$ reach $u$ on the same round,
  with the survivor moving onto the zombie-occupied vertex (and losing).
  If we have $\Delta = m - \ell + 1$, then $s$ reaches $u$ first
  but is caught by $z_2$ on the following round.
  So, to guarantee the survivor has not escaped $P_m$ we need
  \begin{align*}
   \Delta \leq & m- \ell + 1 \\
  \end{align*}
  otherwise the survivor can reach the chord at least two rounds
  before $z_2$ can move to block. We wish to prevent this scenario since
  the survivor could then take the chord and possibly escape, pulling
  both zombies into a loop either on $C_{m}$ or $C_{n}$.
  This constraint on $\Delta$ guarantees that the survivor cannot
  escape $C_m$ before $z_2$'s arrival in Case I.A.

  That is not sufficient, however. We must also ensure that $z_2$ moves
  counter-clockwise (opposite to $z_1$) once it reaches $u$ in order to trap the
  survivor. So we need
  \begin{align*}
   m - \ell - \Delta \leq & 1 + \Delta + \ell \\
\end{align*}
Or, in terms of $\Delta$,
\begin{align*}
   2 \Delta \geq          & m - 2\ell  -1     \\
  \end{align*}

  When we combine all the restrictions we obtain

  \textit{Case I.A. Summary}

  $z_1$ goes clockwise:
  \begin{align*}
   4 \leq & 2 \ell \leq m + 1
  \end{align*}
  and $z_2$ goes counter-clockwise
  \begin{align*}
   2 \Delta \leq & n - m + 2\ell & \text{or} \\
   2 \Delta \leq & n - 1                     \\
  \end{align*}
  the zombies win:
  \begin{align*}
   2 \Delta \leq      & 2 m- 2 \ell + 2 & \text{and} \\
   m - 2\ell  -1 \leq & 2 \Delta                     \\
  \end{align*}

  \textit{Case I.B}
  From assumption B and the constraint on $\ell$, we must have
  \begin{align*}
   m + 1 \leq 2 \ell \leq 2m - 2
  \end{align*}
  and the constraints on $\Delta$ from assumption I are again:
  \begin{align*}
   \Delta + (m - \ell) \leq & n - \Delta + 1 + m - \ell & \text{and} \\
   \Delta + (m - \ell) \leq & n - \Delta + \ell
  \end{align*}
  \begin{center}or\end{center}
  \begin{align*}
   \Delta + 1 + \ell \leq & n - \Delta + 1 + m - \ell & \text{and} \\
   \Delta + 1 + \ell \leq & n - \Delta + \ell
  \end{align*}

  These can be simplified using assumption B:
  \begin{align*}
   2 \Delta \leq & n+1 \leq n-m+2\ell & \text{and} \\
   2 \Delta \leq & n - m + 2\ell
  \end{align*}
  or
  \begin{align*}
   2 \Delta \leq & n+m -2 \ell \leq n-1 & \text{and} \\
   2 \Delta \leq & n -1
  \end{align*}

  So for $z_2$ to go counter-clockwise in this case we must have
  \begin{align*}
   2 \Delta \leq & n + 1         & \text{or} \\
   2 \Delta \leq & n + m - 2\ell
  \end{align*}

  Again we must consider who reaches the chord first. We have assumed that $z_1$
  is going counter-clockwise. If $\ell = \Delta$, then $z_2$ reaches $u$ and $s$
  reaches $v$ on the same round, and therefore $s$ will be caught on the next.
  Therefore, to guarantee the survivor has not escaped $P_m$ in this scenario we need
  \begin{align*}
   \Delta \leq & \ell
  \end{align*}
  otherwise the survivor reaches the chord before $z_2$ and could escape.

  Then, to ensure that $z_2$ traps the survivor by going clockwise once
  it reaches $u$ we need
  \begin{align*}
   1 + \ell - \Delta \leq & \Delta -1 + m - \ell + 1 \\
   2\ell - m + 1 \leq     & 2 \Delta
  \end{align*}

  \textit{Case I.B. Summary}

  $z_1$ goes counter-clockwise:
  \begin{align*}
   m + 1 \leq 2 \ell \leq 2m - 2
  \end{align*}
  and $z_2$ goes counter-clockwise
  \begin{align*}
   2 \Delta \leq & n + 1         & \text{or} \\
   2 \Delta \leq & n + m - 2\ell
  \end{align*}
  the zombies win:
  \begin{align*}
   2 \Delta \leq      & 2 \ell   \\
   2\ell - m + 1 \leq & 2 \Delta
  \end{align*}

  \textit{Case II.A} We have the following constraint on $\ell$ from
  assumption A.
  \begin{align*}
   4 \leq 2 \ell \leq m + 1
  \end{align*}
  and the following constraints on $\Delta$ from assumption II.
  \begin{align*}
   n - \Delta + \ell \leq & \Delta + (m - \ell) & \text{and} \\
   n - \Delta + \ell \leq & \Delta + 1 + \ell
  \end{align*}
  \begin{center}or\end{center}
  \begin{align*}
   n - \Delta + 1 + m - \ell \leq & \Delta + (m - \ell) & \text{and} \\
   n - \Delta + 1 + m - \ell \leq & \Delta + 1 + \ell
  \end{align*}
  These can be simplified with a bit of algebra:
  \begin{align*}
   n-m +2\ell \leq & 2 \Delta & \text{and} \\
   n-1 \leq        & 2\Delta
  \end{align*}
  \begin{center}or\end{center}
  \begin{align*}
   n + 1 \leq         & 2 \Delta & \text{and} \\
   n + m - 2\ell \leq & 2 \Delta
  \end{align*}

  These inequalites are of the form
  \begin{align*}
   n-x \leq & 2 \Delta & \text{and} \\
   n-1 \leq & 2\Delta
  \end{align*}
  \begin{center}or\end{center}
  \begin{align*}
   n + x \leq & 2 \Delta & \text{and} \\
   n + 1 \leq & 2 \Delta
  \end{align*}

  Where $x = m -2\ell$.

  Supposing $x\geq 0$, we have
  \begin{align*}
   n-x \leq & n+x \leq 2 \Delta & \text{and} \\
   n-1 < & n+1 \leq 2 \Delta
  \end{align*}
  and take the lowest bounds because of the disjunction, so that
   $2\Delta \geq n-x = n-m+2\ell$ and $2\Delta \geq n-1$ suffices.

  Since assumption A gives $m-2\ell \geq -1$, supposing $x < 0$ reduces the inequalites to
  \begin{align*}
   n+1 \leq & 2 \Delta & \text{and} \\
   n-1 \leq & 2 \Delta
  \end{align*}
  which is satisfied by $2\Delta \geq n-x = n-m+2\ell$ and $2\Delta \geq n-1$.

  Thus $z_2$ will go clockwise under assumption A if
  \begin{align*}
   2\Delta \geq & n-m+2\ell & \text{and} \\
    2\Delta \geq & n-1
  \end{align*}

  We have assumed that $z_1$ is going clockwise. If $m - \ell = n - \Delta$,
  then $z_2$ reaches $v$ and $s$ reaches $u$ on the same round and $s$
  will be caught on the next. Therefore, to guarantee the survivor has not
  escaped $P_m$ in this scenario we need
  \begin{align*}
   n - \Delta \leq & m - \ell     \\
   \Delta \geq     & n - m + \ell
  \end{align*}
  otherwise the survivor could reach the chord before $z_2$.

  After $n-\Delta$ rounds, we have (insert diagram)

  Then, to ensure that $z_2$ goes counter-clockwise once
  it reaches $v$, we need
  \begin{align*}
   1 + m - \ell - (n - \Delta) \leq & n - \Delta + \ell  \\
   2 \Delta \leq                    & 2n + 2\ell - m - 1
  \end{align*}


  All together this gives
  \textit{Case II.A. Summary}

  $z_1$ goes clockwise:
  \begin{align*}
   4 \leq 2 \ell \leq m + 1
  \end{align*}
  and $z_2$ goes clockwise
  \begin{align*}
   n -m + 2\ell \leq & 2 \Delta & \text{and} \\
   n-1 \leq          & 2 \Delta
  \end{align*}
  the zombies win:
  \begin{align*}
   2 \Delta \geq & 2n - 2m + 2\ell    \\
   2 \Delta \leq & 2n + 2\ell - m - 1
  \end{align*}

  \textit{Case II.B}  We have the following constraint on $\ell$ from
  assumption B.
  \begin{align*}
   m + 1 \leq 2 \ell \leq 2m - 2
  \end{align*}
  and the following constraints on $\Delta$ from assumption II.
  \begin{align*}
   n - \Delta + \ell \leq & \Delta + (m - \ell) & \text{and} \\
   n - \Delta + \ell \leq & \Delta + 1 + \ell
  \end{align*}
  \begin{center}or\end{center}
  \begin{align*}
   n - \Delta + 1 + m - \ell \leq & \Delta + (m - \ell) & \text{and} \\
   n - \Delta + 1 + m - \ell \leq & \Delta + 1 + \ell
  \end{align*}
  These can be simplified further with a bit of algebra:
  \begin{align*}
   n-m+2\ell \leq & 2 \Delta & \text{and} \\
   n-1 \leq       & 2\Delta
  \end{align*}
  or
  \begin{align*}
   n+1 \leq        & 2 \Delta & \text{and} \\
   n+m-2\ell  \leq & 2 \Delta
  \end{align*}

  We have
  \begin{align*}
   n - \Delta + \ell \leq & \Delta + (m - \ell) & \text{and} \\
   n - \Delta + \ell \leq & \Delta + 1 + \ell
  \end{align*}
  \begin{center}or\end{center}
  \begin{align*}
   n - \Delta + 1 + m - \ell \leq & \Delta + (m - \ell) & \text{and} \\
   n - \Delta + 1 + m - \ell \leq & \Delta + 1 + \ell
  \end{align*}
  These can be simplified further with a bit of algebra:
  \begin{align*}
   n-m+2\ell \leq & 2 \Delta & \text{and} \\
   n-1 \leq       & 2\Delta
  \end{align*}
  or
  \begin{align*}
   n+1 \leq        & 2 \Delta & \text{and} \\
   n+m-2\ell  \leq & 2 \Delta
  \end{align*}

  These inequalites are of the form
  \begin{align*}
   n-x \leq & 2 \Delta & \text{and} \\
   n-1 \leq & 2\Delta
  \end{align*}
  \begin{center}or\end{center}
  \begin{align*}
   n + 1 \leq & 2 \Delta & \text{and} \\
   n + x \leq & 2 \Delta
  \end{align*}

  Where $x = m -2\ell$. Now since assumption B gives $m - 2\ell \leq -1$, we
  see that
  \begin{align*}
   n-1 \leq & n-x \leq 2 \Delta \\
            & \text{or}         \\
   n+x \leq & n+1 \leq 2 \Delta
  \end{align*}

  Now we consider: which of $s$ or $z_2$ reaches $v$ first?
  If $n - \Delta = \ell$, then they both reach $u$ at the same time,
  with the survivor moving onto the $z_2$-occupied vertex (and losing).
  If we have $n - \Delta = \ell + 1$, then $s$ reaches $u$ first
  but is caught by $z_2$ on the following round.
  So, to guarantee the survivor has not escaped $P_m$ we need
  \begin{align*}
   n - \Delta \leq & \ell + 1 \\
  \end{align*}
  otherwise the survivor reaches the chord before $z_2$ can move
  to block. If the survivor reaches the chord first, then it could
  take the chord and possibly escape. (more detail??)

  Then, to ensure that $z_2$ takesgoes clockwise once
  it reaches $v$, we need
  \begin{align*}
   \ell - (n - \Delta) \leq & 1 + (n - \Delta - 1) + (m - \ell + 1) \\
   2 \Delta \leq            & 2n + m - 2\ell + 1
  \end{align*}

  \textit{Case II.B. Summary}

  $z_1$ goes counter-clockwise:
  \begin{align*}
   m + 1 \leq 2 \ell \leq 2m - 2
  \end{align*}
  and $z_2$ goes clockwise
  \begin{align*}
   n+1 \leq & 2 \Delta
  \end{align*}
  the zombies win:
  \begin{align*}
   n - \Delta \leq & \ell + 1           \\
   2 \Delta \leq   & 2n + m - 2\ell + 1
  \end{align*}


 \end{proofpart}

\subsection{Forcing the Survivor into a Losing Position}
 \begin{proofpart}
  We now consider the game on this graph in general and show
  how we can guarantee the survivor will be caught.

  Given $m, n$ and $\Delta$ as computed below, we place the
  zombies on $C_{n+1}$ so that the zombies move in
  opposite direction wherever the survivor may start.
  We need only consider the cycle $C_{n+1}$ since, if the survivor
  starts on $C_{m+1} \setminus \{u, v\}$, then the zombies play as
  though the survivor is on $u$ or $v$.

  We choose $k$ such that positioning
  \begin{enumerate}
   \item $z_2$ at $\Delta + k$ clockwise from $u$
   \item $z_1$ at $k$ counter-clockwise from $v$
  \end{enumerate}
  forces the survivor into a losing position: it is either immediately sandwiched on $C_{n+1}$,
  or falls into the trap described above on $C_{m+1}$.

  The survivor cannot start next to the zombies else it loses right away.
  So we choose $k$ such that, even if the survivor is as far
  away from one of the zombies as possible on $C_n$, then the zombies
  still move in opposite directions. This leads to the following inequalities

  \begin{align*}
   n - \Delta - 2k - 2 \leq & \Delta + k +1 + k +2 & \text{and} \\
   \Delta + 2k -1 \leq      & n - \Delta -2k +2x`
  \end{align*}
  Solving for $k$ gives
  \[ n - 2\Delta -5 \leq 4k \leq n-2\Delta +3 \]

  Such $k$ guarantees that the zombies start on vertices such that they must
  move in opposite directions if the survivor plays on $C_n$.

  If the survivor starts between the zombies such that
  access to the chord is blocked, then clearly it has lost. Otherwise,
   the zombies must move towards the chord and in $k$ rounds we reach the
    scenario described in Part 1 when $z_1$ reaches the chord and $z_2$ is
    $\Delta$ away. With suitable $\Delta$, then, the survivor cannot win.
 \end{proofpart}

 \begin{proofpart} Computing the Winning Zombie Start

  Given $m$ and $n$, we choose $\Delta$ so that whenever we reach
  the scenario described in the first part, the survivor will be cornered.
  Such $\Delta$ must satisfy the following constraints for any possible
  value of $\ell$.

  \textit{Case I.A. Summary}

  $z_1$ goes clockwise:
  \begin{align*}
   4 \leq & 2 \ell \leq m + 1
  \end{align*}
  and $z_2$ goes counter-clockwise
  \begin{align*}
   2 \Delta \leq & n - m + 2\ell & \text{or} \\
   2 \Delta \leq & n - 1                     \\
  \end{align*}
  the zombies win:
  \begin{align*}
   2 \Delta \leq      & 2 m- 2 \ell + 2 & \text{and} \\
   m - 2\ell  -1 \leq & 2 \Delta                     \\
  \end{align*}
  \textit{Case I.B. Summary}

  $z_1$ goes counter-clockwise:
  \begin{align*}
   m + 1 \leq 2 \ell \leq 2m - 2
  \end{align*}
  and $z_2$ goes counter-clockwise
  \begin{align*}
   2 \Delta \leq & n + 1         & \text{or} \\
   2 \Delta \leq & n + m - 2\ell
  \end{align*}
  the zombies win:
  \begin{align*}
   2 \Delta \leq      & 2 \ell   \\
   2\ell - m + 1 \leq & 2 \Delta
  \end{align*}
  \textit{Case II.A. Summary}

  $z_1$ goes clockwise:
  \begin{align*}
   4 \leq 2 \ell \leq m + 1
  \end{align*}
  and $z_2$ goes clockwise
  \begin{align*}
   n -m + 2\ell \leq & 2 \Delta & \text{and} \\
   n-1 \leq          & 2 \Delta
  \end{align*}
  the zombies win:
  \begin{align*}
   2 \Delta \geq & 2n - 2m + 2\ell    \\
   2 \Delta \leq & 2n + 2\ell - m - 1
  \end{align*}
  \textit{Case II.B. Summary}

  $z_1$ goes counter-clockwise:
  \begin{align*}
   m + 1 \leq 2 \ell \leq 2m - 2
  \end{align*}
  and $z_2$ goes clockwise
  \begin{align*}
   n+1 \leq & 2 \Delta
  \end{align*}
  the zombies win:
  \begin{align*}
   n - \Delta \leq & \ell + 1           \\
   2 \Delta \leq   & 2n + m - 2\ell + 1
  \end{align*}

A simple algorithm to calculate possible values of $\Delta$ loops
over $0 \leq \Delta \leq n$ and over $2 \leq \ell \leq m-1$ and tests,
for each $\Delta$ and each $\ell$, to determine which of the four cases is
applicable and, if in one of the cases, whether the zombie-win constrains are
satisfied. A value of $\Delta$ is accepted if, for every value of $\ell$, the
zombies win.

Once we have obtained possible $\Delta$, we can then determine $k$ by
calculating the bounds
\[ n - 2\Delta -5 \leq 4k \leq n-2\Delta +3 \]


 \end{proofpart}
\end{proof}

\newpage
\subsection{Existence of Winning Start}

We wish to show that, for any $m, n$, there exist $\Delta$ and $k$
which guarantee the survivor is caught.
First we show that $\Delta = \lfloor \frac{m}{2} \rfloor$ always
works for the cornering strategy.

Note that
\[
2\Delta = 2 \left\lfloor \frac{m}{2} \right\rfloor =
\begin{cases}
m & \text{if $m$ is even} \\
m -1 & \text{if $m$ is odd}
\end{cases}
\]
and so $m -1 \leq 2 \lfloor \frac{m}{2} \rfloor \leq m$.

Suppose that we are in Case I. A. and $\Delta = \lfloor \frac{m}{2} \rfloor$.
Case I. A is characterized by the following constraints:

\[ 4 \leq 2 \ell \leq m+1 \]
\begin{center}and\end{center}
\[ 2\Delta \leq n - m + 2\ell \]
\begin{center}or\end{center}
\[ 2\Delta \leq n-1 \]

The zombies win if

\begin{align*}
 2 \Delta \leq      & 2 m- 2 \ell + 2 & \text{and} \\
 m - 2\ell  -1 \leq & 2 \Delta                     \\
\end{align*}

So if we are in Case I. A. and $\Delta = \left\lfloor \frac{m}{2} \right\rfloor$ the zombies win since

\begin{align*}
  2 \Delta = 2 \lfloor \frac{m}{2} \rfloor \leq m &< 2m - (m+1) + 2\leq 2 m- 2 \ell + 2 & \text{and}\\
  m - 2\ell -1 \leq m - 5 &< 2 \lfloor \frac{m}{2} \rfloor = 2 \Delta
\end{align*}

Which shows that the zombie-win requirements are met.

Suppose now that we are not in Case 1. A. Negating the constraints of Case I. A. gives

\[ 2\Delta \geq n - m + 2\ell +1 \]
\begin{center}and\end{center}
\[ 2 \Delta \geq n - 1 + 1 \]
\begin{center}or\end{center}
\[ m+1 \leq 2 \ell \leq 2m -2 \]

If we assume that $m$ is odd and $ 2 \Delta \geq n$ then we obtain a
contradiction since
\[ 2 \Delta = 2 \lfloor \frac{m}{2} \rfloor = m -1 \geq n \]
and we have assumed that $m \leq n$.

If $m$ even, $m = n$ and $2\Delta \geq n - m + 2\ell +1$ then

\begin{align*}
  2 \Delta & \geq n - m + 2\ell + 1 \\
  m & \geq m - m + 2 \ell + 1 \\
  m & \geq 2 \ell + 1 \\
  2 & \ell \leq m -1
\end{align*}

So, if $m = n$ and they are even, then we are in Case 1. A unless $2 \ell \leq m -1$.

To recap: If we set $\Delta = \lfloor \frac{m}{2} \rfloor$, we are in Case 1.A unless
\[ m = n \qquad \text{and they are even} \]
\[ \Delta = \lfloor \frac{m}{2} \rfloor = \frac{m}{2} \]
\[ 4 \leq 2 \ell \leq m -1 \]

Now, can we be in Case 1. B? Case 1. B is described by the following constraints:

\[ m+1 \leq 2 \ell \leq 2m -2 \]
and
\[2 \Delta \leq n +1 \]
or
\[2\Delta \leq n + m - 2 \ell \]

The negation of which is:
\[2 \Delta \geq n +1 +1 \]
and
\[2\Delta \geq n + m - 2 \ell + 1\]
or
\[4 \leq 2 \ell \leq m +1 \]

But this leads to the contradiction:
\[ n \geq m \geq 2 \Delta \geq n +2 \]

In remains to check if we win in Case 2. A.

Assuming still that
\[ m = n \qquad \text{they are even} \]
\[ \Delta = \frac{m}{2} \]
\[ 4 \leq 2 \ell \leq m -1 \]

The win conditions require
\begin{align*}
2n - 2m + 2\ell \leq 2 \Delta \leq 2n + 2\ell -m -1 \\
2m - 2m + m - 1 \leq 2 \Delta \leq 2m + 4 - m - 1 \\
m - 1 \leq 2 \Delta \leq m + 3
\end{align*}

Which holds for $\Delta = \frac{m}{2}$.
