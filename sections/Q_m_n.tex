\chapter{Cycle With One Chord}\label{chapter q_m_n}

Games played on cycles are straightforward: by Lemma~\ref{lemma subpath} if the zombies are too close to each other, the survivor can lead the zombies in the same direction around the cycle. Otherwise, the zombies are too far apart and whichever side (sub-path of the cycle with the zombies at the end) the survivor may choose, the zombies will move in opposite directions and win. In this Chapter, we investigate the game on cycles augmented by a single chord.

\begin{definition}
 Let $m, n \in \Z$ with $2\leq m \leq n$. Consider a cycle of length $m+n$ and add a chord which
 divides the cycle into paths $P_m$ and $P_n$ of lengths $m$ and $n$, respectively.
 We denote such a graph as $Q_{m,n}$. Let $u$ and $v$ denote the endpoints of the chord. We refer to the subcycles of lengths $m+1$ and $n+1$ formed by paths and the chord as $C_{m+1} = Q_{m,n}[P_m]$ and $C_{n+1} = Q_{m,n}[P_n]$, respectively.
\end{definition}

\begin{figure}
  \centering
 \includegraphics{q_m_n/Q_7_8}
 \caption{$Q_{7,8}$ \label{fig:Q_7_8}}
\end{figure}

See Figure~\ref{fig:Q_7_8} for an illustration of $Q_{7,8}$.
The construction contains three sub-cycles which the survivor could use to fool the zombies.
Let us first examine the construction for small values of $m$ and $n$.

 \begin{lemma}
The zombie-number $z(Q_{2,2}) = 1$.
 \end{lemma}
\begin{proof}
  Setting $m=n=2$ yields a graph with two adjacent cliques $K_3$, which are dominated by a single vertex so it is zombie-win.
\end{proof}

\begin{lemma}
  Let $n \in \Z$ with $n \geq 3$. Then $z(Q_{2,n}) = 2$.
\end{lemma}
\begin{proof}
  For $m=2$ and $n\geq 3$, two zombies win by starting on diametrically
  opposed vertices on the cycle $C_{n+1}$. The additional edge has no impact on a two-zombie cycle strategy. A single zombie does not suffice because of the cycle of length at least 4.
\end{proof}

\begin{lemma}
Let $m,n \in \Z$ with $3 \leq m \leq n \leq 5$. Then $z(Q_{m,n}) = 2$.
\end{lemma}
\begin{proof}
   For $Q_{m,n}$ with $3 \leq m \leq n \leq 5$, placing the zombies on the endpoints
  of the chord divides the graph into two cycles of length at most 5 which can be guarded by two adjacent vertices. A single zombie does not suffice because of the cycle of length at least 4.
\end{proof}

For larger values of $m$ and $n$ the outcome is not as clear. Unfortunately for the survivor, we are able to show the existence of starting positions for the zombies (obtained as a function of $m, n$) which limits the survivor's options and prevents the zombies from being led in the same direction.

\begin{theorem}
 Let $m, n \in Z$ with $3 \leq m \leq n$. The zombie number of $Q_{m,n}$ is 2.
\end{theorem}

We imagine $Q_{m,n}$ as embedded in the plane with $P_m$
-- the shortest side -- on the left.
This does not limit the generality of the following and allows us to define
(counter-)clockwise distance: the length of the path
along a cycle with respect to the given direction on this embedding.

\begin{proof}
  First, observe that one zombie will not suffice for any graph containing an isometric subcycle of length at least four.
  Second, the lemmas above show the statement to be true for $3\leq m \leq n \leq 5$, so for the remainder of the proof we assume that $m \geq 3$ and $n \geq 6$.

  We seek a winning starting position for the zombies for $m \geq 3$ and $n \geq 6$.
  We describe a strategy in three separate parts, which we summarize here.

  First we show how to position the zombies to guarantee a win assuming the survivor starts on $P_m$.
  We will position one of the zombies on an endpoint of the chord and another at some distance  $\Delta$ from the other endpoint. We calculate the values of $\Delta$ which guarantee the survivor will be sandwiched on $P_m$ by considering all possible combinations of directions ``chosen'' by the zombies (refer to Part~\ref{thm q_m_n 1}). The zombies' choice of direction is not really a choice, after all: the choice is forced by the position of the survivor and the length of the possible zombie-survivor paths.

  Next, we show how to position the zombies at the start of the game so that
  no matter where the survivor starts a losing position is guaranteed:
  we offset the zombies on the larger cycle with an additional parameter $k$, which ensures the zombies are not too close together and therefore guard $C_{n+1}$ (refer to Part~\ref{thm q_m_n 2}). After $k$ rounds, the survivor will have no choice but to retreat to the smaller cycle and fall into the carefully   orchestrated trap described in the first part of the proof.

  In Part~\ref{thm q_m_n 3}, we show that such a starting position is available to the zombies
  for any $m \geq 3$, $n \geq 6$. Finally in Part~\ref{thm q_m_n 4} we use these results to give winning $\Delta$ and $k$ for any fixed $m,n$ values.

  \section{Cornering the Survivor on $C_{m+1}$\label{thm q_m_n 1}}

As mentioned, let $u$ and $v$ denote the endpoints of the chord. Let $z_1, z_2$ and $s$ denote the positions of the zombies and the survivor, respectively.
\begin{proofpart}
   Suppose that the game has reached the following state:

  \begin{itemize}
   \item the first zombie $z_1$ is occupying $v$, one endpoint of the chord.
   \item there are $\Delta$ vertices counting clockwise from $z_2$ to $u$ (the other endpoint of the chord).
   \item the survivor is on $P_m$ at a distance of $\ell$ vertices counting
   clockwise from $v$.
  \end{itemize}

  This configuration is illustrated in Figure~\ref{fig:onthechord}.
  Note that we must have
  \[2 \leq \ell \leq m-1 \tag{1} \label{bounds on ell}\]
  else $z_1$ captures the survivor on the next round.

  \begin{figure}
    \centering
    \includegraphics{q_m_n/diagram1}
    \caption{$z_1$ on $v$, $s$ somewhere on $P_m$ \label{fig:onthechord}}
  \end{figure}

  By comparing the lengths of different paths, we calculate the values of $\Delta$
  which guarantees that the survivor will be cornered on $P_m$ from this start configuration. That is to say, the survivor will not be able to return to any of the endpoints of the chord before $z_2$.

  We can assume that once $z_1$ chooses a direction from $v$, it continues in that direction: either the zombie has no choice or both directions around the cycle are of the same length (and so
  $z_1$ may continue in the same direction).

  We can also assume that on its turn the survivor will move away from
  $z_1$ and maintain a distance of $\ell$ (or $m-\ell +1$, if they are moving counter-clockwise)
  since a winning survivor strategy which involves waiting a turn
  or moving backwards is equivalent to a survivor strategy
  which always moves but starts with a smaller (or larger) value of $\ell$.

  These two assumptions allow us to ``fast-forward'' the game by $\Delta$ rounds (or $n-\Delta$ rounds) and determine when the survivor is captured. Since $z_1$ is already on the same sub-cycle as the survivor, there are two possibilities:

  \begin{itemize}
   \item[A.] $z_1$ goes clockwise if $\ell \leq 1 + m - \ell$.
         Combined with \ref{bounds on ell}, we have
         \begin{equation}
           4 \leq 2 \ell \leq m + 1 \tag{A} \label{assumption a} \mperiod
         \end{equation}

   \item[B.] $z_1$ goes counter-clockwise if $1 + m - \ell \leq \ell$.
         Combined \ref{bounds on ell}, we obtain
         \begin{equation}
           m + 1 \leq 2 \ell \leq 2m - 2 \tag{B} \label{assumption b} \mperiod
         \end{equation}

  \end{itemize}

  We must consider four possible paths from $z_2$ to the survivor:

  \begin{itemize}
   \item $P_a$ of length $\Delta + (m - \ell)$,
   \item $P_b$ of length $\Delta + 1 + \ell$,
   \item $P_c$ of length $(n-\Delta) + 1 + (m-\ell)$, and
   \item $P_d$ of length $(n-\Delta) + \ell$.
  \end{itemize}

  These paths are illustrated in Figure~\ref{fig:different_paths}.

  \begin{figure}
    \centering
    \includegraphics{q_m_n/z2_path_Pa}

    \includegraphics{q_m_n/z2_path_Pb}

    \includegraphics{q_m_n/z2_path_Pc}

    \includegraphics{q_m_n/z2_path_Pd}

    \caption{Possible paths from $z_2$ to $s$ \label{fig:different_paths}}
  \end{figure}

  Comparing path lengths we see that:

  $z_2$ moves counter-clockwise if either
   \[ |P_a| \leq \min \{ |P_c|, |P_d| \}\qquad \text{or} \qquad |P_b| \leq \min \{ |P_c|, |P_d| \} \tag{I} \label{assumption i} \mperiod \]

  $z_2$ goes clockwise if either
   \[ |P_c| \leq \min \{ |P_a|, |P_b| \}\qquad \text{or} \qquad |P_d| \leq \min \{ |P_a|, |P_b| \} \tag{II} \label{assumption ii}\mperiod \]

  We will examine all combinations of these possible ``zombie-decisions'' to show that there exist values of $\Delta$ which prevent the survivor's escape in any of the possible games (from this start configuration where the survivor is on $P_m$). We break it down as follows:

  \begin{itemize}
   \item I. $z_2$ goes counter-clockwise
   \item II. $z_2$ goes clockwise.
   \item A. $z_1$ goes clockwise
   \item B. $z_1$ goes counter-clockwise
  \end{itemize}

\begin{itemize}
  \item \textit{Case I.A.} $z_2$ goes counter-clockwise and $z_1$ goes clockwise.

  Suppose the zombies will move as in Figure~\ref{fig:case_1_a_1}.
  \begin{figure}
    \centering
    \includegraphics{q_m_n/case_1_a_1}
    \caption{Case I.A. \label{fig:case_1_a_1}}
  \end{figure}

  We obtain the following constraints on $\ell$ from \ref{assumption a}
  \[ 4 \leq 2 \ell \leq m + 1 \]
  and the following constraints on $\Delta$ from \ref{assumption i}
  \begin{align*}
   \Delta + (m - \ell) &\leq n - \Delta + 1 + m - \ell && \text{and} \\
   \Delta + (m - \ell) &\leq n - \Delta + \ell
  \end{align*}
  \begin{center}or\end{center}
  \begin{align*}
   \Delta + 1 + \ell &\leq n - \Delta + 1 + m - \ell && \text{and} \\
   \Delta + 1 + \ell &\leq n - \Delta + \ell \mperiod
  \end{align*}
  Combining with \ref{assumption a} we can obtain:
  \begin{align*}
   2 \Delta &\leq n+1 && \text{and} \\
   2 \Delta &\leq n - m + 2\ell \leq n+1
  \end{align*}
  \begin{center}or\end{center}
  \begin{align*}
   2 \Delta &\leq n+m -2 \ell && \text{and} \\
   2 \Delta &\leq n -1 \leq n + m - 2\ell
  \end{align*}
  So for $z_2$ to follow either $P_a$ or $P_b$ and go counter-clockwise we must have
  \begin{align*}
   2 \Delta &\leq n - m + 2\ell && \text{or} \\
   2 \Delta &\leq n - 1 \mperiod
  \end{align*}

  We must determine which of $s$ or $z_2$ reaches $u$ first. Consider the game after $\Delta$ rounds, as illustrated in Figure~\ref{fig:case_1_a_2}.

  \begin{figure}
    \centering
    \includegraphics{q_m_n/case_1_a_2}
    \caption{Case I.A. after $\Delta$ rounds \label{fig:case_1_a_2}}
  \end{figure}

  If $\Delta = m - \ell$ both $z_2$ and $s$ reach $u$ on the same round,
  with the survivor moving onto the zombie-occupied vertex (and losing).
  If we have $\Delta = m - \ell + 1$, then $s$ reaches $u$ first
  but is caught by $z_2$ on the following round.
  So, to guarantee the survivor has not escaped $P_m$ we need
  \[ \Delta \leq m- \ell + 1 \]
  otherwise the survivor can reach the chord at least two rounds
  before $z_2$ can move to block. We wish to prevent this scenario since
  the survivor could then take the chord and possibly escape, pulling
  both zombies into a loop either on $C_{m+1}$ or $C_{n+1}$.

  That is not sufficient, however. We must also ensure that $z_2$ moves
  counter-clockwise (opposite to $z_1$) once it reaches $u$ in order to trap the
  survivor. So we need
  \[ m - \ell - \Delta \leq 1 + \Delta + \ell \]
  Or, in terms of $\Delta$,
  \[ 2 \Delta \geq m - 2\ell  -1 \mperiod \]

  When we combine all the restrictions we obtain the following characterization for Case I.A.:

  $z_1$ goes clockwise:
  \[ 4 \leq 2 \ell \leq m + 1 \]
  and $z_2$ goes counter-clockwise:
  \[ 2 \Delta \leq n - m + 2\ell \qquad \text{or} \qquad 2 \Delta \leq n - 1 \mperiod \]
  The zombies win:
  \[ 2 \Delta \leq 2 m- 2 \ell + 2 \qquad \text{and} \qquad m - 2\ell  -1 \leq 2 \Delta \mperiod \]

  \item \textit{Case I.B.} $z_2$ and $z_1$ both go counter-clockwise.

  Suppose the zombies will move as in Figure~\ref{fig:case_1_b_1}.
  \begin{figure}
    \centering
    \includegraphics{q_m_n/case_1_b_1}
    \caption{Case I.B. \label{fig:case_1_b_1}}
  \end{figure}

  From \ref{assumption b} and the constraint on $\ell$, we must have
  \[ m + 1 \leq 2 \ell \leq 2m - 2 \]
  and the constraints on $\Delta$ from \ref{assumption i} are again:
  \begin{align*}
   \Delta + (m - \ell) &\leq n - \Delta + 1 + m - \ell && \text{and} \\
   \Delta + (m - \ell) &\leq n - \Delta + \ell
  \end{align*}
  \begin{center}or\end{center}
  \begin{align*}
   \Delta + 1 + \ell &\leq n - \Delta + 1 + m - \ell && \text{and} \\
   \Delta + 1 + \ell &\leq n - \Delta + \ell \mperiod
  \end{align*}

  Simplifying using \ref{assumption b} yields:
  \begin{align*}
   2 \Delta &\leq n+1 \leq n-m+2\ell && \text{and} \\
   2 \Delta &\leq n - m + 2\ell
  \end{align*}
  \begin{center}or\end{center}
  \begin{align*}
   2 \Delta &\leq n+m -2 \ell \leq n-1 && \text{and} \\
   2 \Delta &\leq n -1 \mperiod
  \end{align*}

  So for $z_2$ to go counter-clockwise in this case we must have
  \begin{align*}
   2 \Delta &\leq n + 1         && \text{or} \\
   2 \Delta &\leq n + m - 2\ell \mperiod
  \end{align*}

  Again we must consider who reaches the chord first. Consider the game after $\Delta$ rounds, as illustrated in Figure~\ref{fig:case_1_b_2}.

  \begin{figure}
    \centering
    \includegraphics{q_m_n/case_1_b_2}
    \caption{Case I.B. after $\Delta$ rounds \label{fig:case_1_b_2}}
  \end{figure}

  If $\ell = \Delta$, then $z_2$ reaches $u$ and $s$ reaches $v$ on the same round, and therefore $s$ will be caught on the next. Therefore, to guarantee the survivor has not escaped $P_m$ in this scenario we need
  \[ \Delta \leq \ell\]
  Otherwise, the survivor reaches the chord before $z_2$ and could escape.

  Then, to ensure that $z_2$ traps the survivor by going clockwise once
  it reaches $u$ we need
  \begin{align*}
   1 + \ell - \Delta &\leq \Delta -1 + m - \ell + 1 && \text{and} \\
   2\ell - m + 1 &\leq 2 \Delta \mperiod
  \end{align*}

  We obtain the following characterization for Case I.B.:

  $z_1$ goes counter-clockwise:
  \[ m + 1 \leq 2 \ell \leq 2m - 2\]
  and $z_2$ goes counter-clockwise
  \[ 2 \Delta \leq n + 1 \qquad \text{or} \qquad 2 \Delta \leq n + m - 2\ell \mperiod\]
  The zombies win:
  \[ 2 \Delta \leq 2 \ell \qquad \text{and} \qquad 2\ell - m + 1 \leq 2 \Delta \mperiod\]

  \item \textit{Case II.A.} $z_2$ and $z_1$ both go clockwise.

  Suppose the zombies will move as in Figure~\ref{fig:case_2_a_1}.
  \begin{figure}
    \centering
    \includegraphics{q_m_n/case_2_a_1}
    \caption{Case II.A. \label{fig:case_2_a_1}}
  \end{figure}

  We have the following constraint on $\ell$ from \ref{assumption b}
  \[ 4 \leq 2 \ell \leq m + 1 \]
  and the following constraints on $\Delta$ from \ref{assumption ii}
  \begin{align*}
   n - \Delta + 1 + m - \ell &\leq \Delta + m - \ell && \text{and} \\
   n - \Delta + 1 + m - \ell &\leq \Delta + 1 + \ell
  \end{align*}
  \begin{center}or\end{center}
    \begin{align*}
     n - \Delta + \ell &\leq \Delta + m - \ell && \text{and} \\
     n - \Delta + \ell &\leq \Delta + 1 + \ell \mperiod
    \end{align*}
  Simplifying with a bit of algebra yields:
  \begin{align*}
   n-m +2\ell &\leq 2 \Delta && \text{and} \\
   n-1 &\leq 2\Delta
  \end{align*}
  \begin{center}or\end{center}
  \begin{align*}
   n + 1 &\leq 2 \Delta && \text{and} \\
   n + m - 2\ell &\leq 2 \Delta \mperiod
  \end{align*}

  These inequalites are of the form
  \begin{align*}
   n-x &\leq 2 \Delta && \text{and} \\
   n-1 &\leq 2\Delta
  \end{align*}
  \begin{center}or\end{center}
  \begin{align*}
   n + x &\leq 2 \Delta && \text{and} \\
   n + 1 &\leq 2 \Delta \mcomma
  \end{align*}
where $x = m -2\ell$.

  Supposing $x\geq 0$, we have
  \begin{align*}
   n-x &\leq n+x \leq 2 \Delta && \text{and} \\
   n-1 &< n+1 \leq 2 \Delta
  \end{align*}
  and take the lowest bounds because of the disjunction, so that
   \[2\Delta \geq n-x = n-m+2\ell \qquad \text{and} \qquad 2\Delta \geq n-1\]
    suffices.

  Since \ref{assumption b} gives $m-2\ell \geq -1$, supposing $x < 0$ reduces the inequalites to
  \begin{align*}
   n+1 &\leq 2 \Delta && \text{and} \\
   n-1 &\leq 2 \Delta
  \end{align*}
  which is satisfied by $2\Delta \geq n-x = n-m+2\ell$ and $2\Delta \geq n-1$.

  Thus $z_2$ will go clockwise under \ref{assumption b} if
  \begin{align*}
   2\Delta &\geq n-m+2\ell && \text{and} \\
    2\Delta &\geq n-1
  \end{align*}

  Consider the game after $n-\Delta$ rounds, as illustrated in Figure~\ref{fig:case_2_a_2}.

  \begin{figure}
    \centering
    \includegraphics{q_m_n/case_2_a_2}
    \caption{Case II.A. after $n-\Delta$ rounds \label{fig:case_2_a_2}}
  \end{figure}

  We have assumed that $z_1$ is going clockwise. If $m - \ell = n - \Delta$,
  then $z_2$ reaches $v$ and $s$ reaches $u$ on the same round and $s$
  will be caught on the next. Therefore, to guarantee the survivor has not
  escaped $P_m$ in this scenario we need
  \begin{align*}
   n - \Delta &\leq m - \ell   && \text{and}  \\
   \Delta &\geq n - m + \ell \mcomma
  \end{align*}
  otherwise the survivor could reach the chord before $z_2$.

  To ensure that $z_2$ goes counter-clockwise once it reaches $v$, we need
  \begin{align*}
   1 + m - \ell - (n - \Delta) &\leq n - \Delta + \ell  \\
   2 \Delta &\leq 2n + 2\ell - m - 1 \mperiod
  \end{align*}

  We obtain the following characterization for Case II.B.:

  $z_1$ goes clockwise:
  \[ 4 \leq 2 \ell \leq m + 1 \mcomma \]
  and $z_2$ goes clockwise
  \[ n -m + 2\ell \leq 2 \Delta \qquad \text{and} \qquad n-1 \leq 2 \Delta \mperiod \]
  The zombies win:
  \[ 2 \Delta \geq 2n - 2m + 2\ell \qquad \text{and} \qquad 2 \Delta \leq 2n + 2\ell - m - 1 \mperiod\]


  \item \textit{Case II.B.} $z_2$ goes clockwise and $z_1$ goes counter-clockwise.

  Suppose the zombies will move as in Figure~\ref{fig:case_2_b_1}.
  \begin{figure}
    \centering
    \includegraphics{q_m_n/case_2_b_1}
    \caption{Case II.B. \label{fig:case_2_b_1}}
  \end{figure}

  We have the following constraint on $\ell$ from \ref{assumption b}
  \begin{align*}
   m + 1 \leq 2 \ell \leq 2m - 2 \mcomma
  \end{align*}
  and the following constraints on $\Delta$ from \ref{assumption ii}
  \begin{align*}
   n - \Delta + 1 + m - \ell &\leq \Delta + m - \ell && \text{and} \\
   n - \Delta + 1 + m - \ell &\leq \Delta + 1 + \ell
  \end{align*}
  \begin{center}or\end{center}
    \begin{align*}
     n - \Delta + \ell &\leq \Delta + m - \ell && \text{and} \\
     n - \Delta + \ell &\leq \Delta + 1 + \ell \mperiod
    \end{align*}

  These can be simplified with a bit of algebra:
  \begin{align*}
   n-m+2\ell &\leq 2 \Delta & \text{and} \\
   n-1 \leq       & 2\Delta
  \end{align*}
  or
  \begin{align*}
   n+1 \leq        & 2 \Delta & \text{and} \\
   n+m-2\ell  &\leq 2 \Delta \mperiod
  \end{align*}

  These inequalities are of the form
  \begin{align*}
   n-x &\leq 2 \Delta & \text{and} \\
   n-1 &\leq 2\Delta
  \end{align*}
  \begin{center}or\end{center}
  \begin{align*}
   n + 1 &\leq 2 \Delta & \text{and} \\
   n + x &\leq 2 \Delta \mcomma
  \end{align*}

  where $x = m -2\ell$. Since \ref{assumption b} gives $m - 2\ell \leq -1$, we
  see that
  \[ n-1 \leq n+1 \leq n-x \leq 2 \Delta \]
  \begin{center}or\end{center}
  \[ n+x \leq n-1 \leq n+1 \leq 2 \Delta \]

  Consider the game after $n-\Delta$ rounds, as illustrated in Figure~\ref{fig:case_2_b_2}.
  \begin{figure}
    \centering
    \includegraphics{q_m_n/case_2_b_2}
    \caption{Case II.B. after $n-\Delta$ rounds \label{fig:case_2_b_2}}
  \end{figure}

  If $n - \Delta = \ell$, then they both reach $u$ at the same time,
  with the survivor moving onto the $z_2$-occupied vertex (and losing).
  If we have $n - \Delta = \ell + 1$, then $s$ reaches $u$ first
  but is caught by $z_2$ on the following round.
  So, to guarantee the survivor has not escaped $P_m$ we need
  \begin{align*}
   n - \Delta &\leq \ell + 1 \mcomma \\
  \end{align*}
  otherwise the survivor reaches the chord before $z_2$ can move
  to block. If the survivor reaches the chord first, then it could
  take the chord and possibly escape.

  To ensure that $z_2$ goes clockwise once it reaches $v$, we need
  \begin{align*}
   \ell - (n - \Delta) &\leq 1 + (n - \Delta - 1) + (m - \ell + 1) \\
   2 \Delta \leq            & 2n + m - 2\ell + 1 \mperiod
  \end{align*}

  We obtain the following characterization for Case II.B.:

  $z_1$ goes counter-clockwise:
  \[ m + 1 \leq 2 \ell \leq 2m - 2 \mcomma \]
  and $z_2$ goes clockwise:
  \[ n+1 \leq 2 \Delta \mperiod\]
  The zombies win:
  \[ n - \Delta \leq \ell + 1 \qquad \text{and} \qquad 2 \Delta \leq 2n + m - 2\ell + 1 \mperiod \]

\end{itemize}

We will show (in Section~\ref{thm q_m_n 3}) that with $\Delta = \left\lfloor \frac{m}{2} \right\rfloor$, the zombies can always (successfully) execute this cornering strategy. Of course, this is not sufficient to show the zombies win: there is no guarantee that the survivor will choose to start along $P_m$ as is assumed here, so we cannot simply start with this zombie configuration. Instead, we must force the survivor's hand.
\end{proofpart}

  \section{Guarding the Largest Cycle $C_{n+1}$\label{thm q_m_n 2}}
\begin{proofpart}
  We consider the game on $Q_{m,n}$ in general and show
  how we can position the zombies on $C_{n+1}$ to limit the survivor's options and thereby guarantee it will be caught.

  %Given $m, n$ and $\Delta$ as computed above, we place the
  %zombies on $C_{n+1}$ so that the zombies move in
  %opposite directions wherever the survivor may start.
  %We need only consider the cycle $C_{n+1}$ since, if the survivor
  %starts on $C_{m+1} \setminus \{u, v\}$, then the zombies play as
  %though the survivor is on $u$ or $v$.

  Choose $k$ such that positioning
  \begin{enumerate}
   \item $z_2$ at $\Delta + k$ clockwise from $u$
   \item $z_1$ at $k$ counter-clockwise from $v$
  \end{enumerate}
  forces the survivor into a losing position: it is either immediately sandwiched on $C_{n+1}$,
  or falls into the trap described above on $C_{m+1}$.

  \begin{figure}
    \centering
    \includegraphics{q_m_n/guarding1}
    \includegraphics{q_m_n/guarding2}
    \caption{Preventing the zombies from turning in same direction on $C_{m+1}$\label{fig:guarding}}
  \end{figure}

  The survivor cannot start next to the zombies else it loses right away.
  So we choose $k$ such that, even if the survivor is as far away from one of the zombies as possible on $C_{n+1}$, the zombies still move in opposite directions. This is when the survivor is at distance two from one of the zombies (refer to Figure~\ref{fig:guarding}) and leads to the following inequalities:

  \begin{align*}
   n - \Delta - 2k - 2 &\leq \Delta + k +1 + k +2 && \text{and} \\
   \Delta + 2k -1 &\leq n - \Delta -2k +2 \mperiod
  \end{align*}
  Solving for $k$ gives
  \[ n - 2\Delta -5 \leq 4k \leq n-2\Delta +3 \mperiod \]

  A choice of $k$ which satisfies these constraints guarantees that the zombies move in opposite directions if the survivor starts on $C_n$.

\end{proofpart}

\section{Existence of $\Delta$ and $k$ for any $m,n$ \label{thm q_m_n 3}}

\begin{proofpart}
We wish to show that, for any $m, n$, there exist $\Delta$ and $k$
which guarantee the survivor is caught.
First, we show that $\Delta = \lfloor \frac{m}{2} \rfloor$ always
works for the Cornering Strategy.

Note that
\[
2\Delta = 2 \left\lfloor \frac{m}{2} \right\rfloor =
\begin{cases}
m & \text{if $m$ is even} \\
m -1 & \text{if $m$ is odd}
\end{cases} \mcomma
\]
and so $m -1 \leq 2 \lfloor \frac{m}{2} \rfloor \leq m$.

Suppose that we are in Case I.A. and $\Delta = \left\lfloor \frac{m}{2} \right\rfloor$.
Case I.A. is characterized by the following constraints:

\[ 4 \leq 2 \ell \leq m+1 \]
\begin{center}and\end{center}
\[ 2\Delta \leq n - m + 2\ell \qquad \text{or} \qquad 2\Delta \leq n-1 \mperiod \]
The zombies win if
\[ 2 \Delta \leq 2 m- 2 \ell + 2 \qquad \text{and} \qquad m - 2\ell  -1 \leq 2 \Delta \mperiod \]

In this case, the zombies win since
\[ 2 \Delta = 2 \lfloor \frac{m}{2} \rfloor \leq m < 2m - (m+1) + 2\leq 2 m- 2 \ell + 2 \]
\begin{center}and\end{center}
\[ m - 2\ell -1 \leq m - 5 < 2 \lfloor \frac{m}{2} \rfloor = 2 \Delta \]
shows that the zombie-win requirements of Case I.A. are met.

Suppose that we are not in Case I.A. Recall that in all cases we must have $2 \leq \ell \leq m-1$. Therefore, negating the constraints of Case I.A. gives

\[2\ell \leq 3 \qquad \text{or} \qquad m+2 \leq 2 \ell \]
\begin{center}or\end{center}
\[ 2\Delta \geq n - m + 2\ell +1 \qquad \text{and} \qquad 2 \Delta \geq n - 1 + 1 \]

But $2\ell \leq 3$ is only possible if $\ell = 1$, and this is not a valid value for $\ell$ (it puts the survivor too close to $z_1$). With the upper bound on $\ell$, the game is not in Case I.A. if
\[m+2 \leq 2 \ell \leq 2m-2\]
\begin{center}or\end{center}
\[ 2\Delta \geq n - m + 2\ell +1 \qquad \text{and} \qquad 2 \Delta \geq n - 1 + 1 = n \tag{*} \label{not i a}\].

Let us examine the consequences of assuming this second line \ref{not i a} to be true.

If we assume that $m$ is odd and $2 \Delta \geq n$ then we obtain a contradiction since
\[ 2 \Delta = 2 \lfloor \frac{m}{2} \rfloor = m -1 \geq n \]
and we have assumed that $m \leq n$.

If $m$ is even and $2\Delta \geq n$, then we must have $m = n$. If also $2\Delta \geq n - m + 2\ell +1$ then we must have

\begin{align*}
  2 \Delta & \geq n - m + 2\ell + 1 \\
  m & \geq m - m + 2 \ell + 1 \\
  m & \geq 2 \ell + 1 \mperiod
\end{align*}

Thus, if we set $\Delta = \lfloor \frac{m}{2} \rfloor$, we are in Case 1.A. unless

\begin{enumerate}
  \item $m+2 \leq 2\ell \leq 2m-2$, or
  \item $m=n$ are even and $m \geq 2\ell +1$.
\end{enumerate}

In the first case, with $m+2 \leq 2\ell \leq 2m-2$, the zombies can apply Case I.B. since it is characterized by the following constraints:

\[ m+1 \leq 2 \ell \leq 2m -2 \]
and
\[2 \Delta \leq n +1 \qquad \text{or} \qquad 2\Delta \leq n + m - 2 \ell \mperiod \]

Because $\Delta  = \left\lfloor \frac{m}{2} \right\rfloor$ and $m+2 \leq 2\ell \leq 2m-2$, satisfies these constraints, the zombies can enact the strategy of Case I.B. They will win since this choice of $\Delta$ also satisfies the win conditions of Case I.B.:

\[ 2\Delta \leq 2 \ell \qquad \text{and} \qquad 2\ell -m+1 \leq 2\Delta \mperiod \]

The first win condition is satisfied since $2\Delta \leq m < m+2 \leq 2\ell$, the second satisfied because $2\ell -m+1 \leq (2m-2)- m+1 = m-2 < m-1 \leq 2\Delta$.

In the second case, we have $m = n$ are even and $m \geq 2\ell+1$. In this case, the zombies can play as in Case II.A. since it is characterized by

\[ 4 \leq 2 \ell \leq m + 1\]
and
\[ n -m + 2\ell \leq 2 \Delta \qquad \text{and} \qquad n-1 \leq 2 \Delta \mperiod \]

Because $\Delta  = \left\lfloor \frac{m}{2} \right\rfloor$ and $2 \ell \leq m -1$ satisfies these constraints, the zombies can enact the strategy of Case II.A. They will win since this choice of $\Delta$ also satisfies the win conditions Case II.A.:

\[ 2 \Delta \geq 2n - 2m + 2\ell \qquad \text{and} \qquad 2 \Delta \leq 2n + 2\ell - m - 1 \mperiod \]

The first win condition is satisfied since $2\Delta = m \geq 2\ell +1 = 2n - 2m +2\ell +1 > 2n-2m+2\ell$, the second satisfied because $2 \Delta =m \leq m + 1 = 2n+2 -m -1 \leq 2n + 2\ell - m - 1$.

It remains to show there exists a suitable value for $k$. Since $k$ is constrained by the following inequalities
\[ n - 2\Delta -5 \leq 4k \leq n-2\Delta +3 \]
it suffices to show that the interval
\[ [ \frac{n-2\Delta -5}{4}, \frac{n-2\Delta+3}{4}] \]
contains an integer, which must be so since
\[ \left\lvert \frac{n-2\Delta+3}{4} - \frac{n-2\Delta -5}{4} \right\rvert = 2 \mperiod \]

To show that there exists $k \geq 0$, suppose we have
\[ n- 2\Delta +3 < 0 \mcomma \]
which means
\[ n < 2\Delta -3 \mperiod \]

With $\Delta = \left\lfloor \frac{m}{2} \right\rfloor$ we obtain a contradiction since we have presumed that $m \leq n$.

\end{proofpart}
\end{proof}

\section{Putting It All Together}\label{thm q_m_n 4}
From the proof we obtain the following strategy:
\begin{corollary}
  Two zombies win on $Q_{m,n}$ by placing $z_1$ at a counter-clockwise distance of
  \[ k = \left\lfloor \frac{n - 2 \left\lfloor\frac{m}{2}\right\rfloor +3}{4} \right\rfloor \]
  from $u$ and placing $z_2$ at a clockwise distance of
  \[ \left\lfloor \frac{m}{2} \right\rfloor + k \]
  from $v$.
\end{corollary}
\begin{proof}
  Place the zombies as described.
  If the survivor starts on $P_n$ between the two zombies, then it loses because the large cycle is guarded as per Section~\ref{thm q_m_n 2}.
  Otherwise, for first $k$ rounds the zombies move in opposite directions with $z_1$ reaching the chord after $k$ rounds.
  If, after $k$ rounds, the survivor is still on $P_n$, then it must be on a subpath of length at least 2 and at most $\Delta = \lfloor \frac{m}{2} \rfloor - 2$ (it cannot be adjacent to either $z_1$ or $z_2$). On the next turn, $z_1$ will take the chord, trapping the survivor on an induced $z_1z_2$-subpath smaller than half the diameter of $C_{n+1}$.
  The only remaining possibility is that the survivor is somewhere on $P_m$, with $z_1$ on $v$ and $z_2$ at a distance $\Delta$ from $u$, so the survivor loses by Section~\ref{thm q_m_n 1}.
\end{proof}
