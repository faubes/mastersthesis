zombie-win
A cop-win spanning tree is defined in \cite{fitzpatrick2016deterministic}
and \cite{clarke2002constrained}. We summarize below for convenience.

A (one-point) retract is an edge preserving function $f : G \mapsto H = G \setminus v$
(aka a homomorphism) such that $f(v) = x$ for some $x \neq v \in V(G)$ and $f$ restricted on $H$ is the identity.
Formally,

\[ f(v) = x \qquad f(u) = u \qquad \forall u \in V(G)\setminus \{ v \} \]

and
\[ xy \in E(G) \implies f(x)f(y) \in E(G \setminus \{ v \} \]

If $G$ is a reflexive graph, then a one-point retract can be seen as joining
two vertices. The edge between two adjacent vertices becomes another loop.
A retract maps a graph $G$ to graph $G'$ with one less vertex, but with connectivity
somehow preserved.

Recall that corners are vertices $v$ whose closed neighbourhoods
are a subset of a neighbours' closed neighbourhood, i.e.

\[u,v\in V(G) \qquad \text{and} \qquad N[v] \subseteq N[u] \]

You can define a retract on corner $v$: if $v$ is a corner, then it is
dominated by some $u \in V(G)$. So if $x \in V(G)$, $x \neq v$ and
$xv \in E(G)$ then $xu \in E(G)$ by definition of a corner. Therefore the map

\[ f(x) = \begin{cases}
u \text{if } x = v \\
x \text{otherwise}
\end{cases} \]

is edge preserving since $f(x)f(v) = xu$ and $xu \in E(G)$, so $xu \in E(H) = E(G - v)$.
For other vertices $x,y \not\in \{u,v\}$, $f(x)f(y) = xy \in E(G)$ so $f(x)f(y) \in E(G- v)$ also.
This shows that $f$ is a homomorphism as required and hence a retract.

This is a formal way of saying that a corner of a graph can be removed, somehow.
After all, the evading player does not want to find itself on such a vertex, lest
it be trapped by the pursuer. So ideally the robber never wants to go into that corner.
If a sequence of eliminating these corners leaves the robber with no options, well...

A dismantling is a sequence of retracts $f_1, f_2, \dots, f_{n-1}$ such that the
composition $F_{n-1} = f_{n-1} \circ f_{n-2} \circ \cdots f_2 \circ f_1$ gives a
function for which $F_{n-1} (G) = K_1$. That is, there is a sequence of retracts
which maps the graph to a single vertex.

Not all vertices of a graph need be corners in order for there
to exist a dismantling: it suffices to have an ordering where each $v_i$ is a corner in
$G[v_i, v_{i+1}, \dots, v_n]$.

Such a sequence of $f_j$'s defines a cop-win ordering
\[ \mathcal{O} = \{ v_1, v_2, \dots, v_n\} \]
 where $v_1$ is a corner in $G_1 = G$, $v_2$ is a corner in $G - v_1$, and so on.

A fundamental result in C \& R is that cop-win graphs -- graphs for which a single
cop is guaranteed to win --  are characterized by the existence of such dismantlings.
A graph is cop-win if and only if it is dismantlable.

A cop-win spanning tree $S$ is defined as a tree where $x,y\in V(G)$
$xy \in E(S)$ if there exists a retract $f_j$ in the dismantling
$F_n = f_{n} \circ f_{n-1} \circ \cdots \circ f_{2} \circ f_1$
such that $f_j (x) = y$ or $f_j (y) = x$.

Cop-win spanning trees give a strategy for the cops to follow: start at the root
(the last vertex in the ordering) and descend the tree in the branch containing the robber.
Lemmas 2.1.2 and 2.1.3 from \cite{clarke2002constrained} show that the cop
can always stay in the same branch (and above) the robber in the tree. So the
robber is eventually stuck in a leaf and caught.

Zombies cannot apply a cornering strategy, however. Fitzpatrick showed that a
graph is zombie-win if a specific spanning tree exists:

\begin{theorem}[6. Fitzpatrick]
If there exists a breadth-first search of a graph $G$ such that the associated spanning tree is also a cop-win spanning tree, then $G$ is zombie-win.
\end{theorem}

This characterizes zombie-win graphs as those for which a specific cop-win tree exists: one equivalent
to a breadth-first search of the graph from some vertex.

A few questions: are cop-win graphs necessarily zombie-win? (No. Smallest counter example?)
What is the dismantling of this cop-win but not-zombie-win graph. Since a dismantling exists,
a cop-win spanning tree exists. Why cannot it take occur as a BFS?

Here \ref{fig:copwin_tree1} is an example of a graph and two dismantlings, one of which results in a BFS tree,
and the other does not.

\begin{figure}
\centering
\includegraphics[scale=0.70]{copwin_tree/copwin_tree_example1}
\caption{A Cop-win Spanning Tree \label{fig:copwin_tree1}}
\end{figure}

Here are two dismantlings, their orderings, and the resulting cop-win spanning trees.
\begin{align*}
  f_1(b) = f \\
  f_2(c) = d \\
  f_3(f) = e \\
  f_4(a) = e \\
  f_5(e) = g \\
  f_6(d) = g
\end{align*}

Gives ordering $\mathcal{O}_1 = \{ b, c, f, a, e, d, g \}$. Whereas

\begin{align*}
  g_1(b) = f \\
  g_2(a) = e \\
  g_3(c) = d \\
  g_4(f) = d \\
  g_5(e) = d \\
  g_6(g) = d
\end{align*}

Also gives a dismantling with ordering $\mathcal{O}_2 = \{b, a, c, f, e, g, d \}$.
This is illustrated in \ref{fig:cop-win_bfs_trees}: only the second dismantling
produces a cop-win tree obtainable as a bread-first search.

\begin{figure}
\centering
\includegraphics[scale=0.70]{copwin_tree/copwin_tree_example1_trees}
\caption{Cop-win but Not Breadth-First Tree \label{fig:copwin_bfs_trees}}
\end{figure}

Moreover, it would seem that a zombie loses if it starts on $g$, but not on $d$.
