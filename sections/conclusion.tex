\chapter{Conclusion}

In Chapter~\ref{chapter planar zombies}, we showed the existence of a graph for which 3 zombies always lose,
thereby showing that the upper bound on the cop-number for planar graphs does not
apply to zombies. This is hardly surprising, since the 3 Cops must effect a sophisticated
strategy in order to capture the Robber, and the Zombies cannot coordinate in this way.

It remains to be shown if there is in fact an upper bound on the zombie-number for
planar graphs. The example obtained in this thesis was a sort of extrapolation from
the example given \cite{fitzpatrick2016deterministic}, which showed that
the cop-number need not always equal the zombie-number, specifically in the case of outerplanar graphs. Is it possible to construct increasingly elaborate graphs (while still being planar) which would always provide the survivor with a winning strategy?

We wanted to investigate a simpler class of graphs: outerplanar ones. In this case, as we have noted, it has been shown \cite{clarke2002constrained} that 2 cops suffice to guarantee a win.
It is also known that maximally-outerplanar graphs are zombie-win \cite{fitzpatrick2016deterministic}
and it is clear that 2 zombies suffice for a cycle, but what can be said about those
outerplanar graphs in between the two extremes?

It has been our experience that 2 zombies often suffice on outerplanar graphs; but
not always. The choice of zombie start is critical. This is the
motivation for our work on $Q_{m,n}$ -- the cycle with a single chord. Perhaps if we
could segment or decompose an outerplanar graph into simpler components, then we could at least give an upper bound: perhaps 1 or 2 zombies per block. It is not clear how we can
generalize our findings however. Adding a single extra chord changes the entire game.

We conclude with a few open questions which have yet to be addressed.

\section{Open Questions}

We have shown that zombies are not as effective as cops on planar graphs. What is the effectiveness of a zombie strategy on planar and outerplanar graphs. An upper bound for the zombie number of these classes of graphs has yet to be found.

In Chapter~\ref{chapter q_m_n} we found a sort of interval for calculating which vertices lead to a two-zombie win. It would be possible to count these intervals and study the game from the probabilistic point of view. That is to say, finding how many zombies you'd need to have even odds of winning on $Q_{m,n}$.

Graphs of the form $Q_{m,n}$ are basically two cycles which share an edge. Perhaps our findings could be generalized for any two overlapping cycles, though these graphs are no longer outerplanar. It is not clear if two zombies always win on these constructions.

Fitzpatrick showed that a graph is zombie-win if it admits a particular spanning tree (a zombie-win tree, see Subsection~\ref{subsection intro deterministic}). Is the existence of a zombie-win tree also a necessary condition?

Finally, a recent result about visibility-augmenting edges \cite{aichholzer2011convexifying} was used to conclude that visibility graphs of simple polygons are cop-win \cite{lubiw2017visibility}. Are visibility graphs also zombie-win? Zombies seem to win handily on these graphs, but it is not clear how they could be proven zombie-win.

We have implemented tools which allow us to search -- brute force -- for zombie-win trees. So far, every polygon tested produces a visibility graph which admits such a tree (see \ref{fig:polygon_with_bfs_dismantling} for an example).

\begin{figure}
  \centering
  \includegraphics[scale=0.75]{polygon/polygon_with_bfs_dismantling}
  \caption{A Polygon Inscribed with a BFS Cop-win Tree \label{fig:polygon_with_bfs_dismantling}}
\end{figure}
