
The Cylinder graph $C_{m,n}$ is a rectangular arrangement of $mn$ vertices in
$m$ rows and $n$ columns much like the Grid, except that vertices on one
boundary edge are joined to vertices on the opposite side.

\begin{figure}[h]
  \centering
  \includegraphics[width=0.35\textwidth]{cylinder/cylinder4by5}
  \caption{$C_{4,5}$ \label{fig:cylinder4by5}}
\end{figure}

Again, the Cylinder graph can be considered as the Cartesian product $C_{m,n} = P_m \square C_n$
of a cycle and a path. Note that this a planar graph for any $m$ and $n$.

We claim now that three zombies suffice to win on this family of graphs since they can execute
a guarding strategy similar to the one detailed in the previous section.

Place two zombies on a row such that $d(z_1, z_2) = \lfloor \frac{n}{2} \rfloor$.
Now observe that if the survivor finishes its turn on a different row, the zombies may move
to a vertex of the same column but closer row.
If the survivor finishes its turn on the same row, then the zombies have a single
zombie move on the same row but to a closer column.

Thus, after a finite number of rounds, the zombies can shadow the survivor's
horizontal shadow. Indeed, if we place these two zombies in the middle
of the cylinder, then the survivor's horizontal shadow is captured in at most
$\lceil \frac{m}{2} \rceil$ rounds.

The survivor is now trapped between these two zombies since they
can always move to recapture its horizontal shadow.
However, it could alternate between rows and thereby defeat the shadowing zombies.

We add another zombie to capture the survivor's vertical shadow.
Once the zombie's horizontal and vertical shadows are captured, the survivor
cannot remain on the same row and is unable to change row
indefinitely and thus will be cornered.
