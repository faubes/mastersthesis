
Vertex pursuit games are adversarial games played on a graphs.
A graph is a mathematical model used to describe structures such as networks and
maps. Players take turns moving tokens on a graph (the game board, if you like) with
the objective to capture (or evade) the other player.

Many variations of these graph pursuit games have been proposed. There are many rules and
parameters to tweak to produce different games:
\begin{enumerate}
\item How much information do the players have?
\item Do they know each others positions?From how far away?
\item Do the players know the playing field, i.e., the graph?
\item Are the players restricted to vertices or edges?
\item Are players obligated to move?
\item Does the graph change over time somehow?
\end{enumerate}

The Game of Cops and Robbers on Graphs \cite{bonato2011game} is perhaps the most
well-known vertex pursuit game. It is a perfect information game with Cops trying
to catch the Robber.

A variation called Zombies and Survivors (Z \& S or Zombie Game) was recently proposed and studied in \cite{fitzpatrick2016deterministic} and \cite{fitzpatrick2018game}.
Z \& S is the same as Cops and Robbers with the added twist that the zombies are required to move directly towards the survivor.

This thesis has been an attempt to better understand this variant and, in particular,
to see if the results obtained for Cops and Robbers still hold when the cops
are constrained in their strategy.

\subsection{How to Play}

To begin, the zombies choose starting vertices.
Then, the survivor choooses a start position.
Then on the next and each following round the zombies (must) move toward the survivor
and, if uncaught, the survivor (may) move. Here a move is an instantaneous jump
along an edge from one vertex to another.

The sophistication of the zombies' strategy gives them their name:
you can imagine the zombies -- arms outstretched -- ambling directly towards the survivor.
In this game, the players have complete information of the graph and the positions of the players. Indeed, the zombies need to know the position of the survivor to enact
their strategy.

The zombies move, the survivor responds and these two turns make one round.
It has been asked by new players if the order of play might be reversed but
then zombies always win.

The game concludes when either:
\begin{itemize}
\item A zombie catches the survivor. That is, a zombie wins by moving onto the vertex
occupied by the survivor.
\item It becomes clear that the survivor will never be caught.
In this case the survivor wins.
\end{itemize}

It is easy to determine that a zombie has won. It is perhaps not as obvious how
to determine the latter. We discuss winning conditions in greater detail in a later section.

The Z \& S games studied herein use a few zombies chasing a single survivor. The game
can be adapted to multiple surivors by making the zombies move toward
to the closest survivor but if a single survivor can't hope to escape then adding more ``survivors''
is a little gruesome.
