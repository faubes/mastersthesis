
This report presents and discusses the Game of Zombies and Survivors on Graphs (Z \& S or the Zombie Game), a variation
of the classic Game of Cops and Robbers recently studied in \cite{fitzpatrick2016deterministic}.
Z \& S is a vertex-pursuit game in which players take turns moving tokens on a graph with
the objective to capture (or evade) the other player.
A graph is a mathematical model used to describe networks, maps and -- appropriately enough -- board games.

\begin{figure}[h!]
\centering
\includegraphics[scale=0.20]{intro/cube.png}
\caption{The Cube Graph} \label{fig:Cube}
\end{figure}

Many games can be modelled using graphs. Monopoly, for example, can be seen as a large cycle in which every vertex is joined to the twelve (?) following vertices in clockwise order. Rolling the die determines which edge to follow.

A graph theoretic perspective can be applied to video games, which often involve pathfinding, mazes and chases.
In their introduction, the authors of \cite{bonato2011game} show how to convert the labyrinth in Pacman into a graph:
the vertices are the intersections of the maze and vertices are connected by an edge whenever a corridor links the intersections.

\begin{figure}[h!]
\centering
\includegraphics[scale=0.20]{intro/maze.png}
\caption{A Maze and its Corresponding Graph \cite{bonato2011game}[p.2]} \label{fig:BonatoMazeGraph}
\end{figure}

\subsection{How to Play}

The Game of Zombies and Survivors is a variant of Cops and Robbers,
a well-studied game. See \cite{bonato2011game} for a comprehensive study of Cops
and Robbers.

If familiar with the Game of Cops and Robbers, the reader may skip this section:
the only difference between the two games is that the zombies must move
along shortest paths towards the survivor. The cops are under no such restriction.

Zombies and Survivors is an adversarial game where one player controls the survivor(s),
the other the zombie(s).
Usually we keep it simple and have a few zombies chase a single survivor. After all,
if a signel survivor can't hope to escape, then adding more is simply gruesome.

Each round, the players take turns moving from vertex to vertex along the edges of a graph.
More precisely, the zombies all move, then the survivor moves.
The zombies seek to capture the survivor (move onto the same vertex)
while the survivor tries to escape indefinitely.

To begin, the zombies choose starting vertices to occupy.
Next, the survivor choooses a start position; ideally, one not too close to any of the zombies.
On the next and each following round the zombies move toward the survivor along a shortest path.

The sophistication of the zombie's strategy gives them their name.
You can almost imagine the zombies, arms outstretched, ambling directly towards the survivor.
Once all the zombies have moved, the survivor can move (flee) to a neighbouring vertex or pass.

The game concludes when either:
\begin{itemize}
\item A zombie catches the survivor. That is, a zombie moves onto the vertex
occupied by the survivor. This is a zombie win.
\item It becomes clear that the survivor will never be caught.
In this case we say it is survivor win.
\end{itemize}

We discuss winning conditions in greater detail in a later section.

The game can be uninteresting depending on the shape of the graph.
For example, if there are more zombies than vertices,
then the survivor cannot even start. If the board is a cycle, a path or a tree,
then the game is decided by the opening round.
The goal then becomes to create graphs to keep the game surprising. The outcome of a game in some way describes a feature of
a graph: its complexity, its connectivity, or perhaps, its ``survivability.'' Indeed, the number of zombies required to guarantee
that the survivor is caught can be seen as a graph parameter.

Before the game begins, the players agree to play on a
particular graph with a specific number of zombies. These could be considered
parameters of the game.

The graphs are assumed to be finite and connected: that is, there is a finite
number of vertices and there exists a path between every pair of vertices.
Playing on graphs with multiple connected components does not make much sense in the context of these games.
In our analysis, the players have complete information about the graph and the positions of the players.
