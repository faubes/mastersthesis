There has been a robbery downtown and the robbers are escaping by car. Officers
already on the streets are notified moments later. The robbers make a desperate dash for the highway but are spotted and soon tailed by police.

The robbers seem to be getting away -- putting some distance between themselves and the sirens.
Suddenly, the driver slams on the breaks. A squad car ahead has thrown out a strip of tire spikes! The left two tires are shredded, causing the driver to lose control. The vehicle veers off the road, flips upside down and eventually comes to a stop in the ditch. A media helicopter hovers overhead, capturing
a chaotic scene flooded by the flashing lights of emergency vehicles.

Was there ever any hope of escape? Perhaps the robbers took the wrong route.
They should have planned a vehicle swap. Or used a tunnel. Could it be that there were
so many police officers that all routes were covered? That capture was inevitable?
Perhaps the advantages of communication and central coordination allow the police to
  cut off likely escape routes, so that the probability of escape is low.

A (somewhat dispassionate) mind might watch these salacious stories on the news and wonder
 if you could apply math to these types of questions. To answer some of the above for sure.
Vertex pursuit games are adversarial games played on graphs which model this sort
of scenario.
By having players take turns moving tokens on a graph (the game board, if you like) with
the objective to capture (or evade) the other player, it is possible to simulate such chases.

Many variations of these graph pursuit games have been proposed \cite{bonato2017graph, bellman1967graphs}. There are many rules and
parameters to tweak to produce different games:

\begin{enumerate}
\item How much information do the players have?
\item Do they know each others positions? From how far away?
\item Do the players know the playing field, i.e., the graph?
\item Are the players restricted to vertices or edges?
\item Are players obligated to move?
\item Does the graph change over time?
\end{enumerate}

The Game of Cops and Robbers on Graphs \cite{bonato2011game} is perhaps the most
well-known vertex pursuit game. It is a perfect information game with Cops trying
to catch the Robber. In a perfect information game, all players know everything about the game.
In this context, the players know each other's positions (they see each other) and they
know the landscape (graph) around them \cite{schaefer1978complexity}.

A variation called Zombies and Survivors (Z \& S or Zombie Game) was recently proposed and studied  \cite{fitzpatrick2016deterministic, fitzpatrick2018game}.
Z \& S is the same as Cops and Robbers with the added twist that the zombies are required to move directly towards the survivor. More precisely, the zombies have to move along an edge
on a shortest path toward the survivor.

This thesis has been an attempt to better understand this variant and
to see if the results obtained for Cops and Robbers still hold when the cops
are constrained in their strategy. In general, we would like to know how different constraints imposed on the pursuers affects the number of pursuers required to win. We investigate ``the cost of being undead'', as Fitzpatrick \cite{fitzpatrick2016deterministic} would call it.
In particular, in Chapter~\ref{chapter planar zombies}
we give an example of a planar graph where 3 zombies always lose.
Then in Chapter~\ref{chapter q_m_n} we show how two zombies can win on a cycle with one chord.
