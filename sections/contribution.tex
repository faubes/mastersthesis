In Chapter 2, we show the existence of a graph for which 3 zombies always lose,
thereby showing that the upper bound on the cop-number for planar graphs does not
apply to zombies. This is hardly surprising, since the 3 Cops must effect a sophisticated
strategy in order to capture the Robber, and the Zombies cannot coordinate in this way.

In Chapter 3, we investigate a simpler class of graphs: a cycle with a single chord.
It has been our experience that 2 Zombies often suffice on outerplanar graphs. But
not always. The choice of zombie start is critical. This is the
motivation for our work on $Q_{m,n}$ -- the cycle with a single chord. Perhaps if we
could segment or decompose an outerplanar graph into simpler components, then we could at least give an upper bound: perhaps 1 or 2 Zombies per block. It is not clear how we can
generalize our findings however. Adding a single extra chord changes the entire game.
