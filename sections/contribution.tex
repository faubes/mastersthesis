In Chapter 2, we showed the existence of a graph for which 3 zombies always lose,
thereby showing that the upper bound on the cop-number for planar graphs does not
apply to zombies. This is hardly surprising, since the 3 Cops must effect a sophisticated
strategy in order to capture the Robber, and the Zombies cannot coordinate in this way.

It remains to be shown if there is in fact an upper bound on the zombie-number for
 planar graphs. The example obtained in this thesis was a sort of extrapolation from
 the example given \cite{fitzpatrick2016deterministic}, which showed that
 the cop-number need not always equal the zombie-number. Is it possible to construct
 increasingly elaborate graphs (while still being planar) which would always provide
 the survivor with a winning strategy?

 Having made no further progress in this direction, we decided to investigate a simpler
 class of graphs: outerplanar ones. In this case, as we have noted, it has been shown
 \cite{clarke2002constrained} that 2 Cops suffice to guarantee a win.

 It is also known that maximally-outerplanar graphs are zombie-win \cite{fitzpatrick2016deterministic}
 and it is clear that 2 Zombies suffice for a cycle, but what can be said about those
 outerplanar graphs in between the two extremes?

 It has been our experience that 2 Zombies often suffice on outerplanar graphs. But
 not always. The choice of zombie start is critical. This is the
 motivation for our work on $Q_{m,n}$ -- the cycle with a single chord. Perhaps if we
 could segment or decompose an outerplanar graph into simpler components, then we could at least give an upper bound: perhaps 1 or 2 Zombies per block. It is not clear how we can
 generalize our findings however. Adding a single extra chord changes the entire game.

 Finally, we spent some considerable time pondering games of Z \& S on visibility graphs.
 Recently, \cite{lubiw2017visibility} applied a result about visibility-augmenting edges from \cite{aichholzer2011convexifying} to conclude that visibility graphs of simple polygons
 are cop-win. A natural question then is to wonder if they are also zombie-win.

We have implemented tools which allow us to search, brute force, for Breadth-First Search
dismantling trees (i.e., zombie-win trees). So far, every polygon tested produces a visibility
graph which admits such a tree. See \ref{fig:polygon_with_bfs_dismantling} for an example.

\begin{figure}
  \centering
  \includegraphics[scale=0.75]{polygon/polygon_with_bfs_dismantling}
  \caption{A Polygon Inscribed with a BFS Cop-win Tree \label{fig:polygon_with_bfs_dismantling}}
\end{figure}
