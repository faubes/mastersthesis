Zombies and Survivors (or more specifically, ``deterministic zombies'') are an interesting variation proposed in \cite{fitzpatrick2016deterministic}. In these games, the
cops are replaced by zombies which must follow a geodesic to the survivor.

\subsection{How to Play Zombies and Survivor}

Zombies and Survivor is similar to C \& R except that zombies move
``directly'' toward the survivor. More precisely, on the zombies' turn
each zombie selects a shortest path toward the survivor (a \textit{geodesic})
and moves along the edge to the next vertex of the path.

The sophistication of the zombies' strategy gives them their name:
you can imagine the zombies -- arms outstretched -- ambling directly towards the survivor.
As in C \& R, the players have complete information of the graph and the positions of the players. Indeed, the zombies need to know the position of the survivor to enact
their strategy. If uncaught, the survivor may move to one of its neighbouring vertices or stay in place. Again, a move is an instantaneous jump along an edge from one vertex to another.

The game is decided if either:
\begin{itemize}
\item A zombie eats the survivor by moving to the survivor's vertex.
\item The survivor can evade the zombies indefinitely.
\end{itemize}

We call $s \in V(G)$ the survivor and $z_i \in V(G)$ are zombies with $i \in \{1, \dots, k\}$.
This notation represents both a player and its position in the graph.
In the games studied there is a single survivor and $k \geq 1$ of zombies.

As in C \& R, we divide the game into \textit{rounds} and \textit{turns}. A round consists of two turns: a zombie turn and a survivor turn.
It is convenient to define the zombie's turn on $t \equiv 0 (\mod{2})$ and the survivor's turn on $t \equiv 1 (\mod{2})$.
Round $r$ is given by $\lfloor \frac{t}{2} \rfloor$.

It is occasionally useful to identify the players' positions over time, in which
case let $z_r^i \in V(G)$ be zombie $i$ on round $r$. Similarly $s_r$ is the
survivor on round $r$. This burdensome notation will be omitted when possible.

We must be careful about the zombies' moves: since there can be multiple shortest paths linking a zombie $z_k$ to a survivor $s$, the zombie may have some limited agency on its turn. According to the rules of the game, on its turn the zombie
``must move on a shortest path'' towards the survivor. The possible \textit{zombie moves}
are those neighbours of $z_k$ which lie on a shortest path toward the survivor, which we denote

\[ Z[z_k, s] = \{ y \in N(x) \mid d(y, s) = d(z_k, s) - 1 \} \]

the zombies moves from $z_k$ given survivor is on $s$.

There is at least one such move since our graph is presumed connected,
so $Z[z_k,s] \neq \emptyset$. If there is only one path, then $z_k$'s has no choice but to move to the next vertex of that path. If all possible shortest paths move through the same next vertex, then again $z_k$ does not have any choice on its move. If, however, there are multiple shortest paths connecting the zombie tothe survivor with different first moves, then the zombie could make multiple moves. In our version of the game (in which we consider all possible outcomes), the zombies can coordinate before choosing their next move. In this way, the survivor will only win if it wins in every possible zombie-play.

\subsection{Deterministic Zombies}

The zombie number of a graph is defined analogously to the cop number: it is the number of Zombies
required to capture the Survivor. However, in Z \& S there are two additional considerations:
the zombie start and the zombie choices.
In Z \& S, the starting locations for the zombies is of utmost importance..
It is much easier to evade zombies which are clustered versus some
that are well-dispersed. So we say $z(G) = k$ if $k$ is the minimum number of
zombies required to guaranteed a win given an appropriate (or optimal) start.

Additionally, the rules of this game permit
some agency to the zombies: when confronted with multiple geodesics, they may have
a choice between neighbouring vertices. Our definition of the zombie number also presumes that
the zombies make the correct choices. So more precisely, the zombie number of a
graph is $k$ if $k$ zombies, suitably positioned, can play a game which guarantees the survivor
is caught.

Unlike cops, these zombies cannot apply a cornering strategy. Or any strategy.
As a consequence, we need at least as many zombies as you need cops.
This is one of the first observations
in \cite{fitzpatrick2016deterministic}: the cop-number $c(G)$ is a lower bound of the zombie-number $z(G)$. The zombies are weaker versions of cops, similar in a way to the ``fully active'' cops from \cite{gromovikov2018fully} wherein the cops are obligated to move on their turn. Both
active and ``lazy'' cops have more freedom of choice than the zombies, and thus
fewer are required to ensure victory.

Does there exist a characterization for zombie-win graphs -- a characterization for graph on which a single zombie can always win? One has yet to be described. Howeover, \cite{fitzpatrick2016deterministic} showed that a graph is zombie-win if a specific spanning tree exists:

\begin{theorem}[Fitzpatrick] If there exists a breadth-first search of a graph $G$ such that the associated spanning tree is also a cop-win spanning tree, then G is zombie-win.
\end{theorem}

Thus a sufficient condition for zombie-win graphs are those for which a specific cop-win tree exists: a cop-win tree obtainable as a breadth-first search of the graph from some root vertex. It is unknown if it is also a necessary condition.

A few questions: are cop-win graphs necessarily zombie-win? No. A counter example \cite{fitzpatrick2016deterministic} is reproduced below (refer to Figure~\ref{fig:copwin_but_notzombie_win}).

\begin{figure}
\centering
\includegraphics[scale=0.50]{copwin_tree/copwin_but_notzombie_win}
\caption{Cop-Win but not Zombie-Win \label{fig:copwin_but_notzombie_win}}
\end{figure}

Below is an example of a graph and two dismantlings, one of which results in a BFS tree, and the other does not (refer to Figure~\ref{fig:copwin_tree_example1}).

\begin{figure}
\centering
\includegraphics[scale=0.70]{copwin_tree/copwin_tree_example1}
\caption{A Cop-win tree \label{fig:copwin_tree_example1}}
\end{figure}

Here are two dismantlings, their orderings, and the resulting copwin spanning trees.
\begin{align*}
  f_1(b) = f \\
  f_2(c) = d \\
  f_3(f) = e \\
  f_4(a) = e \\
  f_5(e) = g \\
  f_6(d) = g
\end{align*}

Gives ordering $\mathcal{O}_1 = \{ b, c, f, a, e, d, g \}$. Whereas

\begin{align*}
  g_1(b) = f \\
  g_2(a) = e \\
  g_3(c) = d \\
  g_4(f) = d \\
  g_5(e) = d \\
  g_6(g) = d
\end{align*}

Also gives a dismantling with ordering $\mathcal{O}_2 = \{b, a, c, f, e, g, d \}$.
But only the second produces a copwin tree obtainable as a bread-first search.

\begin{figure}[h!]
\centering
\includegraphics[scale=0.70]{copwin_tree/copwin_tree_example1_trees}
\end{figure}

Moreover, it would seem that a zombie loses if it starts on $g$, but not on $d$.


\subsection{Probabilistic zombies}

Zombies are often depicted as mindless or aimless. It is a common trope that zombies
idle around, moving in random directions until they somehow (suddenly) distinguish
the uninfected. It is only at this point that the zombies will charge.

Such behavior likely inspired another type of pursuit game \cite{bonato2016probabilistic} in which the zombies start randomly on
the graph. Once the survivor chooses a start vertex, the zombies ``notice'' the survivor and start
moving directly towards it (again by following a shortest path).

Without knowing where the zombies start, however, it is impossible to know the outcome with certainty.
So study of these games becomes probabilistic; zombies win if they have at least a 50\% chance of winning.
The (probabilistic) zombie number of a graph is the minimum number of zombies required for a 50\% chance of winning and this zombie number is
obtained for several classes of graphs in \cite{bonato2016probabilistic} and for
toroidal grids in \cite{pralat2019many}.

The original paper on probabilistic zombies \cite{bonato2016probabilistic} also includes a lemma which is useful for our work
in Chapters~\ref{chapter planar zombies} and \ref{chapter q_m_n}:

\begin{lemma}[3.1, \cite{bonato2016probabilistic}] The survivor wins on $C_n$ against $k \geq 2$ zombies if and only if all zombies are initially located on an induced subpath
containing at most $\lceil\frac{n}{2}\rceil-2$ vertices.
\end{lemma}
