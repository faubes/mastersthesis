In \cite{aigner1984game}, Aigner and Fromme showed that the cop number for a planar graph is at most three.
A natural question, then, is to ask whether there exists an upper bound on the zombie number for planar graphs.
While we have not yet answered this question, we have found a planar graph for which the zombie number is greater than 3.

Such a graph $G$ consists of an interior 5-cycle with 5 outer paths connecting two adjacent vertices of the interior cycle.
There are also edges connecting the second and before-last outer paths to allow the survivor to escape in certain situations.
This graph, which is illustrated in Figure~\ref{fig:planarG2}, is an extension of the graph in \cite{fitzpatrick2016deterministic}[Fig. 2] which has
$z(G) = 3 > 2 = c(G)$.


\begin{figure}
\centering
\includegraphics[scale=0.5]{planar/Graph2.png}
\caption{A graph with $z(G) > 3$ \label{fig:planarG2}}
\end{figure}

We refer to the beginning of the game where the zombies and the survivor choose their starting positions as round zero or the starting round.
We assume that the zombies choose distinct starting vertices to maximize their chances of winning since the game is easily won by the survivor
if there are fewer than 3 zombies (for example, by adding another arbitrary zombie and following one of the strategies described below).

%We refer to the vertex of the $i$'th zombie, $1 \leq i \leq 3$  as $z_i^j$ with round zero being the
%starting or initial round where first the zombies choose a start vertex and then the survivor chooses %a start vertex.

We call vertices
\begin{align*}
C &= \{ 1, \dots, 5 \} & \text{the interior 5-cycle} \\
X &= V(G) \setminus C & \text{those vertices not on the interior 5-cycle} \\
Y &= \{7, 9, 12, 14, 17, 19, 22, 24, 27, 29\} & \text{the vertices of degree 3.} \\
S &= \{ 7,8,9,12,13,14,17,18,19, 22,23,24,27,28,29 \} & \text{the outermost 15-cycle}
%Z &= \{8, 13, 18, 23, 28 \} & \text{the set of peripheral vertices}
\end{align*}

With this notation, we describe how the survivor can escape 3 zombies by providing a strategy for the three possible zombie start configurations:

\begin{itemize}
\item $z_i \in C$ for $1 \leq i \leq 3$: all the zombies start on the interior 5-cycle.
\item $z_1, z_2 \in C$ and $z_3 \in X = V(G) \setminus C$: two of the zombies are on the interior 5-cycle but one is not.
\item $z_1, z_2 \in X$: at least two of the zombies are not on the interior 5-cycle.
\end{itemize}

Our proof relies on a special characteristic of this graph: if the survivor and the three zombies are all on $G[S]$, the outermost 15-cycle,  with
the zombies on the same side of the survivor and within a distance of 2, 3, 4 or 5, then the survivor can win by fleeing away from the zombies around the outermost 15-cycle.

To see this, let $E' = \{xy \in E(G) : x, y \in Y\}$ be the set of edges which connect an exterior 5-path to another and let $G' = G - E'$ be the subgraph without these edges.
These edges are highlighted in red in Figure~\ref{fig:planarG3}:

\begin{figure}
\centering
\includegraphics[scale=0.25]{planar/Graph3.png}
\caption{An escape strategy for the survivor\label{fig:planarG3}}
\end{figure}

If the survivor and zombie are both on the outermost cycle at distances 2 or 3 then
the fact that the zombies must stay in $S$ is obvious.
The following table shows that when the zombie and the survivor are both in $S$ and within a distance of 4 or 5, then the shortest path from the zombie to the survivor is contained entirely in $S$ and thus zombies never have the opportunity to leave the outermost 15-cycle.

\begin{tabular}{c | c | c | c | c | c}
z & s & shortest path in $G$ & $d_{G}(z,s)$ & shortest path in $G'$ & $d_{G'}(z,s)$  \\
\hline
7 & 14 & 7,8,9,12,13,14 & 5 & 7,6,1,2,3,15,14 & 6 \\
8 & 17 & 8,9,12,13,14,17 & 5 & 8,9,10,2,3,16,17 & 6 \\
9 & 18 & 9,12,13,14,17 & 5 & 9,10,2,3,16,17,18 & 6 \\
\hline
8 & 14 & 8,9,12,13,14 & 4 & 8,9,10,2,3,15,14 & 6 \\
9 & 17 & 9,12,13,14,17 & 4 & 9, 10, 2, 3, 16, 17 & 5 \\
12 & 18 & 12, 13, 14, 17, 18 & 4 & 12, 11, 2, 3, 16, 17, 18 & 6
\end{tabular}

We now give winning survivor strategies for each of the possible zombie-start configurations.

\begin{proof}
\textit{Case I}: The three zombies choose vertices on the interior 5-cycle.

Instead of showing that the strategy works for all possible start configurations of 3 zombies on the interior 5-cycle,
we show that the survivor can escape 5 zombies if they all start on the interior 5-cycle.
The zombies occupy the vertices 1--5 and the survivor chooses a vertex of degree 3.
Without loss of generality, say the survivor chooses 12.

If the survivor starts on $y_1 \in Y$ (one of the vertices of degree 3), and moves to $y_2\in Y$ using edge $y_1y_2$ and continues to flee in the same direction along the outermost 15-cycle, then the zombies will not be able to catch the survivor.

Notice that as soon as all three zombies are within a distance of 5 of the survivor on the outermost 15-path, then by the discussion above
 the game is won by the survivor.

\begin{figure}
\centering
\includegraphics[scale=0.25]{planar/Graph2Case1Round0.png}
\caption{Case I, Round 0\label{fig:planarG2C1R0}}
\end{figure}

On the first round, the zombies each have a single shortest path to the survivor on 12 and thus must move as follows:

\begin{itemize}
\item The zombie on 2 moves to 11.
\item The zombies on 1 and 3 collide on 2.
\item The zombies on 4 and 5 move to 3 and 1, respectively.
\end{itemize}

The survivor responds by moving to 9. These moves are illustrated in the following figures:
\begin{figure}
\centering
\includegraphics[scale=0.25]{planar/Graph2Case1Round1.png}
\caption{Case I, Round 1\label{fig:planarG2C1R1}}
\end{figure}

\newpage

Yet again the zombies have a single shortest path to the survivor on 9 and thus move as follows:
\begin{itemize}
\item The zombie on 11 moves to 12.
\item Zombies on 2 move to 10.
\item Zombies on 1 and 3 collide on 2.
\end{itemize}

The survivor responds by moving to 8. These moves are illustrated in the following figure:

\begin{figure}
\centering
\includegraphics[scale=0.25]{planar/Graph2Case1Round2.png}
\caption{Case I, Round 2\label{fig:planarG2C1R2}}
\end{figure}

\newpage

Finally, after round 3 all zombies are within a distance of 3 of the survivor on the outermost 15-cycle, and so the survivor wins by running anti-clockwise
on the cycle $G[S]$.

\begin{figure}
\centering
\includegraphics[scale=0.25]{planar/Graph2Case1Round3.png}
\caption{Case I, Round 3\label{fig:planarG2C1R3}}
\end{figure}

This shows that however the 3 zombies on the interior 5-cycle may be arranged in the initial round, they will not be able to corner the survivor following this
strategy.

\newpage

\textit{Case II}: Two zombies ($z_1$, $z_2$) choose vertices on the interior 5-cycle and one zombie ($z_3$) chooses a vertex in $X = \{ 6, \dots, 30 \}$,
an exterior vertex.

We use the same strategy as in Case I with an additional restriction on the survivor's start vertex.
The survivor starts on $s = y_1 \in Y$  (a vertex of degree 3) such that $3 \leq d_{G[X]}(s, z_3) \leq 4$ and so that the edge connecting $y_1$ to $y_2 \in Y$
is not on the shortest path between $s$ and $z_3$. That is to say, the survivor can flee from $z_3$ along an edge connecting two exterior 5-paths.

This choice of start vertex is always available to the survivor. Without loss of generality, assume that $z_3$ has chosen one of the vertices on the
exterior 5-path 6-10.

\begin{figure}
\centering
\includegraphics[scale=0.25]{planar/Graph2Case2Round0.png}
\caption{Case 2, Round 0\label{fig:planarG2C2R0}}
\end{figure}

\begin{itemize}
\item if $z_3$ chooses to start at 7 or 6, then the survivor chooses 27, which is at a distance of 3 or 4 respectively.
\item if $z_3$ chooses to start at 8, then the survivor can start at either 14 or 27, both of which are at a distance of 4.
\item if $z_3$ chooses to start at 9 or 10, then the survivor chooses 14, which is at a distance 3 or 4 respectively.
\end{itemize}

In round 1, if $z_3$ is not adjacent to the interior 5-cycle (either starting at 7, 8 or 9), then already the zombie has no choice but to pursue the survivor
on the outermost 15-cycle.

If $z_3$ is adjacent to the interior 5-cycle (either starting at 6 or 10), then $z_3$ may choose either to move onto the outermost 15-cycle or to cut through the interior 5-cycle since both are moves on a shortest $s,z_3$ paths.

However, as above, if $z_3$ chooses to move onto a vertex in $S$ and follow along the outermost 15-cycle, then the game is already won for the survivor since
$d(z_3, s) = 4$ and thus the third zombie can been forced to chase around the outermost 15-cycle forever.

If $z_3$ chooses to move to the interior cycle then all three zombies are on the interior 5-cycle and we have reached a situation just as in Case I, Round 1.

This shows that the survivor will always escape the third zombie following this strategy. Now because this strategy is a restricted version of the strategy from
Case 1, we know that the zombies that start on the interior 5-cycle will not be able to corner the survivor. Therefore, this strategy defeats all possible
start configurations where two zombies start on the interior 5-cycle and the third starts on the exterior.

\textit{Case III}: One zombie chooses a vertex on the interior 5-cycle, the two others choose vertices on the exterior.

We were unable to develop an argument to concisely show why the survivor wins in this case.
Instead, in Appendix \ref{appendix:planarZombies} include tables showing the first few moves of a winning survivor strategy
for every possible zombie start (without loss of generality).

\textit{Case IV}: All three zombies choose exterior vertices in $X$.

We separate this case again into sub-cases based on the number of moves required
by the zombies to reach the interior cycle.

\textit{Case IV(a)}: All three zombies require the same number of rounds to reach the interior 5-cycle.

Suppose all the zombies have chosen vertices in $X$ which are adjacent to vertices in $C$. These are vertices $Q = \{6, 10, 11, 15, 16, 20, 21, 25, 26, 30 \}$.
Because there are 3 zombies and 5 interior vertices, there will always be at least two vertices in the interior cycle that are not threatened in
round 0. The survivor starts on one of these safe vertices.

In round 1, the zombies have no choice but to enter the interior 5-cycle since the shortest path from a vertex $q \in Q$ to
$s \in C$ necessarily includes the edge $qc$ for some $c \in C$. Thus, after their first turn, the zombies all occupy vertices in the interior 5-cycle.
The survivor responds by exiting the interior 5-cycle to $s' \in Q$.

In round 2, the zombies again have no choice but to approach the survivor using vertices on the interior 5-cycle. The survivor responds by moving to
some $s'' \in Y$ and we have reached a scenario just like in Case I and so the survivor has a winning strategy.

If all the zombies are at a distance of 2 from the interior 5-cycle (those vertices in $Y$) then the survivor can start on any vertex $s \in C$.

In round 1, the zombies approach the survivor by moving to vertices in $Q$. Let $q_0, q_1 \in Q \cap N(s)$ be the neighbours of the survivor
which are not on the interior 5-cycle. Now, either:

\begin{enumerate}
\item $q_0$ and $q_1$ are occupied by zombies. In this case, there is some $c \in N(s^0) \cap C$ which is not threatened by a zombie (since two of them are adjacent
to $s$). Therefore the survivor can safely move onto another vertex on the interior 5-cycle and, on the following round, move to an occupied vertex in $Q$. After
another round the survivor moves to a vertex in $Y$ and we again have reached a situation as in Case I.

\item $q_0$ and $q_1$ are not both occupied by zombies. In this case, the survivor can exit the interior 5-cycle immediately by moving to a vertex in $Q$. After
 the next round, all three zombies are on the interior 5-cycle and the survivor moves to a vertex in $Y$ and again we are in a situation like Case I.
\end{enumerate}

If all the zombies are at a distance of 3 from the interior 5-cycle, then the survivor may start on any vertex of $C$ and simply pass on the first round. The zombies,
have no choice but to move to vertices in $Y$ and so we find ourselves in the case described before.

Now we must deal with the cases where the zombies are at different distances from the center cycle.

\textit{Case IV(b)}: Two zombies start adjacent to the interior 5-cycle, and the third is at distance 2 from the interior 5-cycle.

Suppose that two of the zombies have chosen vertices in $Q$ and the other has chosen a vertex in $Y$. That is, two zombies are adjacent to the interior 5-cycle
while the third requires two rounds to reach the interior 5-cycle.

There are now at least three unthreatened vertices on the interior 5-cycle for the survivor to choose. The survivor can choose any unthreatened vertex on
the interior 5-cycle.

In round 1, two zombies enter the interior 5-cycle and the third moves to a vertex $q \in Q$ adjacent to the interior 5-cycle.
The survivor exits the interior 5-cycle to another vertex $q_0 \in Q$. This move is always available to the survivor since only one vertex in $Q$ is occupied by
a zombie and every vertex in $C$ is adjacent to two verties in $Q$.

After the next turn, all three zombies are on the interior 5-cycle and the survivor is on a vertex $s^2 \in Y$ and so the survivor has a winning strategy.

\textit{Case IV(c)}: Two zombies start at a distance of 2 from the interior 5-cycle and the third is at a distance of 3.

The survivor may start on any of the vertices on the interior 5-cycle since none are threatened by a zombie.

In round 1, two zombies move to vertices in $Q$ and the third moves to a vertex in $Y$. If the survivor is unthreatened after the first round, she may simply pass.
If the survivor is threatened by one of the zombies adjacent to the interior 5-cycle, then at least one of her neighbours on the interior 5-cycle is unthreatened.

In either case, after round 1 we find ourselves in the situation described in Case IV(b).

\textit{Case IV(d)}: Two zombies start adjacent to the interior 5-cycle, and the third is at distance 3 from the interior 5-cycle.

This scenario is slightly more complicated as the survivor must avoid being trapped by the third zombie. Consider, for example, the start configuration
$\bar{z} = (10, 26, 18)$. If the survivor chooses to start at 4, then the game plays out as follows:

\begin{tabular}{c | c | c | c | c }
Round & $z_1$ & $z_2$ & $z_3$ & $s$ \\
\hline
0 & 10 & 26 & 18 & 4 \\
1 & 2 & 5 & 19 & 21 \\
2 & 3 & 4 & 22 & 21
\end{tabular}

The survivor is cornered by the zombies approaching from the interior 5-cycle and by the third zombie which uses the edge 19-22.
However, the survivor could have started at 1, in which case the game is won by the survivor as follows:

\begin{tabular}{c | c | c | c | c }
Round & $z_1$ & $z_2$ & $z_3$ & $s$ \\
\hline
0 & 10 & 26 & 18 & 1 \\
1 & 2 & 5 & 17 or 19 & 6 \\
2 & 1 & 1 & 16 or 20 & 7 \\
3 & 6 & 6 & 3 or 4 & 29
\end{tabular}

And we see that the survivor has a winning strategy for this start configuration.

Suppose without loss of generality that the zombie at distance 3 from the interior 5-cycle has chosen vertex 18.
Since there are two zombies adjacent to the interior 5-cycle, at least one of the vertices $\{1, 2, 5\}$ must be a safe start for the survivor.

We may disregard the zombies that started at a distance of 1 from the interior 5-cycle in this next analysis since the survivor's strategy will be the
same as in Case IV(a) and so these zombies will not be able to capture the survivor. Having shown above that if 1 is a safe start for the survivor, it
remains to show that the strategy works if only 2 or 5 are safe starts. Since they are symmetric, we show that the strategy works if 2 is a safe start for the
survivor.

\begin{tabular}{c | c | c | c | c }
Round & $z$ & $s$ \\
\hline
0 & 18 & 2 \\
1 & 17 & 10 \\
2 & 16 & 9 \\
3 & 3 & 8 \\
4 & 2 & 7 \\
5 & 1 & 29 \\
6 & 30 & 28
\end{tabular}

Thus after 7 rounds, the survivor has succesfully baited all three zombies onto an exterior 5-path and so the game is won.

\textit{Case IV(e)}: One zombie starts adjacent to the interior 5-cycle, and the other two are at a distance of 2 from the interior 5-cycle.

Again, the survivor's strategy in this case is to waste time on the interior 5-cycle in order to allow all the zombies to approach. Since only one of the
zombies is adjacent to the interior 5-cycle, there are four potential start vertices for the survivor. Any of these will work.

In round 1, the zombie at distance 1 from the interior 5-cycle moves onto the interior 5-cycle and the other two move to vertices $q_0, q_1 \in Q$,
which are adjacent to the interior 5-cycle.

Now, either:

\begin{enumerate}
\item $q_0$ and $q_1$ are adjacent to $s^0$. In this case, the survivor moves to $s^1 \in N(s^0) \cap C$, the neighbour on the interior 5-cycle that is not occupied
by the zombie that has already reached the interior 5-cycle. After the next turn, all three zombies have reached the interior 5-cycle and so the survivor can
exit to some $s^2 \in Q$. Again, after another round we have returned to Case I.

\item $q_0$ and $q_1$ are not both adjacent to $s^0$. In this case, the survivor can exit the interior 5-cycle by moving to a vertex $s^1 \in Q$. After the next
round, all three zombies are on the interior 5-cycle and we are in a situation like Case I.
\end{enumerate}

In either case, the survivor has a simple winning strategy.


\textit{Case IV(f)}: One zombie starts at a distance of 2 from the interior 5-cycle, and the other two are at a distance of 3.

The survivor starts in the interior 5-cycle. None of the vertices on the interior 5-cycle are threatened by the zombies, since they are at a distance at least 2.

In round 1, the zombies approach the interior 5-cycle. The zombie that started at distance 2 from the interior 5-cycle is now on a vertex in $Q$ and the other two
zombies are on vertices in $Y$. If unthreatened, the survivor simply passes.
If the survivor is threatened by the zombie that is adjacent to the interior 5-cycle, then she moves to another vertex on the interior 5-cycle. The other two zombies
pose no threat in this round.

There is now one zombie at distance of 1 from the interior 5-cycle and two zombies at a distance of 2, and so we have returned to the situation describe in Case IV(e).

\textit{Case IV(g)}: One zombie starts at a distance of 1 from the interior 5-cycle, and the other two are at a distance of 3.

The survivor starts on one of the four safe vertices on the interior 5-cycle.

In round 1, one zombie steps onto the interior 5-cycle while the other two zombies move to vertices at distance 2 from the interior 5-cycle.
Only the zombie on the interior 5-cycle can threaten the survivor at this point. If the survivor is safe, then she may pass. Otherwise, since there is
only a single zombie on the interior 5-cycle, at most one of the survivor's neighbours on the interior 5-cycle is threatened.
So the survivor has a safe move to a vertex on interior 5-cycle.

In round 2, the zombie on the interior 5-cycle pursues the survivor ineffectually while the other two zombies move to vertices $q_0, q_1 \in Q$ which are
adjacent to the interior 5-cycle. Now, as in Case IV(e), either

\begin{enumerate}
\item $q_0$ and $q_1$ are adjacent to $s^0$. In this case, the survivor moves to $s^1 \in N(s^0) \cap C$, the neighbour on the interior 5-cycle that is not occupied
by the zombie that has already reached the interior 5-cycle. After the next turn, all three zombies have reached the interior 5-cycle and so the survivor can
exit to some $s^2 \in Q$. Again, after another round we have returned to Case I.

\item $q_0$ and $q_1$ are not both adjacent to $s^0$. In this case, the survivor can exit the interior 5-cycle by moving to a vertex $s^1 \in Q$. After the next
round, all three zombies are on the interior 5-cycle and we are in a situation like Case I.
\end{enumerate}


\textit{Case IV(h)}: The three zombies are at different distances from the interior 5-cycle.

In particular, this means that the zombies are at distances 1, 2 and 3 from the interior 5-cycle.

Observe that there is always a vertex in the interior 5-cycle that is at distance at least 3 from all zombies. This is a start position for the survivor which
will allow her to survive unthreatened for at least two rounds.

In round 1, the closest zombie (more precision here - give label) moves onto the interior 5-cycle, the second closest moves to a vertex adjacent to the interior 5-cycle and the third
moves to a vertex at a distance of 2 from the interior 5-cycle. The survivor remains in place.

In round 2, the closest zombie threatens the survivor, the second closest zombie moves onto the interior 5-cycle, and the last one moves onto a vertex adjacent to the
interior 5-cycle. Now, at least one of the survivor's neighbours is an unoccupied vertex in $Q$, which she can take to escape the interior 5-cycle.

After the next round, all three zombies are on the interior 5-cycle or one step behind the survivor and the survivor has won the game by moving to a vertex in $Y$
as in Case I.

\end{proof}
