
The following sections will use a few basic ideas and definitions from graph theory.
Formally, a graph $G = (V, E)$ is composed of:

\begin{itemize}
\item A set $V$ of vertices.
\item A set $E \subset V \times V$ of edges described by a pair of endpoints.
\end{itemize}

If $G = (V,E)$ is a graph and $x,y \in V$ are vertices, we say that vertices $x$ and $y$ are neighbours if $(x,y) \in E$.
That is, if there is an edge joining $x$ to $y$.
In these games, the edges are assumed to be undirected and so we may write $xy = yx$
and consider the two directions as a single edge.
We call the set of all neighbours of $x$ the neighbourhood of $x$ which we denote $N(x) \subset V$.

[Need to add degree, subgraph, paths,  distance, diameter]

For example, in Figure 1 we have vertices $V = \{ 000, 001, 010, 011, 100, 101, 110, 111 \}$.
Since $000$ and $001$ are connected, $(000, 001) \in E$.
The neighbourhood of $000$ is $N(000) = \{ 001, 010, 100 \}$.

\subsection{Modeling the Game}

We need a way to identify the players' positions over time so let
$z_t^i \in V(G)$ be zombie $i$ on round $t$. Similarly $s_t$ is the
survivor on round $t$. It might be tempting to group the zombies together
into some tuple of the vertex set, but each zombie acts independently of
the others and so this may not always be practical.

We typically use only one survivor, so we normally only use a few zombies, and
say $i \in \{1, \dots, k\}$ for ``small'' values of $k$.

Indeed, in some cases a single zombie may suffice
to capture the survivor. (We examine this scenario in the next section.)

The zombies play, then the survivor plays and these two turns make one round.
In particular, when we start the game we allow $z_0 \in V(G)$ to be
 any vertex of the graph.

\subsection{Paths and Moves}
The zombie strategy requires that we consider all shortests paths connecting
the zombie to the survivor.

For zombie $k$, write
$Z_k = \{  \exists \ell : z_k = u_{i,0}, u_{i, 1}, u_{i, 2}, \dots, u_{i, \ell-1}, s= u_{i, \ell}\}$ be the set of $i$ different $z_ks$-paths of length $\ell$.

There is at least one such path since our graph is presumed connected,
so $i > 0$ and $Z_k \neq \emptyset$.

If there is only one path, then $z_k$'s next move is $u_{i, 1}$. If all $zs$-paths
include $u_{i,1}$, then again $z_k$'s next move must be to that vertex.

If, however, there are multiple $zs$-paths which have different first moves,
then the zombie could make multiple moves.

We call all the set of all neighbours on
a shortest path to the survivor the \textit{zombie moves}, which formally are

\[ Z[x] = \{ y \in N(x) \mid d(y, s) = d(x, s) - 1 \]

In these games we use graph distance (or, equivalently, assume the edges have unit cost).
In other words, the distance between two vertices is the hop length of a (shortest possible) path
connecting them.
(EDIT: The existence and nature of shortest path may warrant further discussion, in lemmas \& observations below.)


\subsection{Rounds and Turns}
We divide the game into rounds and turns. A round consists of two turns:
a zombie turn and a survivor turn. This is necessary because we must
consider distances between the players at different points of the game:
at the beginning and middle of each round.

We track this by counting the turns. It is the zombie's turn
on $t \equiv 0 \mod{2}$ and the survivor's turn on $t \equiv 1 \mod{2}$.
Round $r$ is given by $\lfloor \frac{t}{2} \rfloor$.

The game starts on round 0 with the zombies choosing
initial vertices. The survivor follows. In a sense the game really begins in
round 1 with the zombies finding, selecting and moving along a shortest path.
The survivor responds. The game repeats in this way until the survivor is caught,
or both players agree that the survivor will always escape.

\subsection{Zombie Number}

EDIT: Not quite. Redo

The minimum number zombies guaranteed to win.

If a single zombie is guaranteed to win on a graph, we say that it is zombie-win
and that it has zombie number $z(G) = 1$. If it can be shown that $k$ zombies win,
then $z(G) = k$
