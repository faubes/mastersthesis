
In this thesis, all graphs are assumed to be finite and connected. That is, there is a finite
number of vertices and there exists a path between every pair of vertices.
Playing on graphs with multiple connected components can be reduced to playing
multiple games in parallel.

The following sections will use a few definitions from graph theory.
Formally, a graph $G = (V, E)$ is composed of:

\begin{itemize}
\item A set $V$ of vertices.
\item A set $E$ of edges ${u,v}$ where $u, v \in V$.
\end{itemize}

If $G = (V,E)$ is a graph and $x,y \in V$ are vertices, we say that vertices $x$ and $y$ are neighbours if $xy \in E$.
That is, if there is an edge joining $x$ to $y$.
In these games, the edges are assumed to be undirected and so we may write $xy = yx$
and consider the two directions as a single edge.
We call the set of all neighbours of $x$ the neighbourhood of $x$ which we denote $N(x) \subseteq V$.

For example, in Figure~\ref{fig:hyper-cube} we have vertices $V = \{ 000, 001, 010, 011, 100, 101, 110, 111 \}$.
Since $000$ and $001$ are connected, $(000, 001) \in E$.
The neighbourhood of $000$ is $N(000) = \{ 001, 010, 100 \}$.
[Add definition of closed neighborhood]

\begin{figure}
\centering
\includegraphics[scale = 0.25]{intro/cube.png}
\caption{The Hypercube of Dimension 3 \label{fig:hyper-cube}}
\end{figure}

The degree of a vertex is the number of edges incident to that vertex. The minimum and maximum
degrees of a graph are sometimes denoted as $\delta(G)$ and $\Delta(G)$, respectively.

\subsection{Playing and Modeling the Game}

The survivor is $s \in V(G)$ and $z_i \in V(G)$ are zombies with $i \in \{1, \dots, k\}$.
Notice that the notation represents both a player and its position in the graph.
In the games studied there is a single survivor and $k \geq 1$ of zombies.

We divide the game into rounds and turns. A round consists of two turns:
a zombie turn and a survivor turn.

The game starts on round 0 with the zombies choosing
initial vertices. The survivor follows. In a sense the game really begins in
round 1 with each zombie independently selecting a shortest path
toward the survivor and moving along the next edge.
The survivor responds. The game repeats in this way until the survivor is caught,
or both players agree that the survivor will always escape.

It is convenient to define the zombie's turn on $t \equiv 0 \mod{2}$ and the survivor's turn on $t \equiv 1 \mod{2}$.
Round $r$ is given by $\lfloor \frac{t}{2} \rfloor$.

It is occasionally useful to identify the players' positions over time, in which
case let $z_r^i \in V(G)$ be zombie $i$ on round $r$. Similarly $s_r$ is the
survivor on round $r$. This burdensome notation will be omitted when possible.

It might be tempting to group the zombies together
into some tuple of the vertex set, but each zombie acts independently of
the others and so this may not always be practical.

\subsection{Paths and Moves}
The zombie strategy requires that we consider all shortests paths connecting
each zombie to the survivor. Let us be precisely define these terms.
%In the studied discrete graphs below, the distance between two vertices is the hop length of a %(shortest possible) path connecting them.

A path $P = v_0, v_1, v_2, \dots , v_n$ is a ``strict'' walk: a non-repeating sequence of
adjacent vertices in a graph. A path is an example of a subgraph $G' = (V(P), E(P))$ since $V(P) \subseteq V$ and $E(P) \subseteq E$. A cycle is a path of at least 3 vertices where the
start and endpoints are the same vertex.

Paths allow us to define a distance $d(x,y)$ between vertices as the length of the shortest path connecting them (or infinity if such does not exist) and computing such paths is a classic
problem in computer science.

The diameter and girth of a graph are useful properties which appear in some
of the theorems herein.
The diameter $\diam(G)$ is the length of a longest possible shortest path in $G$ and the girth of a graph is the length of the minimum order subcycle.

Consider zombie $k$. According to the rules of the game, on its turn the zombie
``must move on a shortest path'' towards the survivor. More precisely, this requires
considering $Z_k = \{  \exists \ell : z_k = u_{i,0}, u_{i, 1}, u_{i, 2}, \dots, u_{i, \ell-1}, s= u_{i, \ell}\}$ the set of $i$ different $z_ks$-paths of length $\ell$.

There is at least one such path since our graph is presumed connected,
so $i > 0$ and $Z_k \neq \emptyset$.

If there is only one path, then $z_k$'s next move is $u_{i, 1}$. If all $zs$-paths
include $u_{i,1}$, then again $z_k$'s next move must be to that vertex.

If, however, there are multiple $zs$-paths which have different first moves,
then the zombie could make multiple moves.

We call all the set of all neighbours on
a shortest path to the survivor the \textit{zombie moves}, which could be denoted

\[ Z[x; s] = \{ y \in N(x) \mid d(y, s) = d(x, s) - 1 \]

the zombies moves from $x$ given survivor is on $s$.
