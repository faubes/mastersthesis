\section{Notation}

The following sections will use a few standard definitions from graph theory (and vertex-pursuit theory) which we include here for reference.
Formally, a graph $G = (V, E)$ is composed of:

\begin{itemize}
\item A set $V$ of vertices.
\item A set $E$ of edges $\{u,v\}$ where $u, v \in V$.
\end{itemize}

We also write $V(G)$ for the set of vertices of $G$ and $E(G)$ for the set of edges of $G$.
Let $G = (V,E)$ be a graph with vertices $x,y \in V$.
The graphs studied herein are finite, connected, undirected and reflexive.
There is a \textit{finite} number of vertices and \textit{connected} means there exists a path connecting every pair of vertices.
We limit ourselves to connected graphs because playing on graphs with multiple connected components can be reduced to playing multiple games in parallel: the players are restricted to their starting connected component. By undirected, we mean that an edge from $x$ to $y$ implies an edge from $y$ to $x$ so we treat the two directions as a single edge and write $\{x,y\}$ or simply $xy = yx \in E$.
Lastly, in order to model a player's choice to pass on a turn, we suppose each vertex also has a loop (an edge to itself), making the graph reflexive. This way, players still choose an edge even though they do not move to a different vertex.

We will have occasion to use a few more concepts of graph theory. We say that vertices $x$ and $y$ are \textit{neighbours} if $xy \in E$; that is, if there is an edge joining $x$ to $y$.
The set $N(x) = \{ y \in V | xy \in E \} \subseteq V$ is the \textit{neighbourhood} of $x$.
The \textit{closed neighborhood} of vertex $x$ is the the neighborhood of $x$ along with $x$ itself and is denoted $N[x] = N(x) \cup \{x\} \subseteq V$. A set $S \subseteq V(G)$ is said to be \textit{dominating} if $\cup_x N(x) \supseteq V(G)$. In the context of these games, a dominating set guarantees a win for the pursuers, since all possible start vertices are covered (and the evader loses in round 1).

The \textit{degree} of a vertex is the number of edges incident to that vertex (or equivalently the
cardinality of its neighbourhood $\lvert N(x) \rvert$). The \textit{minimum} and \textit{maximum
degrees} of a graph are $\min \text{ or } \max \{ \lvert N(x)\rvert : x \in V\}$ and are denoted as $\delta(G)$ and $\Delta(G)$, respectively.

For example, in Figure~\ref{fig:hyper-cube} we have vertices $V = \{ a, b, c, d, e, f, g, h \}$.
Since $a$ and $b$ are connected by an edge, we have $ab \in E$.
The neighbourhood of $a$ is $N(a) = \{ b,d,f \}$ and the closed neighbourhood of $a$ is
$N[a] = \{ a, b, d, f \}$. In this example, we also have that $\delta(G) = \Delta(G) = 3$ since all vertices have degree 3.

\begin{figure}
\centering
\includegraphics[scale = 0.25]{intro/cube.png}
\caption{The Hypercube of Dimension 3 \label{fig:hyper-cube}}
\end{figure}

Two basic classes of graphs are important in the study of these games: paths and cycles.
A path $P = v_0, v_1, v_2, \dots , v_n$ is a ``strict'' walk: a non-repeating sequence of
adjacent vertices in a graph. A cycle $C_n$ is a path of length $n \geq 3$ with an additional edge joining the last vertex back to the first (a so-called \textit{closed} path).
We say that a graph contains a path $P$ if $P$ is a \textit{subgraph} of $G$, so $V(P) \subseteq V(G)$ and $E(P) \subseteq E(G)$. More generally, with $S \subseteq V(G)$ a set of vertices, write $G[S]$ for the \textit{induced subgraph}: the graph which contains $S$ and the edges of $G$ which join vertices of $S$.

Paths allow us to define a distance between vertices $d_G(x,y)$ as the length of the shortest path connecting $x$ to $y$ (or infinity if such a path does not exist; never the case in our games). Computing such paths, also known as \textit{geodesics}, is a classic problem in computer science.
A geodesic has the additional property of being \textit{isometric} \cite{pan2006isometric}, meaning that the distance between vertices of an isometric path is preserved in the subgraph induced by the path. This property allows players to guard or patrol isometric paths \cite{aigner1984game}, preventing
the robber from entering (and thus crossing) the path without capture.

Finally, the \textit{diameter} and \textit{girth} of a graph are two useful graph properties which appear in some of the theorems herein. The diameter $\diam(G)$ is the length of a longest possible shortest path in $G$. The girth of a graph is the length of the minimum order subcycle.
