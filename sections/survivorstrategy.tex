
Suppose we could agree on some algorithm to fully determine the zombies' behaviour.
Or, perhaps, assume that all possible games will exhaustively be played by the computer.
How then, should we program the survivor to maximize its chances of survival?
On every round, the survivor may stay in place or move to one of its neighbours.
However, if ever the survivor moves to a vertex adjacent to a zombie, then it loses immediately on the next round.
So the \textit{valid survivor moves} are the neighbours of the survivor (or its current position), minus those adjacent to one of the $k$ zombies.

If the survivor is $s$ and $Z = \{z_1, z_2, \dots, z_k\}$ is the set of zombie positions, then

\[N[s]\setminus N[Z]\]

Where the neighbourhood of the set $Z$ is the union of all of the zombies' neighbourhoods.
These survivor moves can be computed by iterating through the neighbours of $s$ and removing those that are neighbours of a zombie.
Another approach would be to use the results of Floyd-Warshall, as with the zombies:

\begin{enumerate}
\item Scan row $s$ of $A$ for indexes $x$ where $a_{s,x} = 1$. These are the neighbours of $s$. Add each neighbour a set $S$.
\item For each neighbour $x$ and for each zombie $z$, $1 \leq z \leq k$, probe $a_{z,x}$. This is the distance from the neighbour to the zombie.
\item If $a_{z,x} = 1$, then $x$ is adjacent to a zombie and so $S = S \setminus \{x\}$.
\item Return $S$
\end{enumerate}

If the set of valid survivor moves is empty, then the survivor is cornered. The only remaining move is to pass, and be caught after another round.
If the set returned is a singleton, then circumstances have forced the survivor's hand.
If, however, there many possible moves, then how best do we choose among them?

Perhaps the simplest strategy is to invert the strategy used by the zombies: the survivor makes the move that maximizes its distance from all of the zombies.
While running the algorithm described above, we could simultaneously compute $\sum_{i=1}^{k} d(x, z_i)$, the sum of all the distances from the neighbour to the
zombies, and choose one the moves that maximizes this value.

This cowardly strategy is amusingly similar to that of the zombies. It is also a poor strategy. The only way to escape the zombies is to lead them into some sort of cycle,
as we discuss next. So the survivor needs to act with more sophistication than just fleeing in the opposite direction. The game depicted below is an example
where the survivor has an easy win, but the strategy above fails.

\begin{figure}
\centering
\includegraphics[scale=0.20]{survivorstrategy/BadSurvivorStrat.png}
\caption{The Cowardly Strategy Fails} \label{fig:BadSurvivorStrat}
\end{figure}
