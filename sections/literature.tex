The following is a summary of results relevant to the Game of Zombies and Survivors.

\subsection{Cops and Robbers, Cop-Number}

Study of vertex pursuit games is first attributed to Quilliot \cite{quilliot1978jeux, quilliot1983problemes}, and Nowakowski and Winkler \cite{nowakowski1983vertex}.
Both authors independently consider games of Cops and Robbers with a single Cop and characterize by way of a relation those graphs where the Cop always wins. These are now known as Cop-win graphs
and can be recognized by the existence of an ordering of the vertices called a \textit{dismantling};
so-called because it is the successive deletion of \textit{corners} resulting in a single vertex
(see the last section on dismantlings, cop-win trees and visiblity graphs ).

The Cop-number of a graph (denoted $c(G)$) is introduced by Aigner and Fromme in \cite{aigner1984game}) and defined as the minimum number of Cops required
to guarantee a Cop win on a graph $G$.

It is possible to recognize graphs for which $k$ Cops are guaranteed to win, i.e. $c(G) = k$ \cite{berarducci1993cop} and \cite{hahn2003characterisation}.
A graph is $k$-cop win if and only if there
exists a function (on a $k$-product of the graph to represent the position of the Cops)
which satisfies certain properties; essentially it is a function which takes as input a position $C$ of Cops
and returns the next position for the Cops that guarantees a win (see \cite{bonato2011game}[p. 119]).
There exists a polynomial-time algorithm for deciding whether a graph is $k$-Cop-win by iteratively
solving for this function.

Another important line of inquiry relating to the Cop-number is the investigation of Meyniel's conjecture, which posits that $\bigO(\sqrt{n})$ is an upper bound on the Cop-number \cite{frankl1987cops}.
Incremental progress has been made on special classes of graphs as well as for graphs in general. See also for a recent
overview \cite{gera2016graph}[p. 31].

\subsection{The Cop-Number and the Genus of the Graph}

One of the most surprising results about the Game of Cops and Robbers are owed to Aigner and Fromme \cite{aigner1984game}, who showed that the cop number of a planar graph is at most 3.
Basically, a graph is planar if it can be drawn in the plane (say, on a piece of paper) without crossing any edges. Aigner and Fromme describe a 3-Cop strategy which uses \textit{isometric} paths of the graph to encircle and entrap the Robber.

Outerplanar graphs are planar graphs whose vertices all belong to the outer face.  \cite{clarke2002constrained} shows that the cop number of outerplanar graphs is 2 by considering
two possible cases: those with and without cut vertices.

Planar graphs of order $n$ \cite{bonato2017topological}

\subsection{Relation to the Girth and Minimum Degree of a Graph}

Aigner and Fromme also show a relationship between the Cop-number, the girth of a graph and
its minimum degree \cite{aigner1984game}. More precisely, if $G$ has girth at least 5, then $c(G)\geq \delta(G)$ where $\delta(G)$ is the minimum degree of $G$.

This result has since been refined by \cite{} and again recently in \cite{}.


\subsection{Dismantlings, Cop-win Trees, Zombie-win Trees}

Quilliot and Nowakowski both indepedently characterized cop-win graphs as those which
admit a \textit{dismantling}.

A (one-point) retract is an edge preserving function $f : G \mapsto H = G \setminus v$
(aka a homomorphism) such that $f(v) = x$ for some $x \neq v \in V(G)$ and $f$ restricted on $H$ is the identity.
Formally,

\[ f(v) = x \qquad f(u) = u \qquad \forall u \in V(G)\setminus \{ v \} \]

and
\[ xy \in E(G) \implies f(x)f(y) \in E(G \setminus \{ v \}) \]

If $G$ is a reflexive graph, then a one-point retract can be seen as joining
two vertices. The edge between two adjacent vertices becomes another loop.
The retract maps a graph $G$ to graph $G'$ with one less vertex.

Recall that corners are vertices $v$ whose closed neighbourhoods
are a subset of a neighbours' closed neighbourhood, i.e.

\[u,v\in V(G) \qquad \text{and} \qquad N[v] \subseteq N[u] \]

You can define a retract on corner $v$: if $v$ is a corner, then it is
dominated by some $u \in V(G)$. So if $x \in V(G)$, $x \neq v$ and
$xv \in E(G)$ then $xu \in E(G)$ by definition of a corner. Therefore the map

\[ f(x) = \begin{cases}
u & \text{if } x = v \\
x & \text{otherwise}
\end{cases} \]

is edge preserving since $f(x)f(v) = xu$ and $xu \in E(G)$, so $xu \in E(H) = E(G - v)$.
For other vertices $x,y \not\in \{u,v\}$, $f(x)f(y) = xy \in E(G)$ so $f(x)f(y) \in E(G- v)$ also.
This shows that $f$ is a homomorphism as required and hence a retract.

This is a formal way of saying that a corner of a graph can be folded into a
dominating vertex: an astute Robber would never move into a corner.

A dismantling is a sequence of retracts $f_1, f_2, \dots, f_{n-1}$ such that the
composition $F_{n-1} = f_{n-1} \circ f_{n-2} \circ \cdots f_2 \circ f_1$ gives a
function for which $F_{n-1} (G) = K_1$. That is, there is a sequence of retracts
which maps the graph to a single vertex.

Not all vertices of a graph need be corners in order for there
to exist a dismantling: it suffices to have an ordering where each $v_i$ is a corner in
$G[v_i, v_{i+1}, \dots, v_n]$.

Such a sequence of $f_j$'s defines a copwin ordering
\[ \mathcal{O} = \{ v_1, v_2, \dots, v_n\} \]
 where $v_1$ is a corner in $G_1 = G$, $v_2$ is a corner in $G - v_1$, and so on.

A fundamental result in C \& R is that copwin graphs -- graphs for which a single
cop is guaranteed to win --  are characterized by the existence of such dismantlings.
A graph is copwin if and only if it is dismantlable.


A Cop-win spanning tree combines the idea of a dismantling with a spanning tree
and was first proposed in \cite{clarke2002constrained}.

A copwin spanning tree $S$ is defined as a tree where $x,y\in V(G)$
$xy \in E(S)$ if there exists a retract $f_j$ in the dismantling
$F_n = f_{n} \circ f_{n-1} \circ \cdots \circ f_{2} \circ f_1$
such that $f_j (x) = y$ or $f_j (y) = x$.

Copwin spanning trees give a strategy for the cops to follow: start at the root
(the last vertex in the ordering) and descend the tree in the branch containing the robber.
Lemmas 2.1.2 and 2.1.3 from \cite{clarke2002constrained} show that the cop
can always stay in the same branch (and above) the robber in the tree. So the
robber is eventually stuck in a leaf and caught.

Zombies cannot apply a cornering strategy, however. Fitzpatrick showed that a
graph is zombie-win if a specific spanning tree exists \cite{fitzpatrick2016deterministic}:

Theorem 6. [Fitzpatrick] If there exists a breadth-first search of a graph G such that the associated spanning tree is also a cop-win spanning tree, then G is zombie-win.

Thus a sufficient condition for zombie-win graphs are those for which a specific copwin tree exists: one equivalent to a breadth-first search of the graph from some vertex. It remains unclear if it is also a necessary condition.

A few questions: are copwin graphs necessarily zombie win? (No. Smallest counter example?)
What is the dismantling of this copwin but not-zombie win graph. Since a dismantling exists,
a copwin spanning tree exists.

Here is an example of a graph and two dismantlings, one of which results in a BFS tree,
and the other does not.

\begin{figure}[h!]
\centering
\includegraphics[scale=0.70]{copwin_tree/copwin_tree_example1}
\end{figure}

Here are two dismantlings, their orderings, and the resulting copwin spanning trees.
\begin{align*}
  f_1(b) = f \\
  f_2(c) = d \\
  f_3(f) = e \\
  f_4(a) = e \\
  f_5(e) = g \\
  f_6(d) = g
\end{align*}

Gives ordering $\mathcal{O}_1 = \{ b, c, f, a, e, d, g \}$. Whereas

\begin{align*}
  g_1(b) = f \\
  g_2(a) = e \\
  g_3(c) = d \\
  g_4(f) = d \\
  g_5(e) = d \\
  g_6(g) = d
\end{align*}

Also gives a dismantling with ordering $\mathcal{O}_2 = \{b, a, c, f, e, g, d \}$.
But only the second produces a copwin tree obtainable as a bread-first search.

\begin{figure}[h!]
\centering
\includegraphics[scale=0.70]{copwin_tree/copwin_tree_example1_trees}
\end{figure}

Moreover, it would seem that a zombie loses if it starts on $g$, but not on $d$.


\subsection{Probabilistic zombies}

Zombies are often depicted as mindless or aimless. It is a common trope that zombies
idle around, moving in random directions until they somehow (suddenly) distinguish
the uninfected. It is only at this point that the zombies will charge.

Such behavior likely inspired another type of pursuit game in which the zombies start randomly on
the graph. Once the survivor chooses a start vertex, the zombies ``notice'' the survivor and start
moving directly towards it.

Without knowing where the zombies start, however, it is impossible to know the outcome with certainty.
So study of these games becomes probabilistic; zombies win if they have at least a 50\% chance of winning.
The (probabilistic) zombie number of a graph is the number of zombies required for a 50\% chance of winning and this zombie number is
obtained for several classes of graphs in \cite{bonato2016probabilistic} and for
toroidal grids in \cite{pralat2019many}.

\subsection{Deterministic zombies}

This is the version of the game we study below.
