
% Math and Comp Sci Student
% Homework template by Joel Faubert
% Ottawa University
% Winter 2018
%
% This template uses packages fancyhdr, multicol, esvect and amsmath
% http://ctan.cms.math.ca/tex-archive/macros/latex/contrib/esvect/esvect.pdf
% http://www.ams.org/publications/authors/tex/amslatex
%
% These packages are freely distribxted for use but not
% modification under the LaTeX Project Public License
% http://www.latex-project.org/lppl.txt

\documentclass[letterpaper, 10pt]{article}
% \usepackage[text={8in,10in},centering, margin=1in,headheight=28pt]{geometry}
\usepackage[margin=1in, centering, headheight=28pt]{geometry}
\usepackage{fancyhdr}
\usepackage{esvect}
\usepackage{amsmath}
\usepackage{bbold}
\usepackage{amsfonts}
\usepackage{amssymb}
\usepackage{amsthm}
\usepackage{mathrsfs}
\usepackage{mathtools}
\usepackage{multicol}
\usepackage{enumitem}
\usepackage{verbatimbox}
\usepackage{fancyvrb}
\usepackage{hyperref}
\usepackage[pdftex]{graphicx}
\usepackage{bm}
\usepackage{color}
%\usepackage{minted}

%\usepackage{sxbfigxre}

% Configure margins
\pagestyle{fancy}
% \hoffset -0.75pt
% \voffset -0.8pt
% \oddsidemargin 0pt
% \topmargin 0pt
% \headheight 25pt
% \headsep 20pt
% \textheight 8.25in
% \textwidth 6.25 in
% \marginparsep 5pt
% \marginparwidth 0.5in
% \footskip 10pt
% \marginparpush 0pt
\paperwidth 8.5in
\paperheight 11in

% Configxre headers and footers

\lhead{Prof. Jean-Lou De Carufel }
\rhead{Jo\"el Faubert \\ Student \# 2560106}
\chead{Zombies \& Survivors Working Document \\ 10-7-2018}
\rfoot{\today}
\fancyhfoffset[l]{40pt}
\fancyhfoffset[r]{40pt}
\renewcommand{\headrulewidth}{0.4pt}
\renewcommand{\footrulewidth}{0.4pt}
\setlength{\parskip}{10pt}
\setlist[enumerate]{parsep=5pt, itemsep=0pt}

% Define shortcuts

\newcommand{\floor}[1]{\lfloor #1 \rfloor}
\newcommand{\ceil}[1]{\lceil #1 \rcleil}

% matrices
%\newcommand{\bpm}{\begin{bmatrix}}
%\newcommand{\epm}{\end{bmatrix}}
%\newcommand{\vm}[3]{\begin{bmatrix}#1\\#2\\#3\end{bmatrix}}
%\newcommand{\Dmnt}[9]{\begin{vmatrix}#1 & #2 & #3 \\ #4 & #5 & #6 \\ #7 & #8 & #9 \end{vmatrix}}
%\newcommand{\dmnt}[4]{\begin{vmatrix}#1 & #2 \\ #3 & #4 \end{vmatrix}}
%\newcommand{\mat}[4]{\begin{bmatrix}#1 & #2\\#3 & #4\end{bmatrix}}

% common sets
\newcommand{\R}{\mathbb{R}}
\newcommand{\Qu}{\mathbb{Q}}
\newcommand{\Na}{\mathbb{N}}
\newcommand{\Z}{\mathbb{Z}}
\newcommand{\Rel}{\mathcal{R}}
\newcommand{\F}{\mathcal{F}}
\newcommand{\U}{\mathcal{U}}
\newcommand{\V}{\mathcal{V}}
\newcommand{\K}{\mathcal{K}}
\newcommand{\M}{\mathcal{M}}

% Power set
\newcommand{\PU}{\mathcal{P}(\mathcal{U})}

%norm shortcut
\DeclarePairedDelimiter{\norm}{\lVert}{\rVert}

% projection, vectors
\DeclareMathOperator{\proj}{Proj}
\newcommand{\vctproj}[2][]{\proj_{\vv{#1}}\vv{#2}}
\newcommand{\dotprod}[2]{\vv{#1}\cdot\vv{#2}}
\newcommand{\uvec}[1]{\boldsymbol{\hat{\textbf{#1}}}}

% derivative
\def\D{\mathrm{d}}

% big O
\newcommand{\bigO}{\mathcal{O}}

% probability
\newcommand{\Expected}{\mathrm{E}}
\newcommand{\Var}{\mathrm{Var}}
\newcommand{\Cov}{\mathrm{Cov}}
\newcommand{\Entropy}{\mathrm{H}}
\newcommand{\KL}{\mathrm{KL}}

\DeclareMathOperator*{\argmax}{arg\,max}
\DeclareMathOperator*{\argmin}{arg\,min}

\setlist[enumerate]{itemsep=0pt, partopsep=5pt, parsep=0pt}
\setlist[itemize]{itemsep=0pt, partopsep=5pt, parsep=0pt}

\begin{document}

\theoremstyle{definition}
\newtheorem{definition}{Definition}
\newtheorem{theorem}{Theorem}
\newtheorem{proposition}{Proposition}
\newtheorem{corollary}{Corollary}
\newtheorem{lemma}{Lemma}
\newtheorem{proofpart}{Part}
\makeatletter
\@addtoreset{proofpart}{theorem}
\makeatother

EDITS: Conjecture on upper and lower bounds of Delta

Prove said conjecture

Modify code to remove values of delta which will not permit offset for general zombie start

Find all possible zombie positions where $z_2$ starts on smaller cycle.

Run Floyd-Warshall on torus grid.

Recurse backwards from "zombie loss" positions to find all losing zombie positions

\section{Zombie number of cycle with one chord $Q_{m,n}$}

We analyze the Game of Zombies \& Survivors on a cycle with a single chord.

\begin{definition}
 Take a cycle of length $m+n$ and add a chord which
 divides the cycle into paths $P_m$ and $P_n$ of lengths $m$ and $n$.
 Without loss of generality $m \leq n$. We denote such a cycle as $Q_{m,n}$.
\end{definition}

\begin{center}
 \includegraphics[scale=0.20]{Q_m_n_basic}
\end{center}

\begin{theorem}
 The single-chord cycle $Q_{m,n}$ is 2-zombie win.
\end{theorem}

\begin{proof}
 First, we show that a certain game state is a losing position for the survivor.
 Second, we show how to position the zombies at the start of the game so that --
 no matter where the survivor starts -- a losing position is inevitably reached.

 We use the following notation.
 Denote as $P_m$ and $P_n$ the paths of lengths $m$ and $n$ respectively.
 Call the endpoints for the chord $u$ and $v$.
 We think of $Q_{m,n}$ as embbeded in the plane with $P_m$
 -- the shortest side -- on the left.
 This does not limit the generality of the following and allows us to define (counter-)clockwise distance: the length of the path
 along the outer cycle with respect to this embedding.

 Following the rules of the game, the zombies always move along a shortest path toward the survivor we call $zs$-paths. Let
 $Z_k = \{  \exists \ell : z_k = u_{i,0}, u_{i, 1}, u_{i, 2}, \dots, u_{i, \ell-1}, s= u_{i, \ell}\}$ be the set of $i$ different $zs$-paths of length $\ell$ for zombie $k$.

 There is at least one such path since our graph is presumed connected,
 thus $1 \leq i \leq p$.

 If there is only one path, then $z_k$'s next move is $u_{i, 1}$. If all $zs$-paths
 include $u_{i,1}$, then again $z_k$'s next move must be to that vertex.

 If, however, there are multiple $zs$-paths which have different first moves,
 then the zombie could make multiple moves.

 If $P_1$ and $P_2$ are two possible $zs$-paths with distinct next moves and
 \[ |P_1| \leq |P_2| \]
 then our argument we will suppose that the zombie follows $|P_1|$ since that is a valid move.

 \newpage

 \begin{proofpart} Cornering the Survivor on the Smallest Cycle

  Suppose that the game has reached the following state:
  \begin{itemize}
   \item the first zombie is on an endpoint of the chord, say $v$
   \item the second zombie is $\Delta$ vertices away from $u$, the other endpoint (counting clockwise from $u$ to $z_2$).
   \item the survivor is somewhere on $P_m \setminus N(v)$.
  \end{itemize}
  Denote as $\ell$ the length of the clockwise
  path from $z_1 = v$ to $s$. Note that we must have $2 \leq \ell \leq m-1$ else the survivor loses on the next round.
  \begin{center}
   \includegraphics[scale=0.20]{diagram1}
  \end{center}

  Because $z_1$ and $s$ are on the sub-cycle $C_{m+1}$ formed by
  $P_m + \{uv\}$, $z_1$'s move on the next turn depends on the
  value of $\ell$. If $2\ell < m+1$ then $z_1$ goes clockwise
  on the subcycle. If $2\ell > m+1$ then $z_1$ takes the chord and
  goes counter-clockwise.  If we have equality, then $z_1$ may
  choose either direction since both are paths of equal lengths.

  We assume that once $z_1$ chooses a direction on its move from $v$,
  that it will continue in that direction:
  either the zombie has no choice or both directions around
  the cycle are of the same length (and so
  may continue in the same direction).

  We can also assume that on its turn the survivor will move away from
  $z_1$ and keep a distance of $\ell$ (or $m-\ell +1$):
  any winning survivor strategy involving waiting a turn
  or moving backwards is equivalent to a survivor strategy which always moves but starts with a smaller (or larger) value of $\ell$.

  Since $z_1$ is already on the same cycle as the survivor, it only has two options:

  \begin{itemize}
   \item[A.] $z_1$ goes clockwise if $\ell \leq 1 + m - \ell$.
         Combined with the bounds on $\ell$, this gives $4 \leq 2 \ell \leq m + 1$

   \item[B.] $z_1$ goes counter-clockwise if $1 + m - \ell \leq \ell$.
         Combined with the bounds on $\ell$, we obtain $m + 1 \leq 2 \ell \leq 2m - 2$
  \end{itemize}

  For $z_2$, there are four possible shortest paths to the survivor:

  \begin{itemize}
   \item $P_a$ of length $\Delta + (m - \ell)$
   \item $P_b$ of length $\Delta + 1 + \ell$
   \item $P_c$ of length $(n-\Delta) + 1 + (m-\ell)$
   \item $P_d$ of length $(n-\Delta) + \ell$
  \end{itemize}

  Comparing path lengths we see that:

  \begin{itemize}
   \item[I.] $z_2$ moves counter-clockwise if either $|P_a| \leq \min \{ |P_c|, |P_d| \}$ or $|P_b| \leq \min \{ |P_c|, |P_d| \}$.

   \item[II.] $z_2$ goes clockwise if either $|P_c| \leq \min \{ |P_a|, |P_b| \}$ or $|P_d| \leq \min \{ |P_a|, |P_b| \}$.
  \end{itemize}

  We will examine all combinations of the possible decisions
  made by the zombies from this configuration:

  \begin{itemize}
   \item I. $z_2$ goes counter-clockwise
   \item II. $z_2$ goes clockwise.
   \item A. $z_1$ goes clockwise
   \item B. $z_1$ goes counter-clockwise
  \end{itemize}

  \textit{Case I.A} We have the following constraint on $\ell$ from
  assumption A.
  \begin{align*}
   4 \leq 2 \ell \leq m + 1
  \end{align*}
  and the following constraints on $\Delta$ from assumption I.
  \begin{align*}
   \Delta + (m - \ell) \leq & n - \Delta + 1 + m - \ell & \text{and} \\
   \Delta + (m - \ell) \leq & n - \Delta + \ell
  \end{align*}
  \begin{center}or\end{center}
  \begin{align*}
   \Delta + 1 + \ell \leq & n - \Delta + 1 + m - \ell & \text{and} \\
   \Delta + 1 + \ell \leq & n - \Delta + \ell
  \end{align*}
  These can be simplified further with a bit of algebra and  assumption A:
  \begin{align*}
   2 \Delta \leq & n+1                    & \text{and} \\
   2 \Delta \leq & n - m + 2\ell \leq n+1
  \end{align*}
  or
  \begin{align*}
   2 \Delta \leq & n+m -2 \ell             & \text{and} \\
   2 \Delta \leq & n -1 \leq n + m - 2\ell
  \end{align*}
  So for $z_2$ to follow either $P_a$ or $P_b$ and go counter-clockwise we must have
  \begin{align*}
   2 \Delta \leq & n - m + 2\ell & \text{or} \\
   2 \Delta \leq & n - 1                     \\
  \end{align*}

  Next we consider: which of $s$ or $z_2$ reaches $u$ first?
  If $\Delta = m - \ell$ both reach $u$ at the same time, with the survivor moving onto the $z_2$-occupied vertex (and losing).
  If we have $\Delta = m - \ell + 1$, then $s$ reaches $u$ first
  but is caught by $z_2$ on the following round.
  So, to guarantee the survivor has not escaped $P_m$ we need
  \begin{align*}
   \Delta \leq & m- \ell + 1 \\
  \end{align*}
  otherwise the survivor reaches the chord at least two rounds
  before $z_2$ can move to block. We wish to prevent this scenario since
  the survivor could then take the chord and possibly escape, pulling
  both zombies into a loop on $C_{n}$.

  Lastly, to ensure that $z_2$ moves counter-clockwise once
  it reaches $u$ and traps the survivor we need
  \begin{align*}
   m - \ell - \Delta \leq & 1 + \Delta + \ell \\
   2 \Delta \geq          & m - 2\ell  -1     \\
  \end{align*}

  When we combine all the restrictions we obtain

  \textit{Case I.A. Summary}

  $z_1$ goes clockwise:
  \begin{align*}
   4 \leq & 2 \ell \leq m + 1
  \end{align*}
  and $z_2$ goes counter-clockwise
  \begin{align*}
   2 \Delta \leq & n - m + 2\ell & \text{or} \\
   2 \Delta \leq & n - 1                     \\
  \end{align*}
  the zombies win:
  \begin{align*}
   2 \Delta \leq      & 2 m- 2 \ell + 2 & \text{and} \\
   m - 2\ell  -1 \leq & 2 \Delta                     \\
  \end{align*}

  \textit{Case I.B}
  From assumption B and the constraint on $\ell$, we must have
  \begin{align*}
   m + 1 \leq 2 \ell \leq 2m - 2
  \end{align*}
  and the constraints on $\Delta$ from assumption I are again:
  \begin{align*}
   \Delta + (m - \ell) \leq & n - \Delta + 1 + m - \ell & \text{and} \\
   \Delta + (m - \ell) \leq & n - \Delta + \ell
  \end{align*}
  \begin{center}or\end{center}
  \begin{align*}
   \Delta + 1 + \ell \leq & n - \Delta + 1 + m - \ell & \text{and} \\
   \Delta + 1 + \ell \leq & n - \Delta + \ell
  \end{align*}

  These can be simplified using assumption B:
  \begin{align*}
   2 \Delta \leq & n+1 \leq n-m+2\ell & \text{and} \\
   2 \Delta \leq & n - m + 2\ell
  \end{align*}
  or
  \begin{align*}
   2 \Delta \leq & n+m -2 \ell \leq n-1 & \text{and} \\
   2 \Delta \leq & n -1
  \end{align*}

  So for $z_2$ to go counter-clockwise in this case we must have
  \begin{align*}
   2 \Delta \leq & n + 1         & \text{or} \\
   2 \Delta \leq & n + m - 2\ell
  \end{align*}

  Again we must consider who reaches the chord first. We have assumed that $z_1$ is going counter-clockwise. If $\ell = \Delta$, then $z_2$ reaches $u$ and $s$ reaches $v$ on the same round, and therefore $s$ will be caught on the next. Therefore, to guarantee the survivor has not escaped $P_m$ in this scenario we need
  \begin{align*}
   \Delta \leq & \ell
  \end{align*}
  otherwise the survivor reaches the chord before $z_2$ and
  could escape.

  Then, to ensure that $z_2$ traps the survivor by going clockwise once
  it reaches $u$ we need
  \begin{align*}
   1 + \ell - \Delta \leq & \Delta -1 + m - \ell + 1 \\
   2\ell - m + 1 \leq     & 2 \Delta
  \end{align*}

  \textit{Case I.B. Summary}

  $z_1$ goes counter-clockwise:
  \begin{align*}
   m + 1 \leq 2 \ell \leq 2m - 2
  \end{align*}
  and $z_2$ goes counter-clockwise
  \begin{align*}
   2 \Delta \leq & n + 1         & \text{or} \\
   2 \Delta \leq & n + m - 2\ell
  \end{align*}
  the zombies win:
  \begin{align*}
   2 \Delta \leq      & 2 \ell   \\
   2\ell - m + 1 \leq & 2 \Delta
  \end{align*}

  \textit{Case II.A} We have the following constraint on $\ell$ from
  assumption A.
  \begin{align*}
   4 \leq 2 \ell \leq m + 1
  \end{align*}
  and the following constraints on $\Delta$ from assumption II.
  \begin{align*}
   n - \Delta + \ell \leq & \Delta + (m - \ell) & \text{and} \\
   n - \Delta + \ell \leq & \Delta + 1 + \ell
  \end{align*}
  \begin{center}or\end{center}
  \begin{align*}
   n - \Delta + 1 + m - \ell \leq & \Delta + (m - \ell) & \text{and} \\
   n - \Delta + 1 + m - \ell \leq & \Delta + 1 + \ell
  \end{align*}
  These can be simplified further with a bit of algebra:
  \begin{align*}
   n-m +2\ell \leq & 2 \Delta & \text{and} \\
   n-1 \leq        & 2\Delta
  \end{align*}
  or
  \begin{align*}
   n + 1 \leq         & 2 \Delta & \text{and} \\
   n + m - 2\ell \leq & 2 \Delta
  \end{align*}
  Which we can combine and deduce that (footnote: see appendix I).
  \begin{align*}
   n -m + 2\ell \leq & 2 \Delta & \text{and} \\
   n-1 \leq          & 2 \Delta
  \end{align*}

  Again we must consider who reaches the chord first. We have assumed that $z_1$ is going clockwise. If $m - \ell = n - \Delta$, then $z_2$ reaches $v$ and $s$ reaches $u$ on the same round, and therefore $s$ will be caught on the next. Therefore, to guarantee the survivor has not escaped $P_m$ in this scenario we need
  \begin{align*}
   n - \Delta \leq & m - \ell     \\
   \Delta \geq     & n - m + \ell
  \end{align*}
  otherwise the survivor reaches the chord before $z_2$ and
  could escape.

  Then, to ensure that $z_2$ takes the chord and goes counter-clockwise once
  it reaches $v$, we need
  \begin{align*}
   1 + m - \ell - (n - \Delta) \leq & n - \Delta + \ell  \\
   2 \Delta \leq                    & 2n + 2\ell - m - 1
  \end{align*}

  \textit{Case II.A. Summary}

  $z_1$ goes clockwise:
  \begin{align*}
   4 \leq 2 \ell \leq m + 1
  \end{align*}
  and $z_2$ goes clockwise
  \begin{align*}
   n -m + 2\ell \leq & 2 \Delta & \text{and} \\
   n-1 \leq          & 2 \Delta
  \end{align*}
  the zombies win:
  \begin{align*}
   2 \Delta \geq & 2n - 2m + 2\ell    \\
   2 \Delta \leq & 2n + 2\ell - m - 1
  \end{align*}

  \textit{Case II.B}  We have the following constraint on $\ell$ from
  assumption B.
  \begin{align*}
   m + 1 \leq 2 \ell \leq 2m - 2
  \end{align*}
  and the following constraints on $\Delta$ from assumption II.
  \begin{align*}
   n - \Delta + \ell \leq & \Delta + (m - \ell) & \text{and} \\
   n - \Delta + \ell \leq & \Delta + 1 + \ell
  \end{align*}
  \begin{center}or\end{center}
  \begin{align*}
   n - \Delta + 1 + m - \ell \leq & \Delta + (m - \ell) & \text{and} \\
   n - \Delta + 1 + m - \ell \leq & \Delta + 1 + \ell
  \end{align*}
  These can be simplified further with a bit of algebra:
  \begin{align*}
   n-m+2\ell \leq & 2 \Delta & \text{and} \\
   n-1 \leq       & 2\Delta
  \end{align*}
  or
  \begin{align*}
   n+1 \leq        & 2 \Delta & \text{and} \\
   n+m-2\ell  \leq & 2 \Delta
  \end{align*}

  From which we conclude that $n+1 \leq 2 \Delta$ (see appendix II).

  Now we consider: which of $s$ or $z_2$ reaches $v$ first?
  If $n - \Delta = \ell$, then they both reach $u$ at the same time,
  with the survivor moving onto the $z_2$-occupied vertex (and losing).
  If we have $n - \Delta = \ell + 1$, then $s$ reaches $u$ first
  but is caught by $z_2$ on the following round.
  So, to guarantee the survivor has not escaped $P_m$ we need
  \begin{align*}
   n - \Delta \leq & \ell + 1 \\
  \end{align*}
  otherwise the survivor reaches the chord before $z_2$ can move
  to block. If the survivor reaches the chord first, then it could
  take the chord and possibly escape. (more detail??)

  Then, to ensure that $z_2$ takesgoes clockwise once
  it reaches $v$, we need
  \begin{align*}
   \ell - (n - \Delta) \leq & 1 + (n - \Delta - 1) + (m - \ell + 1) \\
   2 \Delta \leq            & 2n + m - 2\ell + 1
  \end{align*}

  \textit{Case II.B. Summary}

  $z_1$ goes counter-clockwise:
  \begin{align*}
   m + 1 \leq 2 \ell \leq 2m - 2
  \end{align*}
  and $z_2$ goes clockwise
  \begin{align*}
   n+1 \leq & 2 \Delta
  \end{align*}
  the zombies win:
  \begin{align*}
   n - \Delta \leq & \ell + 1           \\
   2 \Delta \leq   & 2n + m - 2\ell + 1
  \end{align*}


 \end{proofpart}

 \begin{proofpart} Forcing the Survivor into a Losing Position.
  We now consider the game on this graph in general and show
  how we can guarantee the survivor will be caught.

  Given $m, n$ and $\Delta$, we place the
  zombies on $C_{n+1}$ so that the zombies move in
  opposite direction wherever the survivor may start.
  We need only consider the cycle $C_{n+1}$ since, if the survivor
  starts on $C_{m+1} \setminus \{u, v\}$, then the zombies play as
  though the survivor is on $u$ or $v$.

  We choose $k$ such that positioning
  \begin{enumerate}
   \item $z_2$ at $\Delta + k$ clockwise from $u$
   \item $z_1$ at $k$ counter-clockwise from $v$
  \end{enumerate}
  forces the survivor into a losing position: it is either immediately sandwiched on $C_{n+1}$, or inevitably on $C_{m+1}$. (EDIT: elaborate)

  The survivor cannot be next to the zombies else it loses right away.
  So we choose $k$ such that, even if the survivor is as far
  away from one of the zombies as possible, then the zombies still move in opposite directions. This leads to the following inequalities


  \begin{align*}
   n - \Delta - 2k - 2 \leq & \Delta + k +1 + k +2 & \text{and} \\
   \Delta + 2k -1 \leq      & n - 2\Delta +3
  \end{align*}
  Solving for $k$ gives
  \[ n - 2\Delta -5 \leq 4k \leq n-2\Delta +3 \]

 \end{proofpart}

 \begin{proofpart} Computing the Winning Zombie Start

  Given $m$ and $n$, we choose $\Delta$ so that whenever we reach
  the scenario described in the first part, the survivor will be cornered.
  (EDIT: finish)
 \end{proofpart}
\end{proof}

\newpage
\appendix
\section{Simplifying $z_2$'s inequalities for Case II.A.}
We have
\begin{align*}
 n - \Delta + \ell \leq & \Delta + (m - \ell) & \text{and} \\
 n - \Delta + \ell \leq & \Delta + 1 + \ell
\end{align*}
\begin{center}or\end{center}
\begin{align*}
 n - \Delta + 1 + m - \ell \leq & \Delta + (m - \ell) & \text{and} \\
 n - \Delta + 1 + m - \ell \leq & \Delta + 1 + \ell
\end{align*}
These can be simplified further with a bit of algebra:
\begin{align*}
 n-m +2\ell \leq & 2 \Delta & \text{and} \\
 n-1 \leq        & 2\Delta
\end{align*}
or
\begin{align*}
 n + 1 \leq         & 2 \Delta & \text{and} \\
 n + m - 2\ell \leq & 2 \Delta
\end{align*}

These inequalites are of the form
\begin{align*}
 n-x \leq & 2 \Delta & \text{and} \\
 n-1 \leq & 2\Delta
\end{align*}
\begin{center}or\end{center}
\begin{align*}
 n + x \leq & 2 \Delta & \text{and} \\
 n + 1 \leq & 2 \Delta
\end{align*}

Where $x = m -2\ell$.

Supposing $x\geq 0$, we have
\begin{align*}
 n-x \leq & n+x \leq 2 \Delta & \text{and} \\
 n-1 \leq & n+1 \leq 2\Delta
\end{align*}

Whereas if $x <0$, then from assumption A we must have $m-2\ell = -1$, so that
our constraints reduce to
\begin{align*}
 n+1 \leq & 2 \Delta & \text{and} \\
 n-1 \leq & 2 \Delta
\end{align*}

\newpage
\section{Simplifying $z_2$'s inequalities for Case II.B.}

We have
\begin{align*}
 n - \Delta + \ell \leq & \Delta + (m - \ell) & \text{and} \\
 n - \Delta + \ell \leq & \Delta + 1 + \ell
\end{align*}
\begin{center}or\end{center}
\begin{align*}
 n - \Delta + 1 + m - \ell \leq & \Delta + (m - \ell) & \text{and} \\
 n - \Delta + 1 + m - \ell \leq & \Delta + 1 + \ell
\end{align*}
These can be simplified further with a bit of algebra:
\begin{align*}
 n-m+2\ell \leq & 2 \Delta & \text{and} \\
 n-1 \leq       & 2\Delta
\end{align*}
or
\begin{align*}
 n+1 \leq        & 2 \Delta & \text{and} \\
 n+m-2\ell  \leq & 2 \Delta
\end{align*}

These inequalites are of the form
\begin{align*}
 n-x \leq & 2 \Delta & \text{and} \\
 n-1 \leq & 2\Delta
\end{align*}
\begin{center}or\end{center}
\begin{align*}
 n + 1 \leq & 2 \Delta & \text{and} \\
 n + x \leq & 2 \Delta
\end{align*}

Where $x = m -2\ell$. Now since assumption B gives $m - 2\ell \leq -1$, we
see that
\begin{align*}
 n-1 \leq & n-x \leq 2 \Delta \\
          & \text{or}         \\
 n+x \leq & n+1 \leq 2 \Delta
\end{align*}

\end{document}
