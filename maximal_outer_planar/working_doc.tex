% Math and Comp Sci Student
% Homework template by Joel Faubert
% Ottawa University
% Winter 2018
%
% This template uses packages fancyhdr, multicol, esvect and amsmath
% http://ctan.cms.math.ca/tex-archive/macros/latex/contrib/esvect/esvect.pdf
% http://www.ams.org/publications/authors/tex/amslatex
%
% These packages are freely distribxted for use but not
% modification under the LaTeX Project Public License
% http://www.latex-project.org/lppl.txt

\documentclass[letterpaper, 10pt]{article}
% \usepackage[text={8in,10in},centering, margin=1in,headheight=28pt]{geometry}
\usepackage[margin=1in, centering, headheight=28pt]{geometry}
\usepackage{fancyhdr}
\usepackage{esvect}
\usepackage{amsmath}
\usepackage{bbold}
\usepackage{amsfonts}
\usepackage{amssymb}
\usepackage{amsthm}
\usepackage{mathrsfs}
\usepackage{mathtools}
\usepackage{multicol}
\usepackage{enumitem}
\usepackage{verbatimbox}
\usepackage{fancyvrb}
\usepackage{hyperref}
\usepackage[pdftex]{graphicx}
\usepackage{bm}
\usepackage{minted}

%\usepackage{sxbfigxre}

% Configure margins
\pagestyle{fancy}
% \hoffset -0.75pt
% \voffset -0.8pt
% \oddsidemargin 0pt
% \topmargin 0pt
% \headheight 25pt
% \headsep 20pt
% \textheight 8.25in
% \textwidth 6.25 in
% \marginparsep 5pt
% \marginparwidth 0.5in
% \footskip 10pt
% \marginparpush 0pt
\paperwidth 8.5in
\paperheight 11in

% Configxre headers and footers

\lhead{Prof. Jean-Lou De Carufel }
\rhead{Jo\"el Faubert \\ Student \# 2560106}
\chead{Zombies \& Survivors Working Document \\ 2-5-2018}
\rfoot{\today}
\fancyhfoffset[l]{40pt}
\fancyhfoffset[r]{40pt}
\renewcommand{\headrulewidth}{0.4pt}
\renewcommand{\footrulewidth}{0.4pt}
\setlength{\parskip}{10pt}
\setlist[enumerate]{parsep=10pt, itemsep=10pt}

% Define shortcuts

\newcommand{\floor}[1]{\lfloor #1 \rfloor}
\newcommand{\ceil}[1]{\lceil #1 \rcleil}

% matrices
%\newcommand{\bpm}{\begin{bmatrix}}
%\newcommand{\epm}{\end{bmatrix}}
%\newcommand{\vm}[3]{\begin{bmatrix}#1\\#2\\#3\end{bmatrix}}
%\newcommand{\Dmnt}[9]{\begin{vmatrix}#1 & #2 & #3 \\ #4 & #5 & #6 \\ #7 & #8 & #9 \end{vmatrix}}
%\newcommand{\dmnt}[4]{\begin{vmatrix}#1 & #2 \\ #3 & #4 \end{vmatrix}}
%\newcommand{\mat}[4]{\begin{bmatrix}#1 & #2\\#3 & #4\end{bmatrix}}

% common sets
\newcommand{\R}{\mathbb{R}}
\newcommand{\Qu}{\mathbb{Q}}
\newcommand{\Na}{\mathbb{N}}
\newcommand{\Z}{\mathbb{Z}}
\newcommand{\Rel}{\mathcal{R}}
\newcommand{\F}{\mathcal{F}}
\newcommand{\U}{\mathcal{U}}
\newcommand{\V}{\mathcal{V}}
\newcommand{\K}{\mathcal{K}}
\newcommand{\M}{\mathcal{M}}

% Power set
\newcommand{\PU}{\mathcal{P}(\mathcal{U})}

%norm shortcut
\DeclarePairedDelimiter{\norm}{\lVert}{\rVert}

% projection, vectors
\DeclareMathOperator{\proj}{Proj}
\newcommand{\vctproj}[2][]{\proj_{\vv{#1}}\vv{#2}}
\newcommand{\dotprod}[2]{\vv{#1}\cdot\vv{#2}}
\newcommand{\uvec}[1]{\boldsymbol{\hat{\textbf{#1}}}}

% derivative
\def\D{\mathrm{d}}

% big O
\newcommand{\bigO}{\mathcal{O}}

% probability
\newcommand{\Expected}{\mathrm{E}}
\newcommand{\Var}{\mathrm{Var}}
\newcommand{\Cov}{\mathrm{Cov}}
\newcommand{\Entropy}{\mathrm{H}}
\newcommand{\KL}{\mathrm{KL}}

\DeclareMathOperator*{\argmax}{arg\,max}
\DeclareMathOperator*{\argmin}{arg\,min}

\setlist[enumerate]{itemsep=10pt, partopsep=5pt, parsep=10pt}
\setlist[itemize]{itemsep=5pt, partopsep=5pt, parsep=10pt}

\begin{document}

\newtheorem{definition}{Definition}
\newtheorem{theorem}{Theorem}
\newtheorem{proposition}{Proposition}
\newtheorem{corollary}{Corollary}
\newtheorem{lemma}{Lemma}

\begin{definition}
A graph $G$ is \emph{planar} if it can be embedded in the plane such that its edges
intersect only at vertices \cite{bondy2008graph}.
\end{definition}

\begin{definition}
A graph $G$ is said to be \emph{outer-planar} if it is planar and has and embedding
in which all of the vertices of $G$ lie on the outer face \cite{campos2013dominating}.
\end{definition}

\begin{definition}
A graph $G$ is \emph{maximally outer planar} if the addition of any edge violates
 outer-planarity.
\end{definition}

Given an outer planar graph $G$ and an embedding, we define:

\begin{definition}
The \emph{outer face} is the Hamilton cycle formed by the vertice and the edges
connecting them on the outer face.
\end{definition}

\begin{definition}
A \emph{chord} is an edge which is not on the outer face.
\end{definition}

\begin{definition}
An \emph{ear} is vertex of degree 2. A maximally outer-planar graph has at least 2 ears.
\end{definition}

A maximally outer-planar graph is composed of triangles. In \cite{campos2013dominating},
they distinguish two types of triangles:

\begin{definition}
A \emph{marginal} triangle shares at least one edge with the outer face while
 an \emph{internal} triangle has no edges on the outer face.
\end{definition}

There is an interesting relationship between internal triangles and the number
of ears \cite{campos2013dominating}. If $k$ is the number of internal triangles, then there are $k+2$ ears.
This is because it takes 3 marginal triangles to create an internal triangle, and
adding another ear (without violating outer-planarity) creates a new internal triangle.

\begin{definition}
Finally, a maximally outer-planar graph is called \emph{striped} if it does not
contain any internal triangles.
\end{definition}

\begin{definition}
A \emph{round} of the game consists of two turns: the zombie's turn followed by the survivor's turn.
\end{definition}

With these definitions in mind, we seek to prove the following:

\begin{theorem}
Maximally outer-planar graphs are zombie-win, regardless of the zombie's initial position.
That is to say, a single zombie
playing on a maximally outer-planar graph will always catch the survivor by following
the greedy strategy.
\end{theorem}

\begin{proof}
Our approach is to show that the \emph{survivor zone}, the set of vertices $S \subseteq V$
which the survivor can occupy without losing on the next round shrinks at every round of the game.

Observe that $K_3$ is zombie-win, so we can assume that $|V| = n > 3$.

The zombie and survivor choose initial positions $z^{(1)}, s^{(1)} \in V$,
the zombie player going first.

First assume that $z^{(1)}$ is an ear. If $s^{(1)} \in N(z^{(1)})$, then the game is won by the zombie
in round 1.
Otherwise, $z^{(2)} \in N(z^{(1)})$ is not an ear (since $n > 3$) and so has
degree at least 3. The survivor either passes or moves to some $s^{(2)}$ and, by the end of this
round, the zombie is not on an ear.

If the survivor is on an ear adjacent to the zombie, then the zombie captures the survivor on its turn.
Otherwise, we claim that the zombie does not move to an ear. Indeed, if the zombie is on a vertex
without any adjacent ears, then the claim is clear.
Assume then, that the zombie is on a vertex adjacent to an ear.
The survivor is not on this ear vertex by assumption.
Now supposing that the shortest path includes this ear leads to a contradiction by the triangle
inequality.

So the zombie always moves to non-ear vertices unless it is to capture the survivor,
 and thus we can now assume that $z^{(r)}$ (for $r>1$) is a vertex of degree at least 3.

Given an embedding of the graph, label the vertices $v_0, v_1, \dots, v_{n-1}$ clockwise such that
\[H_G = v_0v_1, v_1v_2 \dots, v_{n-2}v_{n-1}, v_{n-1}v_0\] is the Hamilton cycle formed by
the outer face (subscripts modulo $n$).

Now say that $z^{(r)} = v_i$ for some $1 \leq i \leq n$. The vertices $v_{i-1}$
and $v_{i+1}$ are the predecessor and successor of $v_i$ on the outer face.

Since $v_i$ is not an ear and $n>3$, we also have that the set of vertices attached to $v_i$ by a chord:
\[ U = \{ u \in V\setminus \{v_{i-1}, v_{i+1}\} \vert \, v_i u \in E \} = \{v_{k_1}, v_{k_2}, \dots, v_{k_\ell} \}\]
is not empty.

%Let $v_k$ be an arbitrary vertex in $U$.
%Its neighbours on the outer face are $v_{k-1}$ and $v_{k+1}$.

  \begin{center}
  \includegraphics[scale=0.25]{survivor_zone_round_r+1_case_1}
  \end{center}

Notice that because the graph is a triangulation and $v_iv_{k_1}$ is assumed to be the first clockwise
chord from $v_i$, there must also be an edge $v_{i+1}v_{k_1}$.


If the survivor is adjacent to $v_i$ then the zombie captures the survivor.
Otherwise, observe that $s^{(r)}$ is in $\{v_{i+1}, \dots, v_{k_\ell}\}$ or
$\{v_{k_1}, \dots, v_{i-1}\}$ (or both). Assume without loss of generality that
$s^{(r)}$ is in $\{v_{i+1}, \dots, v_{k_\ell}\}$. Therefore, the survivor zone
at round $r$ is $S^{(r)} = \{v_{i+1}, \dots, v_{k_\ell}\} \setminus \left( \{v_{i+1}\} \cup \{v_{k_1}, \dots, v_{k_\ell} \} \right)$.

Now, the shortest $zs$-path either starts with (1) a hull edge or (2) a chord edge.
\begin{enumerate}
  \item If the shortest $zs$-path starts with a hull edge then it is $v_iv_{i+1}$.
  In this case, the survivor must be in $\{v_{i+2}, \dots, v_{k_1 -1 }\}$, else
  we obtain a contradiction using the triangle inequality.
  Since the graph is outer-planar, the survivor cannot leave this set without passing through
  either $v_{i+1}$ or $v_{k_1}$.


  Therefore, the survivor zone at round $r+1$ is
  $S^{(r+1)} \subseteq \{v_{i+2}, \dots, v_{k_1 -1 }\} \setminus \{v_{i+2}\} \subsetneq S^{(r)}$.
  This shows that the survivor zone shrank by at least one vertex.

  \item Assume that the shortest $zs$-path starts with a chord edge. We consider two subcases:
    (a) $s^{(r)} \in \{v_{i+2}, \dots, v_{k_1 -1} \}$, or
    (b) $s^{(r)} \in \{v_{k_1}, v_{k_1 +1}, \dots, v_{k_\ell -1}, v_{k_\ell}\}$

    \begin{enumerate}
      \item Subcase $s^{(r)} \in \{v_{i+2}, \dots, v_{k_1 -1} \}$. In this case, the
      chord edge of the shortest $zs$-path must $v_i v_{k_1}$, since it is the first in clockwise order.
      As in case 1, the survivor cannot escape $\{v_{i+2}, \dots, v_{k_1 -1} \}$ without passing through
      either $v_{i+1}$ or $v_{k_1}$.

      So we see that $S^{(r+1)} \subseteq \{v_{i+2}, \dots, v_{k_1 -1 }\} \setminus \{v_{k_1-1}\} \subsetneq S^{(r)}$
       has again shrunk by at least one vertex.

      \item Subcase $s^{(r)} \in \{v_{k_1}, v_{k_1 +1}, \dots, v_{k_\ell -1}, v_{k_\ell}\}$.
      In this case, there exists $v_{k_j} \in U$ such that
      the survivor is somewhere in the arc defined by $v_{k_j}$ and $v_{k_{j +1}}$.
      That is, the survivor is in $v_{k_j +1}, v_{k_j + 2}, \dots, v_{k_{j+1} -1}, v_{k_{j+1}}$ the vertices of
      the outer face within the span of said arc.
      Observe that the first edge on the shortest $zs$-path must be $v_iv_{k_j}$ or $v_iv_{k_{j +1}}$.

      \begin{center}
      \includegraphics[scale=0.25]{survivor_zone_round_r+1_case_2b}
      \end{center}

      As before, the survivor now cannot escape $\{v_{k_j}, v_{k_j +1}, \dots, v_{k_{j+1}} \}$ without passing through
      either $v_{k_j}$ or $v_{k_{j+1}}$.
      Since $v_iv_{k_j}$ and $v_iv_{k_{j+1}}$ are two consecutive clockwise chords attached to $v_i$ and the
      graph is maximally outer planar, we must have edge $v_{k_j}v_{k_{j+1}}$.
      Therefore, $S^{(r+1)} \subseteq \{v_{k_j}, v_{k_j +1}, \dots, v_{k_{j+1}} \}
      \setminus \{ v_{k_j}, v_{k_j +1}\} \subsetneq S^{(r)}$.


    \end{enumerate}

\end{enumerate}

In these three exhaustive cases, we have shown that the survivor zone shrinks by at least one vertex.
Since the graph is finite, the survivor is inevitably restricted to a single vertex adjacent
to the zombie and the survivor is caught on the next round.

\end{proof}

\bibliographystyle{IEEEtran}
\bibliography{references}

\end{document}
