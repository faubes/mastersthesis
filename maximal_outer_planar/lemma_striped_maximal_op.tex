% Math and Comp Sci Student
% Homework template by Joel Faubert
% Ottawa University
% Winter 2018
%
% This template uses packages fancyhdr, multicol, esvect and amsmath
% http://ctan.cms.math.ca/tex-archive/macros/latex/contrib/esvect/esvect.pdf
% http://www.ams.org/publications/authors/tex/amslatex
%
% These packages are freely distribxted for use but not
% modification under the LaTeX Project Public License
% http://www.latex-project.org/lppl.txt

\documentclass[letterpaper, 10pt]{article}
% \usepackage[text={8in,10in},centering, margin=1in,headheight=28pt]{geometry}
\usepackage[margin=1in, centering, headheight=28pt]{geometry}
\usepackage{fancyhdr}
\usepackage{esvect}
\usepackage{amsmath}
\usepackage{bbold}
\usepackage{amsfonts}
\usepackage{amssymb}
\usepackage{amsthm}
\usepackage{mathrsfs}
\usepackage{mathtools}
\usepackage{multicol}
\usepackage{enumitem}
\usepackage{verbatimbox}
\usepackage{fancyvrb}
\usepackage{hyperref}
\usepackage[pdftex]{graphicx}
\usepackage{bm}
\usepackage{minted}

%\usepackage{sxbfigxre}

% Configure margins
\pagestyle{fancy}
% \hoffset -0.75pt
% \voffset -0.8pt
% \oddsidemargin 0pt
% \topmargin 0pt
% \headheight 25pt
% \headsep 20pt
% \textheight 8.25in
% \textwidth 6.25 in
% \marginparsep 5pt
% \marginparwidth 0.5in
% \footskip 10pt
% \marginparpush 0pt
\paperwidth 8.5in
\paperheight 11in

% Configxre headers and footers

\lhead{Prof. Jean-Lou De Carufel }
\rhead{Jo\"el Faubert \\ Student \# 2560106}
\chead{Zombies \& Survivors Working Document \\ 2-5-2018}
\rfoot{\today}
\fancyhfoffset[l]{40pt}
\fancyhfoffset[r]{40pt}
\renewcommand{\headrulewidth}{0.4pt}
\renewcommand{\footrulewidth}{0.4pt}
\setlength{\parskip}{10pt}
\setlist[enumerate]{parsep=10pt, itemsep=10pt}

% Define shortcuts

\newcommand{\floor}[1]{\lfloor #1 \rfloor}
\newcommand{\ceil}[1]{\lceil #1 \rcleil}

% matrices
%\newcommand{\bpm}{\begin{bmatrix}}
%\newcommand{\epm}{\end{bmatrix}}
%\newcommand{\vm}[3]{\begin{bmatrix}#1\\#2\\#3\end{bmatrix}}
%\newcommand{\Dmnt}[9]{\begin{vmatrix}#1 & #2 & #3 \\ #4 & #5 & #6 \\ #7 & #8 & #9 \end{vmatrix}}
%\newcommand{\dmnt}[4]{\begin{vmatrix}#1 & #2 \\ #3 & #4 \end{vmatrix}}
%\newcommand{\mat}[4]{\begin{bmatrix}#1 & #2\\#3 & #4\end{bmatrix}}

% common sets
\newcommand{\R}{\mathbb{R}}
\newcommand{\Qu}{\mathbb{Q}}
\newcommand{\Na}{\mathbb{N}}
\newcommand{\Z}{\mathbb{Z}}
\newcommand{\Rel}{\mathcal{R}}
\newcommand{\F}{\mathcal{F}}
\newcommand{\U}{\mathcal{U}}
\newcommand{\V}{\mathcal{V}}
\newcommand{\K}{\mathcal{K}}
\newcommand{\M}{\mathcal{M}}

% Power set
\newcommand{\PU}{\mathcal{P}(\mathcal{U})}

%norm shortcut
\DeclarePairedDelimiter{\norm}{\lVert}{\rVert}

% projection, vectors
\DeclareMathOperator{\proj}{Proj}
\newcommand{\vctproj}[2][]{\proj_{\vv{#1}}\vv{#2}}
\newcommand{\dotprod}[2]{\vv{#1}\cdot\vv{#2}}
\newcommand{\uvec}[1]{\boldsymbol{\hat{\textbf{#1}}}}

% derivative
\def\D{\mathrm{d}}

% big O
\newcommand{\bigO}{\mathcal{O}}

% probability
\newcommand{\Expected}{\mathrm{E}}
\newcommand{\Var}{\mathrm{Var}}
\newcommand{\Cov}{\mathrm{Cov}}
\newcommand{\Entropy}{\mathrm{H}}
\newcommand{\KL}{\mathrm{KL}}

\DeclareMathOperator*{\argmax}{arg\,max}
\DeclareMathOperator*{\argmin}{arg\,min}

\setlist[enumerate]{itemsep=10pt, partopsep=5pt, parsep=10pt}
\setlist[itemize]{itemsep=5pt, partopsep=5pt, parsep=10pt}

\begin{document}

\newtheorem{definition}{Definition}
\newtheorem{theorem}{Theorem}
\newtheorem{proposition}{Proposition}
\newtheorem{corollary}{Corollary}
\newtheorem{lemma}{Lemma}

\begin{definition}
A graph $G$ is \emph{planar} if it can be embedded in the plane such that its edges
intersect only at vertices \cite{bondy2008graph}.
\end{definition}

\begin{definition}
A graph $G$ is said to be \emph{outer-planar} if it is planar and has and embedding
in which all of the vertices of $G$ lie on the outer face \cite{campos2013dominating}.
\end{definition}

\begin{definition}
A graph $G$ is \emph{maximally outer planar} if the addition of any edge violates
 outer-planarity.
\end{definition}

Given an outer planar graph $G$ and an embedding, we define:

\begin{definition}
The \emph{outer hull} is the Hamilton cycle formed by the vertice and the edges
connecting them on the outer face.
\end{definition}

\begin{definition}
A \emph{chord} is an edge which is not on the outer hull.
\end{definition}

\begin{definition}
An \emph{ear} is vertex of degree 2. A maximally outer-planar graph has at least 2 ears.
\end{definition}

A maximally outer-planar graph is composed of triangles. In \cite{campos2013dominating},
they distinguish two types of triangles:

\begin{definition}
A \emph{marginal} triangle shares at least one edge with the outer face while
 an \emph{internal} triangle has no edges on the outer face.
\end{definition}

There is an interesting relationship between internal triangles and the number
of ears \cite{campos2013dominating}. If $k$ is the number of internal triangles, then there are $k+2$ ears.
This is because it takes 3 marginal triangles to create an internal triangle, and
adding another ear (without violating outer-planarity) creates a new internal triangle.

\begin{definition}
Finally, a maximally outer-planar graph is called \emph{striped} if it does not
contain any internal triangles.
\end{definition}

\begin{definition}
A \emph{round} of the game consists of two turns: the zombie's turn followed by the survivor's turn.
\end{definition}

With these definitions in mind, we seek to prove the following:

\begin{lemma}
  The zombie number of a striped maximally outer-planar graph is 1.
\end{lemma}

\begin{proof}
  The following diagram shows $v_0$ one ear of the graph ($v_k$ the other) and $v_1, v_{n-1}$
  its neighbours. Note that we must have one edge $v_1v_{n-2}$ or $v_{n-1}v_2$,
  so we can assume without loss of generality that the first few vertices are
  arranged as follows:
  \begin{center}
  \includegraphics[scale=0.25]{striped1.png}
  \end{center}
  The zombie is free to choose a starting point and so takes $v_0$.
  This is not optimal but convenient for the proof.

  The survivor must start on $v_i$ for some $2 \leq i \leq n-2$.
  For brevity we'll just say that the survivor is restricted to $[2,n-2]$.

  The zombie moves to either $v_1$ or $v_{n-1}$ one of which has degree 2 and
  the other has degree at least 3. Supposing $v_1$ is the vertex of degree at least 3.
  Notice that if $\text{deg}(v_1)= k > 3$ there is a chord $v_1v_{n-l}$ and there must also be chords
  $\{v_1v_{n-l+1}, v_1v_{n-l+2}, \dots, v_1v_{n-2} \}$ since this graph is assumed
  to be striped and omitting such an edge creates a cycle, which contradicts maximality.
  In order to fix maximality we need to add an edge $v_{n-l+1}v_{{n-m}}$ but this creates
  internal triangle $v_1v_{n-l+1}v_{n-m}$ which again contradicts our assumption that the graph is striped.



  If the zombie goes to $v_1$, then the survivor is restricted to $[3, n-3]$.
  If the zombie goes to $v_{n-1}$, then the survivor is restricted to $[2, n-3]$,
  but the survivor can't be on 2, since otherwise the zombie would have gone to $v_1$.
  So in either case, the survivor is now restricted to $[3, n-3]$, and the next
  move can be seen in the same way: no matter which vertex is chosen, the
  survivor is restricted to a zone smaller by 2. Inevitably, the survivor is
  cornered in the ear $v_k$ and is captured.
\end{proof}

\begin{theorem}
Maximally outer-planar graphs are zombie-win, regardless of the zombie's initial position.
That is to say, a single zombie
playing on a maximally outer-planar graph will always catch the survivor by following
the greedy strategy.
\end{theorem}

\begin{proof}
Our approach is to show that the \emph{survivor zone}, the set of vertices $S \subset V$
which the survivor can occupy without losing on the next round shrinks at every round of the game.

Observe that $K_3$ is zombie-win, so we can assume that $|V| = n > 3$.

The zombie and survivor choose initial positions $z^{(1)}, s^{(1)} \in V$,
the zombie player going first.

First assume that $z^{(1)}$ is an ear. If $s^{(1)} \in N(z^{(1)}$, then the game is won by the zombie
in round 1.
Otherwise, $z^{(2)} \in N(z^{(1)})$ is not an ear (since $n > 3$) and so has
degree at least 3. The survivor either passes or moves to some $s^{(2)}$ and, at the end of this
round the zombie is not on an ear.

If the survivor is on an ear adjacent to the zombie, then the zombie captures the survivor on its turn.
Otherwise, we claim that the zombie does not move to an ear.

If the zombie is on a vertex without any adjacent ears, then the claim is clear.
Assume then, that the zombie is on a vertex adjacent to an ear.
The survivor is not on this ear vertex by assumption.
Now supposing that the shortest path includes this ear leads to a contradiction by the triangle
inequality.

So the zombie always moves to non-ear vertices uness it is to capture the survivor,
 and thus we can now assume that $z^{(r)}$ (for $r>1$) is a vertex of degree at least 3.

Given an embedding of the graph, label the vertices $v_0, v_1, \dots, v_{n-1}$ clockwise such that
\[H_G = v_0v_1, v_1v_2 \dots, v_{n-2}v_{n-1}, v_{n-1}v_0\] is the outer hull (subscripts modulo $n$).

Now say that $z^{(r)} = v_i$ for some $1 \leq i \leq n$. The vertices $v_{i-1}$
and $v_{i+1}$ are the predecessor and successor of $v_i$ on the outer hull.

Since $v_i$ is not an ear and $n>3$, we also have that the set of vertices attached to $v_i$ by a chord:
\[ U = \{ u \in V\setminus \{v_{i-1}, v_{i+1}\} \vert \, v_i u \in E \} = \{v_{k_1}, v_{k_2}, \dots, v_{k_\ell} \}\]
is not empty.

%Let $v_k$ be an arbitrary vertex in $U$.
%Its neighbours on the outer hull are $v_{k-1}$ and $v_{k+1}$.

If the survivor is adjacent to $v_i$ then the zombie captures the survivor.
Otherwise, observe that $s^{(r)}$ is in $\{v_{i+1}, \dots, v_{k_\ell}\}$ or
$\{v_{k_1}, \dots, v_{i-1}\}$ (or both). Assume without loss of generality that
$s^{(r)}$ is in $\{v_{i+1}, \dots, v_{k_\ell}\}$. Therefore, the survivor zone is
$S^{(r)} = \{v_{i+1}, \dots, v_{k_\ell}\} \setminus \{v_{i+1}, v_{k_2}, \dots, v_{k_\ell}\}$.

Now, the shortest $zs$-path either starts with (1) a hull edge or (2) a chord edge.
\begin{enumerate}
  \item If the shortest $zs$-path starts with a hull edge then it is $v_iv_{i+1}$.
  Therefore, by the triangle inequality, the survivor must be in $\{v_{i+2}, \dots, v_{k_1 -1 }\}$.
  Since the graph is outer-planar, the survivor cannot leave this set without passing through
  either $v_{i+1}$ or $v_{k_1}$.

  (add figure (took picture))

  Since the graph is a triangulation and $v_iv_{k_1}$ is assumed to be the first clockwise chord from $v_i$,
  there must be an edge $v_{i+1}v_{k_1}$.
  Therefore, the survivor zone at round $r+1$ is
  $S^{(r+1)} \subseteq \{v_{i+2}, \dots, v_{k_1 -1 }\} \setminus \{v_{i+2}\} \subset S^{(r)}$.
  This shows that the survivor zone shrank by at least one vertex.

  \item If the shortest $zs$-path is a chord edge then it is $v_iv_{k_j}$ for some
    $v_{k_j} \in U$.

\end{enumerate}
Additionally, $\{v_i, v_k\}$ is a 2-vertex cut since no edges can cross $v_iv_k$ without
violating planarity. Thus, all paths from a vertex in $\{v_{k+1}, \dots, v_{i-1}\}$ to
a vertex in $\{v_{i+1}, \dots, v_{k-1}\}$ (or vice-versa) must pass through either $v_i$ or $v_k$.


Let us eliminate a few scenarios from consideration.

If $z^{(1)}$ is adjacent to all of the vertices in $S$, then the survivor is dominated.

If there are two chords $v_iv_k$ and $v_iv_j$, both of which lie on a shortest path to the survivor,
then the survivor must be on an ear adjacent to $v_k$ and $v_j$. Thus, either move puts $z^{(2)}$ in
a dominating position and the survivor loses on the next round.



Now consider any shortest $zs$-path. Denote as $z^{(2)} \in N(z^{(1)})$,
the vertex adjacent to the zombie along this shortest path.


We are left with two possibilities: either $z^{(2)} = v_{i+1}$, or $z^{(2)} = v_k \in U$.
In both cases, we know $\text{deg}(z^{(2)}) > 2$.

Let us begin with the first scenario, wherein $z^{(2)} = v_{i+1}$. That is to say,
the zombie follows a shortest path along the outer hull.

We must show that the survivor cannot escape the survivor zone. It is clear that the survivor cannot
move $s^{(2)} = v_i \in N(z^{(2)})$. It remains to show why the zombie cannot sneak by through
$v_k$.

If there is an edge $v_{i+1}v_k$, then it is also obvious that the survivor cannot move there.
Supposing there is no such edge, then there must be another edge $v_{i+1}v_{k-l}$ with $l\geq 1$
since the degree of $v_{i+1}$ is at least 3, and no edge may cross $v_iv_k$.

But then this forms a cycle $v_{i}, v_{i+1}, v_{k-l}, v_{k-l+1}, \dots, v_{k-1}, v_k$, which
implies that there are edges $v_iv_{k-1}$.

\end{proof}

\bibliographystyle{IEEEtran}
\bibliography{references}

\end{document}
