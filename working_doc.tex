
% Math and Comp Sci Student
% Homework template by Joel Faubert
% Ottawa University
% Winter 2018
%
% This template uses packages fancyhdr, multicol, esvect and amsmath
% http://ctan.cms.math.ca/tex-archive/macros/latex/contrib/esvect/esvect.pdf
% http://www.ams.org/publications/authors/tex/amslatex
%
% These packages are freely distribxted for use but not
% modification under the LaTeX Project Public License
% http://www.latex-project.org/lppl.txt

\documentclass[letterpaper, 10pt]{article}
% \usepackage[text={8in,10in},centering, margin=1in,headheight=28pt]{geometry}
\usepackage[margin=1in, centering, headheight=28pt]{geometry}
\usepackage{fancyhdr}
\usepackage{esvect}
\usepackage{amsmath}
\usepackage{bbold}
\usepackage{amsfonts}
\usepackage{amssymb}
\usepackage{amsthm}
\usepackage{mathrsfs}
\usepackage{mathtools}
\usepackage{multicol}
\usepackage{enumitem}
\usepackage{verbatimbox}
\usepackage{fancyvrb}
\usepackage{hyperref}
\usepackage[pdftex]{graphicx}
\usepackage{bm}
\usepackage{minted}

%\usepackage{sxbfigxre}

% Configure margins
\pagestyle{fancy}
% \hoffset -0.75pt
% \voffset -0.8pt
% \oddsidemargin 0pt
% \topmargin 0pt
% \headheight 25pt
% \headsep 20pt
% \textheight 8.25in
% \textwidth 6.25 in
% \marginparsep 5pt
% \marginparwidth 0.5in
% \footskip 10pt
% \marginparpush 0pt
\paperwidth 8.5in
\paperheight 11in

% Configxre headers and footers

\lhead{Prof. Jean-Lou De Carufel }
\rhead{Jo\"el Faubert \\ Student \# 2560106}
\chead{Zombies \& Survivors Working Document \\ 2-5-2018}
\rfoot{\today}
\fancyhfoffset[l]{40pt}
\fancyhfoffset[r]{40pt}
\renewcommand{\headrulewidth}{0.4pt}
\renewcommand{\footrulewidth}{0.4pt}
\setlength{\parskip}{10pt}
\setlist[enumerate]{parsep=10pt, itemsep=10pt}

% Define shortcuts

\newcommand{\floor}[1]{\lfloor #1 \rfloor}
\newcommand{\ceil}[1]{\lceil #1 \rcleil}

% matrices
%\newcommand{\bpm}{\begin{bmatrix}}
%\newcommand{\epm}{\end{bmatrix}}
%\newcommand{\vm}[3]{\begin{bmatrix}#1\\#2\\#3\end{bmatrix}}
%\newcommand{\Dmnt}[9]{\begin{vmatrix}#1 & #2 & #3 \\ #4 & #5 & #6 \\ #7 & #8 & #9 \end{vmatrix}}
%\newcommand{\dmnt}[4]{\begin{vmatrix}#1 & #2 \\ #3 & #4 \end{vmatrix}}
%\newcommand{\mat}[4]{\begin{bmatrix}#1 & #2\\#3 & #4\end{bmatrix}}

% common sets
\newcommand{\R}{\mathbb{R}}
\newcommand{\Qu}{\mathbb{Q}}
\newcommand{\Na}{\mathbb{N}}
\newcommand{\Z}{\mathbb{Z}}
\newcommand{\Rel}{\mathcal{R}}
\newcommand{\F}{\mathcal{F}}
\newcommand{\U}{\mathcal{U}}
\newcommand{\V}{\mathcal{V}}
\newcommand{\K}{\mathcal{K}}
\newcommand{\M}{\mathcal{M}}

% Power set
\newcommand{\PU}{\mathcal{P}(\mathcal{U})}

%norm shortcut
\DeclarePairedDelimiter{\norm}{\lVert}{\rVert}

% projection, vectors
\DeclareMathOperator{\proj}{Proj}
\newcommand{\vctproj}[2][]{\proj_{\vv{#1}}\vv{#2}}
\newcommand{\dotprod}[2]{\vv{#1}\cdot\vv{#2}}
\newcommand{\uvec}[1]{\boldsymbol{\hat{\textbf{#1}}}}

% derivative
\def\D{\mathrm{d}}

% big O
\newcommand{\bigO}{\mathcal{O}}

% probability
\newcommand{\Expected}{\mathrm{E}}
\newcommand{\Var}{\mathrm{Var}}
\newcommand{\Cov}{\mathrm{Cov}}
\newcommand{\Entropy}{\mathrm{H}}
\newcommand{\KL}{\mathrm{KL}}

\DeclareMathOperator*{\argmax}{arg\,max}
\DeclareMathOperator*{\argmin}{arg\,min}

\setlist[enumerate]{itemsep=10pt, partopsep=5pt, parsep=10pt}
\setlist[itemize]{itemsep=5pt, partopsep=5pt, parsep=10pt}

\begin{document}
\begin{enumerate}

\item Proof that Bifurcated cycle $Q_{2,n}$ is 2-zombie win

Given the following configuration for the zombie and survivor start positions:

\begin{center}
  \includegraphics[scale=0.25]{bifurcated_cycle_2_n_type_A.png}
\end{center}

Assuming
\begin{align*}
  n_1 + n_2 + n_3 + n_4 &= n \\
  n_1 &\geq 0 \\
  n_2 &\geq 2 \\
  n_3 &\geq 1 \\
  n_4 &\geq 0 \\
  n_1 + n_4 &\geq 1 & \text{and}\\
  n_i \in \Z^{+}
\end{align*}

Explanation: we must have $n_2 \geq 2$ and $n_1 + n_4 \geq 1$ or the survivor starts adjacent to a zombie and
loses immediately.

Now, we see that the survivor wins by running counter-clockwise if

\begin{align*}
  n_2 & < n_1 + n_3 + n_4 + 1 & \text{and}\\
  n_2 + n_3 & < n_1 + n_4 + 1
\end{align*}

or by running clockwise if

\begin{align*}
  n_2 & > n_1 + n_3 + n_4 + 1 & \text{and}\\
  n_2 + n_3 & > n_1 + n_4 + 1
\end{align*}

If either of these conditions are met, the zombies have no choice but to follow the shortest path to the
survivor around a cycle, which leads to a survivor win. In fact, these conditions are necessary (?) for the
survivor to have a chance.

Now, we can guarantee that these conditions are violated by ensuring that the two zombies are positioned
on opposite sides of the cycle. That is, by making $n_1 + n_2 + n_4 = n_3 + 1$ if $n$ is even, or
$n_1 + n_2 + n_4 = n_3 $ if $n$ is odd (?).

If $n_1 + n_2 + n_4 +1 = n_3$ then

\begin{align*}
  n_2 + (n_1 + n_2 + n_4 + 1) & < n_1 + n_4 + 1 \\
  2 n_2 & < 0
\end{align*}

Which contradicts our assumption that $n_2 \geq 2$. Similarly, if $n_1 + n_2 + n_4 = n_3$ then

\begin{align*}
  n_2 + (n_1 + n_2 + n_4) & < n_1 + n_4 + 1 \\
  2 n_2 & < 1
\end{align*}

Again a contradiction. This shows that there is a zombie start which makes it impossible for the survivor
to win in this configuration.

Suppose then that the players are positioned as follows:

\begin{center}
  \includegraphics[scale=0.25]{bifurcated_cycle_2_n_type_B.png}
\end{center}

Assuming
\begin{align*}
  n_1 + n_2 + n_3 + n_4 &= n \\
  n_1 &\geq 0 \\
  n_2 &\geq 2 \\
  n_3 &\geq 2 \\
  n_4 &\geq 0 \\
  n_i \in \Z^{+}
\end{align*}

Explanation: we must have $n_2 \geq 2$ and $n_3 \geq 2$ or the survivor starts adjacent to a zombie and
loses immediately.

Now, we see that the survivor wins by running counter-clockwise if

\begin{align*}
  n_1 + n_3 + n_4 + 1&< n_2 & \text{and}\\
  n_3 &< n_1 + n_2 + n_4 + 1
\end{align*}

or by running clockwise if

\begin{align*}
  n_1 + n_3 + n_4 + 1 &> n_2 & \text{and}\\
  n_3 &> n_1 + n_2 + n_4 + 1
\end{align*}


\end{enumerate}


\end{document}
