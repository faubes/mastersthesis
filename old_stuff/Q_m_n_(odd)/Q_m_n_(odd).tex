
% Math and Comp Sci Student
% Homework template by Joel Faubert
% Ottawa University
% Winter 2018
%
% This template uses packages fancyhdr, multicol, esvect and amsmath
% http://ctan.cms.math.ca/tex-archive/macros/latex/contrib/esvect/esvect.pdf
% http://www.ams.org/publications/authors/tex/amslatex
%
% These packages are freely distribxted for use but not
% modification under the LaTeX Project Public License
% http://www.latex-project.org/lppl.txt

\documentclass[letterpaper, 10pt]{article}
% \usepackage[text={8in,10in},centering, margin=1in,headheight=28pt]{geometry}
\usepackage[margin=1in, centering, headheight=28pt]{geometry}
\usepackage{fancyhdr}
\usepackage{esvect}
\usepackage{amsmath}
\usepackage{bbold}
\usepackage{amsfonts}
\usepackage{amssymb}
\usepackage{amsthm}
\usepackage{mathrsfs}
\usepackage{mathtools}
\usepackage{multicol}
\usepackage{enumitem}
\usepackage{verbatimbox}
\usepackage{fancyvrb}
\usepackage{hyperref}
\usepackage[pdftex]{graphicx}
\usepackage{bm}
\usepackage{minted}

%\usepackage{sxbfigxre}

% Configure margins
\pagestyle{fancy}
% \hoffset -0.75pt
% \voffset -0.8pt
% \oddsidemargin 0pt
% \topmargin 0pt
% \headheight 25pt
% \headsep 20pt
% \textheight 8.25in
% \textwidth 6.25 in
% \marginparsep 5pt
% \marginparwidth 0.5in
% \footskip 10pt
% \marginparpush 0pt
\paperwidth 8.5in
\paperheight 11in

% Configxre headers and footers

\lhead{Prof. Jean-Lou De Carufel }
\rhead{Jo\"el Faubert \\ Student \# 2560106}
\chead{Zombies \& Survivors Working Document \\ 10-7-2018}
\rfoot{\today}
\fancyhfoffset[l]{40pt}
\fancyhfoffset[r]{40pt}
\renewcommand{\headrulewidth}{0.4pt}
\renewcommand{\footrulewidth}{0.4pt}
\setlength{\parskip}{10pt}
\setlist[enumerate]{parsep=5pt, itemsep=0pt}

% Define shortcuts

\newcommand{\floor}[1]{\lfloor #1 \rfloor}
\newcommand{\ceil}[1]{\lceil #1 \rcleil}

% matrices
%\newcommand{\bpm}{\begin{bmatrix}}
%\newcommand{\epm}{\end{bmatrix}}
%\newcommand{\vm}[3]{\begin{bmatrix}#1\\#2\\#3\end{bmatrix}}
%\newcommand{\Dmnt}[9]{\begin{vmatrix}#1 & #2 & #3 \\ #4 & #5 & #6 \\ #7 & #8 & #9 \end{vmatrix}}
%\newcommand{\dmnt}[4]{\begin{vmatrix}#1 & #2 \\ #3 & #4 \end{vmatrix}}
%\newcommand{\mat}[4]{\begin{bmatrix}#1 & #2\\#3 & #4\end{bmatrix}}

% common sets
\newcommand{\R}{\mathbb{R}}
\newcommand{\Qu}{\mathbb{Q}}
\newcommand{\Na}{\mathbb{N}}
\newcommand{\Z}{\mathbb{Z}}
\newcommand{\Rel}{\mathcal{R}}
\newcommand{\F}{\mathcal{F}}
\newcommand{\U}{\mathcal{U}}
\newcommand{\V}{\mathcal{V}}
\newcommand{\K}{\mathcal{K}}
\newcommand{\M}{\mathcal{M}}

% Power set
\newcommand{\PU}{\mathcal{P}(\mathcal{U})}

%norm shortcut
\DeclarePairedDelimiter{\norm}{\lVert}{\rVert}

% projection, vectors
\DeclareMathOperator{\proj}{Proj}
\newcommand{\vctproj}[2][]{\proj_{\vv{#1}}\vv{#2}}
\newcommand{\dotprod}[2]{\vv{#1}\cdot\vv{#2}}
\newcommand{\uvec}[1]{\boldsymbol{\hat{\textbf{#1}}}}

% derivative
\def\D{\mathrm{d}}

% big O
\newcommand{\bigO}{\mathcal{O}}

% probability
\newcommand{\Expected}{\mathrm{E}}
\newcommand{\Var}{\mathrm{Var}}
\newcommand{\Cov}{\mathrm{Cov}}
\newcommand{\Entropy}{\mathrm{H}}
\newcommand{\KL}{\mathrm{KL}}

\DeclareMathOperator*{\argmax}{arg\,max}
\DeclareMathOperator*{\argmin}{arg\,min}

\setlist[enumerate]{itemsep=0pt, partopsep=5pt, parsep=0pt}
\setlist[itemize]{itemsep=0pt, partopsep=5pt, parsep=0pt}

\begin{document}

\theoremstyle{definition}
\newtheorem{definition}{Definition}
\newtheorem{theorem}{Theorem}
\newtheorem{proposition}{Proposition}
\newtheorem{corollary}{Corollary}
\newtheorem{lemma}{Lemma}
\newtheorem{proofpart}{Part}
\makeatletter
\@addtoreset{proofpart}{theorem}
\makeatother

\begin{definition}
 We define a family of graphs we call \emph{bifurcated cycles} and denote as $Q_{m,n}$.
 As the name suggests, bifurcated cycles are cycles of length $m+n$ with a single chord
 which divides the cycle into paths $P_1$ and $P_2$ of lengths $m$ and $n$.
\end{definition}

\begin{center}
 \includegraphics[scale=0.20]{Q_m_n}

 $Q_{m,n}$
\end{center}

\begin{theorem}
The Bifurcated cycle $Q_{m,n}$ is 2-zombie win if $m, n$ are odd.
\end{theorem}

\begin{proof}
We place the two zombies on the longest half of the bifurcated cycle with $z_1$ at a distance of $k$
from a chorded vertex and $z_2$ at a further distance of $k + \frac{n}{2}$.

Given this start configuration, we describe all winning strategies for $s$ in terms of
$m$, $n$, and $k$ by comparing lengths of available paths when zombies must make decisions.

From that, we will show that for all $m$ and $n$, there exists at least one value of $k$ such that
none of these winning strategies are viable and thus that the survivor will be captured.

\begin{proofpart} Notation

Formally, let $u,v \in V(Q_{m,n})$ denote the endpoints of the
chord and $P_1$, $P_2$ denote the paths on either side of the chord.

By construction we have $\lvert P_1 \rvert = m$ and $\lvert P_2 \rvert = n$ and we
can assume, without loss of generality, that $m \leq n$. We also assume $m,n \geq 2$, since
otherwise the construction adds parallel edges or degenerates to $K_2$.

Let $C_1$ and $C_2$ be the subcycles of length $m+1$ and $n+1$ induced by
$P_1$ and $P_2$ respectively.

Each round of the game is composed of two turns: first the zombies' turn, followed by the
survivor's turn. We denote as $z_i^{(t)} \in V(Q_{m,n})$ the position of zombie $i$
(and $s^{(t)}$ the position of the survivor) at round $t$.

\end{proofpart}

\begin{proofpart} The Zombie Start Positions

Now, as mentioned above, we place the two zombies on vertices $z_1^{(0)}$ and $z_2^{(0)}$ on $P_2$ such that

\begin{enumerate}
\item The distance between the two zombies is $d(z_1^{(0)}, z_2^{(0)}) = (n-1)/2$,  and
\item There is a path $P_5 = v, v_1, v_2, \dots, v_k = z_1^{(0)}$ of length $k$ between $z_1^{(0)}$ and
    the chorded vertex $v$.
    If $k=0$, then $P_5$ is the trivial path $v$, and $z_1^{(0)} = v$.
\end{enumerate}

\begin{center}
\includegraphics[scale=0.15]{diagram1.png}
\end{center}

Without loss of generality, we can assume that $0 \leq k \leq n/4$,
else we reflect the graph and rename the vertices.

These zombie positions divide $P_2$ into sub-paths $P_3 = u \dots z_2^{(0)}$,
$P_4 = z_2^{(0)} \dots z_1^{(0)}$, and
$P_5 = v \dots z_1^{(0)}$.
\end{proofpart}

\begin{proofpart} The First $k$ Moves of the Game

Let us first assume that $k>0$. We consider the case when $k=0$ in the following part.
Notice that if the survivor chooses to start on $P_4$, then the zombies are guaranteed to win
since

\[ 2 \leq d(z_i^{(0)}, s^{(0)}) \leq n/2 - 2 \qquad \text{for $i = 1,2$} \]

\begin{center}
\includegraphics[scale=0.15]{diagram3}
\end{center}

Play is effectively restricted to $P_4$. The zombies
move in opposite directions towards the survivor and inevitably corner it.

So we can assume that the survivor does not start on $P_4$. The survivor must
then be on $P_1$, $P_3$ or $P_5$ and, in all of these cases, the zombies' first $k$
moves are clear: the zombies move towards the corded vertices $u$ and $v$.

\begin{center}
\includegraphics[scale=0.15]{diagram4}
\end{center}

To see this, suppose first that the survivor is on $P_3$ or $P_5$. Then the zombies move in opposite directions
because the survivor must be at distance at least two and so we have

\[ 2 \leq d(z_i^{(0)}, s^{(0)}) \leq |P_3| + |P_5| +1 -2 = n/2 - 1 \qquad \text{for $i = 1,2$} \]

so neither zombie can choose to follow $P_4$.

If the survivor is on $P_1$, then all shortest paths to the survivor must
include $u$ or $v$ (since these are cut vertices)
and the shortest paths to these vertices from $z_1^{(0)}$
and $z_2^{(0)}$ cannot include $P_4$ because

\[ d(z_1^{(0)}, v) = |P_5| = k \leq \frac{n}{4} < \frac{n}{2} = |P_4| \qquad \forall n > 0\]
\[ d(z_1^{(0)}, u) \leq |P_5| + 1 = k + 1 \leq \frac{n}{4} +1 \leq \frac{n}{4} + \frac{n}{2} = \frac{3n}{4} < n -k = |P_4| + |P_3| \qquad \forall n > 0, 0 \leq k \leq \frac{n}{4}, k \in \Z\]
\[ d(z_2^{(0)}, u) = |P_3| = \frac{n}{2} - k < \frac{n}{2} = |P_4| \qquad \forall n > 0\]
\[ d(z_2^{(0)}, v) \leq |P_3| + 1 = \frac{n}{2} - k + 1 < \frac{n}{2} + k = |P_4| + |P_5| \qquad \forall n > 0, 0 < k \leq \frac{n}{4}, k \in \Z\]

We inevitably reach the following scenario: $z_1^{(k)}$ is on the chorded vertex $v$
and $z_2^{(k)}$ is approaching $u$ at a distance of $n/2 -2k$.

\begin{center}
\includegraphics[scale=0.15]{diagram5}
\end{center}

\end{proofpart}

\begin{proofpart} Once $z_1$ Reaches the Chord

If the survivor lies on the path between $u$ and $z_2^{(k)}$, then
$z_1^{(k)}$ follows the chord across the cycle and the survivor is encircled.
So we can assume now that, if the game is to continue, the survivor must be
somewhere on $P_1$.

Note that if $k=0$ and this is in fact the first turn of the game, then
the survivor loses by starting anywhere on $P_2$. So we can still assume
that the survivor is somewhere on $P_1$ at a distance $2 \leq \ell \leq m-1$ from $v$.

Since $s$ must be on $P_1$, we can consider all possible zombie decisions and their outcomes.
First, $z_2$ can go clockwise or counterclockwise.
Second, $z_1$ can go clockwise or take the chord to go counterclockwise.

\begin{center}
\includegraphics[scale=0.15]{diagram6}
\end{center}

Each zombie has two possible decisions (which depend on the position of the survivor)
for a total of four possibilities. We systematically analyze each of these possibilities
in the following way:

\begin{itemize}
  \item Case I $z_2$ goes clockwise.
  \begin{itemize}
    \item Case I (A) $z_1$ goes clockwise.
    \item Case I (B) $z_1$ goes counterclockwise.
  \end{itemize}
  \item Case II $z_2$ goes counterclockwise.
  \begin{itemize}
    \item Case I (A) $z_1$ goes clockwise.
    \item Case I (B) $z_1$ goes counterclockwise.
  \end{itemize}
\end{itemize}

\emph{Case I}: $z_2$ goes clockwise.

Let us first consider the possibility that
$z_2$ goes clockwise as it is a little different: it is only possible if $k=0$ since
comparing lengths of available $z_2v$-paths shows

\[\frac{n}{2} +2k \leq \frac{n}{2} -2k + 1\iff k=0 \]

So this outcome is possible only in the following situation: $z_2$ is exactly at the midpoint of $P_2$,
with paths of length $n/2$ on either side.

\begin{center}
\includegraphics[scale=0.15]{diagram7}
\end{center}

\emph{Case I(A)}: $z_1$ goes clockwise.

Since we assume here that $z_2$ moves clockwise, we must have $k=0$ and $\ell \leq m/2$.
This eliminates the possibility of \emph{Case I(B)}: if $z_2$ goes clockwise,
$z_1$ cannot go counterclockwise.

Note that if $\ell = m/2$ then $z_2$ may go either way and we must include this
possibility in both cases.

Since $\ell \leq m/2$, $z_1$ is forced to follow $s$ clockwise around $C_1$.

\begin{center}
\includegraphics[scale=0.15]{diagram8}
\end{center}

The survivor wishes to maintain distance at least 2 and so is forced to move around $C_1$. We can assume
the initial distance $\ell$ is preserved since the survivor passing (or even reversing) on its turn
is equivalent to choosing smaller initial distance of $\ell$.

We fast-forward the game and look at the next event: when the one of the players next attain the chord.
Note that if $s$ and $z_2$ reach $u$ and $v$ on the same round, then $z_2$ captures
the survivor on the next turn.

So either
\begin{enumerate}
\item $z_2$ reaches $v$ before $s$ reaches $u$; or
\item $s$ reaches $u$ before $z_2$ reaches $v$.
\end{enumerate}

\emph{Subcase I(A)1}: $z_2$ reaches $v$ before $s$ reaches $u$.

Since $z_2$ was at a distance of $n/2$, this event must occur $n/2$ rounds later and $z_1$ will
have pursued the survivor that length around $P_1$.

\begin{center}
\includegraphics[scale=0.15]{diagramCaseIA1_1}
\end{center}

We have supposed here that $s$ hasn't yet reached the chord, so there exists a path of length
\[m - \ell -n/2 \geq 1 \]
between $s$ and $u$.

On the following round, $z_2$ can either follow $z_1$ clockwise along a hull edge or go
counterclockwise using the chord edge. But since

\[ m - \ell - \frac{n}{2} +1 \leq m - 2 -\frac{n}{2}  +1 \leq n - 2 - \frac{n}{2} +1 = \frac{n}{2} -1 < \frac{n}{2} + \ell\]

We see that the shortest $z_2s$-path cannot follow the hull edge.
So $z_2$ takes the chord and moves counterclockwise.
After this zombie turn, we have
\[d(z_1, z_2) = \ell - 1 + m - \ell - \frac{n}{2} \leq \frac{m-1}{2}\]

So that the survivor is caught between two zombies on less than half the diameter of $C_1$.
This allows us to conlude that if the zombies start with $k=0$ and
\[ m - \ell -\frac{n}{2}  \geq 0 \]
then the survivor will lose.

To avoid this scenario, the survivor must choose $\ell > m - \frac{n}{2}$
(i.e. $\ell \geq m - \frac{n}{2} +1$)
while still respecting the restriction that $\ell \leq \frac{m}{2}$.

In order to choose such $\ell$ we must have
\[ m - \frac{n}{2} +1 \leq \ell \leq \frac{m}{2} \]
or, simply,
\[ m + 2 \leq n \]

Such choice for $\ell$ is impossible for the survivor whenever
$m+2 > n$, so we have a simple winning zombie-strategy for these configurations:
choose $k=0$.
%($m+2 > n$  and $m$,$n$ even implies $m=n$).

\emph{Subcase I(A)2}: $s$ reaches $v$ before $z_2$ reaches $u$.

\begin{center}
\includegraphics[scale=0.15]{diagramCaseIA2_1}
\end{center}

It takes $m-\ell$ rounds for $s$ to complete its circuit around $C_1$ and reach $u$. So
we must have $z_2$ at distance now $n/2 - (m-\ell)$ from $v$. This means we require

\[ \frac{n}{2} - (m-\ell) \geq 1 \]

This inequality allows us to bound $\ell$
\[ m - \frac{n}{2} +1 \leq \ell \leq \frac{m}{2}\]
which simplifies to
\[ n \geq m+2 \]

Notice that the survivor has won in this scenario since
\[ d(s,z_1) = \ell \leq \frac{m}{2} \leq \frac{n}{2} \]
and
\[ d(s,z_2) = \frac{n}{2} - (m - \ell) + 1 \leq \frac{n}{2} - m + \left(\frac{m}{2} -1\right) +1 = \frac{n}{2} - \frac{m}{2} < \frac{n}{2} \]

That is to say, the two zombies are now on the same side of $C_1$ at distance at most $\frac{n}{2}$
from the survivor, so the survivor can win by circling clockwise around $C_2$.

\emph{Case II}: $z_2$ goes counterclockwise.

Now, either
\begin{enumerate}
\item[(A)] $\ell \leq \frac{m}{2}$
    which forces $z_1$ to follow a hull edge onto $P_1$, or
\item[(B)] $\ell \geq \frac{m}{2} + 1$,
    which forces $z_1$ take the chord edge to $u$.
\end{enumerate}

\emph{Subcase II(A)}: We have $\ell \leq \frac{m}{2}$, so that $z_1$ follows a hull edge towards $s$.

\emph{Subcase II(A)1}: $z_2$ reaches the chord before the survivor.

We have assumed that $z_1$ is following $s$ in a clockwise direction. We must consider the distances at
round $n/2 - k$, when $z_2$ attains the chord.

\begin{center}
\includegraphics[scale=0.15]{diagramCaseIIA1_2}
\end{center}

Here $z_1$ must continue in the same direction. In order for the survivor to win, we must have
$z_2$ forced to take the chord on the next move and follow in clockwise direction. This implies that

\begin{align*}
1 + n/2-2k + \ell & < m - \ell - (n/2-2k)   \\
2\ell             & < m -n +4k -1           \\
2\ell             & \leq m - n +4k -2       \\
\ell              & \leq \frac{m-n+4k-2}{2}
\end{align*}

Since we know $\ell \geq 2$, this allows us to bound $\ell$:

\[ 2 \leq \ell \leq \frac{m-n+4k-2}{2} \]

In order to be able to choose $\ell$, we must then have
\[ 2 \leq \frac{m -n + 4k -2}{2} \]
or
\[ k \geq \frac{n-m+6}{4} \]

\emph{Subcase II(A)2}: The survivor is able to reach the chord before $z_2$ closes in.

\begin{center}
\includegraphics[scale=0.15]{diagramCaseIIA2_1}
\end{center}

In order for the survivor to win in this scenario, we must have $s$ able to
reach the chord before $z_2$ gets to $u$'s neighbour on $P_2$. This implies that

\begin{align*}
\frac{n}{2} - 2k - (m-\ell) & \geq 2                        \\
  \ell                        & \geq m + 2k - \frac{n}{2} + 2
\end{align*}

Now since $\ell \leq \frac{m}{2}$ we have

\[ m+2k-\frac{n}{2} +2 \leq \ell \leq \frac{m}{2} \]

So to be able to choose $\ell$ to make this strategy viable we require

\[ m+2k-\frac{n}{2} +2 \leq \frac{m}{2} \]

And solving for $k$ gives

\[ k \leq \frac{n-m-4}{4} \]

\emph{Subcase II(B)}: We have $\ell \geq \frac{m}{2}+1$, so that $z_1$ follows the chord edge towards $s$.

\emph{Subcase II(B)1}: $z_2$ reaches the chord before $s$.

\begin{center}
\includegraphics[scale=0.15]{diagramCaseIIB1_2}
\end{center}

We again assume that the survivor preserves its distances of $m-\ell+1$ from $z_1$,
since moving back or staying still is equivalent to choosing a larger initial value of $\ell$.
In order for the survivor to win, we must have $z_2$ forced to follow in the same direction.
This implies that

\begin{align*}
\frac{n}{2} -2k -1 + (m-\ell+1) & < 1 + 2k + \ell - \frac{n}{2} \\
n+m                          & < 1 + 4k + 2\ell              \\
2\ell                           & > n+m - 4k -1              \\
2\ell                           & \geq n+m -4k               \\
\ell                            & \geq \frac{n+m-4k}{2}
\end{align*}

Since $\ell \leq m -1$, we see that
\[ \frac{n+m-4k}{2} \leq \ell \leq m-1 \]

So in order to choose $\ell$ to enact this strategy we need

\[ \frac{n+m-4k}{2} \leq m-1 \]

Which allows us to conclude that
\[ k \geq \frac{n-m+2}{4} \]

\emph{Subcase II(B)2}: $z_1$ follows the chord edge and $s$ reaches the chord before $z_2$

We start with the same scenario as in (B)1; $z_1$ is forced to take the chord edge since
$\ell \geq \frac{m}{2} +1$.

\begin{center}
\includegraphics[scale=0.15]{diagramCaseIIB2_2}
\end{center}

$z_2$ was at a distance of $n/2-2k$ from the chorded vertex $u$ and $s$ requires
$\ell$ turns in order to reach $v$. Thus, in order for the survivor to escape
we must have

\[ \frac{n}{2} -2k - \ell \geq 1 \]

Solving for $\ell$ gives
\[ \ell \leq \frac{n}{2} -2k -1 \]

Combined with our lower bound for $\ell$ this gives

\[ \frac{m+2}{2} \leq \ell \leq \frac{n}{2} -2k -1 \]

So to be able to choose $\ell$ to make this strategy viable we need

\[ \frac{m+2}{2} \leq \frac{n}{2} -2k -1 \]

Solving for $k$ gives

\[ k \leq \frac{n-m-4}{4} \]

\end{proofpart}

\begin{proofpart} Conclusion


All together now, we have the following constraints for the different survivor-win scenarios:

\begin{itemize}
\item[II(A)1.] $k \geq \frac{n-m+6}{4}$
\item[II(A)2.] $k \leq \frac{n-m-4}{4}$
\item[II(B)1.] $k \geq \frac{n-m+2}{4}$
\item[II(B)2.] $k \leq \frac{n-m-4}{4}$

\end{itemize}

If any of these conditions on $k$ are true, then the surivor has a winning strategy.
So, to guarantee that none of these strategies will work, we must choose $k$ such that

\begin{itemize}
\item[II(A)1.] $k < \frac{n-m+6}{4}$
\item[II(A)2.] $k > \frac{n-m-4}{4}$
\item[II(B)1.] $k < \frac{n-m+2}{4}$
\item[II(B)2.] $k > \frac{n-m-4}{4}$
\end{itemize}

Are all satisfied. Or, equivalently,

\begin{itemize}
\item[II(A)1.] $k \leq \frac{n-m+5}{4}$
\item[II(A)2.] $k \geq \frac{n-m-3}{4}$
\item[II(B)1.] $k \leq \frac{n-m+1}{4}$
\item[II(B)2.] $k \geq \frac{n-m-3}{4}$
\end{itemize}

Now because

\[ \frac{n-m-3}{4} < \frac{n-m+1}{4} < \frac{n-m+5}{4} \]

We must choose $k \in [\frac{n-m-3}{4}, \frac{n-m+1}{4}]$. We know there exists
such an integer $k$ since:

\[ \frac{n-m+1}{4} - \frac{n-m-3}{4} = 1 \]

\end{proofpart}

\end{proof}
\end{document}
