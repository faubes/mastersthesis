
% Math and Comp Sci Student
% Homework template by Joel Faubert
% Ottawa University
% Winter 2018
%
% This template uses packages fancyhdr, multicol, esvect and amsmath
% http://ctan.cms.math.ca/tex-archive/macros/latex/contrib/esvect/esvect.pdf
% http://www.ams.org/publications/authors/tex/amslatex
%
% These packages are freely distribxted for use but not
% modification under the LaTeX Project Public License
% http://www.latex-project.org/lppl.txt

\documentclass[letterpaper, 10pt]{article}
% \usepackage[text={8in,10in},centering, margin=1in,headheight=28pt]{geometry}
\usepackage[margin=1in, centering, headheight=28pt]{geometry}
\usepackage{fancyhdr}
\usepackage{esvect}
\usepackage{amsmath}
\usepackage{bbold}
\usepackage{amsfonts}
\usepackage{amssymb}
\usepackage{amsthm}
\usepackage{mathrsfs}
\usepackage{mathtools}
\usepackage{multicol}
\usepackage{enumitem}
\usepackage{verbatimbox}
\usepackage{fancyvrb}
\usepackage{hyperref}
\usepackage[pdftex]{graphicx}
\usepackage{bm}
\usepackage{minted}

%\usepackage{sxbfigxre}

% Configure margins
\pagestyle{fancy}
% \hoffset -0.75pt
% \voffset -0.8pt
% \oddsidemargin 0pt
% \topmargin 0pt
% \headheight 25pt
% \headsep 20pt
% \textheight 8.25in
% \textwidth 6.25 in
% \marginparsep 5pt
% \marginparwidth 0.5in
% \footskip 10pt
% \marginparpush 0pt
\paperwidth 8.5in
\paperheight 11in

% Configxre headers and footers

\lhead{Prof. Jean-Lou De Carufel }
\rhead{Jo\"el Faubert \\ Student \# 2560106}
\chead{Zombies \& Survivors Working Document \\ 10-7-2018}
\rfoot{\today}
\fancyhfoffset[l]{40pt}
\fancyhfoffset[r]{40pt}
\renewcommand{\headrulewidth}{0.4pt}
\renewcommand{\footrulewidth}{0.4pt}
\setlength{\parskip}{10pt}
\setlist[enumerate]{parsep=5pt, itemsep=0pt}

% Define shortcuts

\newcommand{\floor}[1]{\lfloor #1 \rfloor}
\newcommand{\ceil}[1]{\lceil #1 \rcleil}

% matrices
%\newcommand{\bpm}{\begin{bmatrix}}
%\newcommand{\epm}{\end{bmatrix}}
%\newcommand{\vm}[3]{\begin{bmatrix}#1\\#2\\#3\end{bmatrix}}
%\newcommand{\Dmnt}[9]{\begin{vmatrix}#1 & #2 & #3 \\ #4 & #5 & #6 \\ #7 & #8 & #9 \end{vmatrix}}
%\newcommand{\dmnt}[4]{\begin{vmatrix}#1 & #2 \\ #3 & #4 \end{vmatrix}}
%\newcommand{\mat}[4]{\begin{bmatrix}#1 & #2\\#3 & #4\end{bmatrix}}

% common sets
\newcommand{\R}{\mathbb{R}}
\newcommand{\Qu}{\mathbb{Q}}
\newcommand{\Na}{\mathbb{N}}
\newcommand{\Z}{\mathbb{Z}}
\newcommand{\Rel}{\mathcal{R}}
\newcommand{\F}{\mathcal{F}}
\newcommand{\U}{\mathcal{U}}
\newcommand{\V}{\mathcal{V}}
\newcommand{\K}{\mathcal{K}}
\newcommand{\M}{\mathcal{M}}

% Power set
\newcommand{\PU}{\mathcal{P}(\mathcal{U})}

%norm shortcut
\DeclarePairedDelimiter{\norm}{\lVert}{\rVert}

% projection, vectors
\DeclareMathOperator{\proj}{Proj}
\newcommand{\vctproj}[2][]{\proj_{\vv{#1}}\vv{#2}}
\newcommand{\dotprod}[2]{\vv{#1}\cdot\vv{#2}}
\newcommand{\uvec}[1]{\boldsymbol{\hat{\textbf{#1}}}}

% derivative
\def\D{\mathrm{d}}

% big O
\newcommand{\bigO}{\mathcal{O}}

% probability
\newcommand{\Expected}{\mathrm{E}}
\newcommand{\Var}{\mathrm{Var}}
\newcommand{\Cov}{\mathrm{Cov}}
\newcommand{\Entropy}{\mathrm{H}}
\newcommand{\KL}{\mathrm{KL}}

\DeclareMathOperator*{\argmax}{arg\,max}
\DeclareMathOperator*{\argmin}{arg\,min}

\setlist[enumerate]{itemsep=0pt, partopsep=5pt, parsep=0pt}
\setlist[itemize]{itemsep=0pt, partopsep=5pt, parsep=0pt}

\begin{document}

\theoremstyle{definition}
\newtheorem{definition}{Definition}
\newtheorem{theorem}{Theorem}
\newtheorem{proposition}{Proposition}
\newtheorem{corollary}{Corollary}
\newtheorem{lemma}{Lemma}
\newtheorem{proofpart}{Part}
\makeatletter
\@addtoreset{proofpart}{theorem}
\makeatother

\begin{definition}
 We define a family of graphs we call \emph{bifurcated cycles} and denote as $Q_{m,n}$.
 As the name suggests, bifurcated cycles are cycles of length $m+n$ with a single chord
 which divides the cycle into paths $P_1$ and $P_2$ of lengths $m$ and $n$.
\end{definition}

\begin{center}
 \includegraphics[scale=0.20]{Q_m_n}

 $Q_{m,n}$
\end{center}

\begin{theorem}
The Bifurcated cycle $Q_{m,n}$ is 2-zombie win.
\end{theorem}

\begin{proof}
We place the two zombies on the longest half of the bifurcated cycle with $z_1$ at a distance of $k$
from a chorded vertex and $z_2$ at a further distance of $k + \lfloor \frac{n}{2}\rfloor$.

Given this start configuration, we describe all winning strategies for $s$ in terms of
$m$, $n$, and $k$.

We show that for all $m$ and $n$, there exists at least one value of $k$ such that
none of these winning strategies are viable and thus that the survivor will be captured.

\begin{proofpart} Notation

Formally, let $u,v \in V(Q_{m,n})$ denote the endpoints of the
chord and $P_1$, $P_2$ denote the paths on either side of the chord.

By construction we have $\lvert P_1 \rvert = m$ and $\lvert P_2 \rvert = n$ and we
can assume, without loss of generality, that $m \leq n$. We also assume $m,n \geq 2$, since
otherwise the construction adds parallel edges or degenerates to $K_2$.
In fact, we can assume that $m, n \geq 3$ since $Q_{2,2}$ has a dominating set of size 1 and
so is zombie-win.

Let $C_1$ and $C_2$ be the subcycles of length $m+1$ and $n+1$ induced by
$P_1$ and $P_2$ respectively.

Each round of the game is composed of two turns: first the zombies' turn, followed by the
survivor's turn. We denote as $z_i^{(t)} \in V(Q_{m,n})$ the position of zombie $i$
(and $s^{(t)}$ the position of the survivor) at round $t$.

\end{proofpart}

\begin{proofpart} The Zombie Start Positions

Now, as mentioned above, we place the two zombies on vertices $z_1^{(0)}$ and $z_2^{(0)}$ on $P_2$ such that

\begin{enumerate}
\item The distance between the two zombies is $d(z_1^{(0)}, z_2^{(0)}) = \lfloor n/2 \rfloor$,  and
\item There is a path $P_5 = v, v_1, v_2, \dots, v_k = z_1^{(0)}$ of length $k$ between $z_1^{(0)}$ and
    the chorded vertex $v$.
    If $k=0$, then $P_5$ is the trivial path $z_1^{(0)}v$.
\end{enumerate}

\begin{center}
\includegraphics[scale=0.15]{diagram1.png}
\end{center}

Without loss of generality, we can assume that $0 \leq k \leq n/4$,
else we reflect the graph and rename the vertices.

These zombie positions divide $P_2$ into sub-paths $P_3 = u \dots z_2^{(0)}$,
$P_4 = z_2^{(0)} \dots z_1^{(0)}$, and
$P_5 = v \dots z_1^{(0)}$.
\end{proofpart}

\begin{proofpart} The Survivor Cannot Start on $P_4$.

If the survivor chooses to start on $P_4$, then the zombies are guaranteed to win
since

\[ 2 \leq d(z_i^{(0)}, s^{(0)}) \leq \lfloor n/2 \rfloor - 2 \qquad \text{for $i = 1,2$} \]

\begin{center}
\includegraphics[scale=0.15]{diagram2}
\end{center}

Play is effectively restricted to $P_4$. The zombies
move in opposite directions towards the survivor and inevitably corner it.

So we can assume that the survivor does not start on $P_4$.

\end{proofpart}

\begin{proofpart} The First $k$ Rounds of the Game.

Since $s$ is not on $P_4$ we know that $z_1$ must move clockwise towards $v$:
\begin{itemize}
\item if $s$ starts on $C_2$ then
$d(z_1^{0}, s^{(0)}) \leq |P_5| + 1 + |P_4| - 2
= k + 1 + \lceil\frac{n}{2} \rceil -k - 1
= \lceil \frac{n}{2} \rceil$
whereas any path following $P_4$ must have length at least $\lfloor n/2 \rfloor +2$.
\item if $s$ starts on $C_1$ then the shortest $z_1s$-path includes $u$ or $v$ and for all $n \geq3$
\begin{itemize}
  \item $P_5$ plus the chord has length $k+1 \leq \frac{n}{4} + 1$
  whereas $|P_4| + |P_5| = n -k \geq \frac{3n}{4}$ so $|P_5| +1 < |P_4|+|P_3|$.
  \item $P_5$ has length $k \leq \frac{n}{4}$ whereas
  $|P_4| + |P_3| + 1 = n - k + 1 \geq \frac{3n}{4} +1$ so $|P_5| < |P_4| + |P_3| + 1$.
\end{itemize}
\end{itemize}
Thus, no matter where on $\V(Q_{m,n}) \setminus {P_4}$ the survivor starts, on round 1 $z_1$ moves
clockwise towards $v$. For the next $k$ rounds, $z_1$ must continue along this path
until it reaches $v$ since the survivor can never enter the opposite half of $C_2$ to
cause $z_1$ to change direction since it is guarded by $z_2$.

We know break the analysis into the first two cases: either $z_2$ goes clockwise or counterclockwise.

\end{proofpart}

\end{proof}
\end{document}
