
% Math and Comp Sci Student
% Homework template by Joel Faubert
% Ottawa University
% Winter 2018
%
% This template uses packages fancyhdr, multicol, esvect and amsmath
% http://ctan.cms.math.ca/tex-archive/macros/latex/contrib/esvect/esvect.pdf
% http://www.ams.org/publications/authors/tex/amslatex
%
% These packages are freely distribxted for use but not
% modification under the LaTeX Project Public License
% http://www.latex-project.org/lppl.txt

\documentclass[letterpaper, 10pt]{article}
% \usepackage[text={8in,10in},centering, margin=1in,headheight=28pt]{geometry}
\usepackage[margin=1in, centering, headheight=28pt]{geometry}
\usepackage{fancyhdr}
\usepackage{esvect}
\usepackage{amsmath}
\usepackage{bbold}
\usepackage{amsfonts}
\usepackage{amssymb}
\usepackage{amsthm}
\usepackage{mathrsfs}
\usepackage{mathtools}
\usepackage{multicol}
\usepackage{enumitem}
\usepackage{verbatimbox}
\usepackage{fancyvrb}
\usepackage{hyperref}
\usepackage[pdftex]{graphicx}
\usepackage{bm}
\usepackage{minted}

%\usepackage{sxbfigxre}

% Configure margins
\pagestyle{fancy}
% \hoffset -0.75pt
% \voffset -0.8pt
% \oddsidemargin 0pt
% \topmargin 0pt
% \headheight 25pt
% \headsep 20pt
% \textheight 8.25in
% \textwidth 6.25 in
% \marginparsep 5pt
% \marginparwidth 0.5in
% \footskip 10pt
% \marginparpush 0pt
\paperwidth 8.5in
\paperheight 11in

% Configxre headers and footers

\lhead{Prof. Jean-Lou De Carufel }
\rhead{Jo\"el Faubert \\ Student \# 2560106}
\chead{Zombies \& Survivors Working Document \\ 29-5-2018}
\rfoot{\today}
\fancyhfoffset[l]{40pt}
\fancyhfoffset[r]{40pt}
\renewcommand{\headrulewidth}{0.4pt}
\renewcommand{\footrulewidth}{0.4pt}
\setlength{\parskip}{10pt}
\setlist[enumerate]{parsep=5pt, itemsep=0pt}

% Define shortcuts

\newcommand{\floor}[1]{\lfloor #1 \rfloor}
\newcommand{\ceil}[1]{\lceil #1 \rcleil}

% matrices
%\newcommand{\bpm}{\begin{bmatrix}}
%\newcommand{\epm}{\end{bmatrix}}
%\newcommand{\vm}[3]{\begin{bmatrix}#1\\#2\\#3\end{bmatrix}}
%\newcommand{\Dmnt}[9]{\begin{vmatrix}#1 & #2 & #3 \\ #4 & #5 & #6 \\ #7 & #8 & #9 \end{vmatrix}}
%\newcommand{\dmnt}[4]{\begin{vmatrix}#1 & #2 \\ #3 & #4 \end{vmatrix}}
%\newcommand{\mat}[4]{\begin{bmatrix}#1 & #2\\#3 & #4\end{bmatrix}}

% common sets
\newcommand{\R}{\mathbb{R}}
\newcommand{\Qu}{\mathbb{Q}}
\newcommand{\Na}{\mathbb{N}}
\newcommand{\Z}{\mathbb{Z}}
\newcommand{\Rel}{\mathcal{R}}
\newcommand{\F}{\mathcal{F}}
\newcommand{\U}{\mathcal{U}}
\newcommand{\V}{\mathcal{V}}
\newcommand{\K}{\mathcal{K}}
\newcommand{\M}{\mathcal{M}}

% Power set
\newcommand{\PU}{\mathcal{P}(\mathcal{U})}

%norm shortcut
\DeclarePairedDelimiter{\norm}{\lVert}{\rVert}

% projection, vectors
\DeclareMathOperator{\proj}{Proj}
\newcommand{\vctproj}[2][]{\proj_{\vv{#1}}\vv{#2}}
\newcommand{\dotprod}[2]{\vv{#1}\cdot\vv{#2}}
\newcommand{\uvec}[1]{\boldsymbol{\hat{\textbf{#1}}}}

% derivative
\def\D{\mathrm{d}}

% big O
\newcommand{\bigO}{\mathcal{O}}

% probability
\newcommand{\Expected}{\mathrm{E}}
\newcommand{\Var}{\mathrm{Var}}
\newcommand{\Cov}{\mathrm{Cov}}
\newcommand{\Entropy}{\mathrm{H}}
\newcommand{\KL}{\mathrm{KL}}

\DeclareMathOperator*{\argmax}{arg\,max}
\DeclareMathOperator*{\argmin}{arg\,min}

\setlist[enumerate]{itemsep=10pt, partopsep=5pt, parsep=10pt}
\setlist[itemize]{itemsep=5pt, partopsep=5pt, parsep=10pt}

\begin{document}

\newtheorem{definition}{Definition}
\newtheorem{theorem}{Theorem}
\newtheorem{proposition}{Proposition}
\newtheorem{corollary}{Corollary}
\newtheorem{lemma}{Lemma}

\begin{definition}
 We define a family of graphs we call \emph{bifurcated cycles} and denote as $Q_{m,n}$.
 As the name suggests, bifurcated cycles are cycles of length $m+n$ with a single chord
 which divides the cycle into paths $P_1$ and $P_2$ of lengths $m$ and $n$.
\end{definition}

\begin{center}
 \includegraphics[scale=0.20]{Q_m_n}

 $Q_{m,n}$
\end{center}

\begin{enumerate}

 \item Proof that Bifurcated cycle $Q_{m,n}$ is 2-zombie win if $m, n$ are even.

       We demonstrate the existence of a foolproof 2-zombie-start.

       Let $u,v \in V(G)$ denote the endpoints of the chord and $P_1$ and $P_2$ denote the paths
       on either side of the cycle.

       By construction we have $\vert P_1 \vert = m$ and $\vert P_2 \vert = n$ and we
       can assume, without loss of generality, that $m \leq n$. We also assume $m,n \geq 2$, since
       otherwise the construction adds parallel edges or degenerates to $K_2$.

       Finally, let $C_1$ and $C_2$ be the subcycles of length $m+1$ and $n+1$ induced by
       $P_1$ and $P_2$ respectively.

       We place the two zombies on vertices $z_1^{(0)}$ and $z_2^{(0)}$ on $P_2$ such that

       \begin{enumerate}
        \item The distance between the two zombies is $d(z_1^{(0)}, z_2^{(0)}) = n/2$,  and
        \item There is a path $P_5 = v, v_1, v_2, \dots, v_k = z_1^{(0)}$ of length $k$ between $z_1^{(0)}$ and
              the chorded vertex $v$. $\vert P_5 \vert = d(z_1^{(0)}, v) = k \geq 0$.
              If $k=0$, then $P_5$ is the trivial path $v$, and $z_1^{(0)} = v$.
       \end{enumerate}

       \begin{center}
        \includegraphics[scale=0.15]{diagram1.png}
       \end{center}


       Without loss of generality, we can assume that $0 \leq k \leq n/4$,
       else we reflect the graph and rename the vertices.

       These zombie positions divide $P_2$ into sub-paths $P_3 = u \dots z_2^{(0)}$,
       $P_4 = z_2^{(0)} \dots z_1^{(0)}$, and
       $P_5 = v \dots z_1^{(0)}$.

       %Importantly, placing the zombies at a distance of $n/2$ from one
       %another divides $C_2$ almost neatly down the middle. If the survivor stays on $C_2$ for
       %too long (How long? $n/4 +1$ rounds?), the zombies mercilessly surround and capture the survivor.

       %So with this start, we know that the survivor, if it wishes to prolong the game, must flee to
       %$C_1$ and somehow trick the zombies into a distances of less than $n/2$ (or perhaps $m/2$,
       %if the survivor somehow returns to $C_2$) and lead them on an infinite loop in the same direction.

       Given this start configuration, we analyze all possible survivor-win scenarios
       to obtain relationships between $m$, $n$, and $k$.
       Then we show that there exists $k \geq 0$ such that the survivor will always be captured.

       At the crux of this argument is the fact, given initial conditions,
       the zombies behave predictably.

       First, notice that if the survivor chooses to start on $P_4$, then the zombies are guaranteed to win
       since

       \[ 2 \leq d(z_i^{(0)}, s^{(0)}) \leq n/2 - 2 \qquad \text{for $i = 1,2$} \]

       \begin{center}
        \includegraphics[scale=0.15]{diagramStartOnP4}
       \end{center}

       Play is effectively restricted to $P_4$. The zombies
       move in opposite directions towards the survivor and inevitably corner it.

       So we can assume that the survivor does not start on $P_4$.

       %Next, if the survivor starts on the other half of $C_2$ -- the path formed by $P_3$, $P_5$ and the chord --
       %then again the zombies must move in opposite directions towards the survivor because

       %\[ 2 \leq d(z_i^{(0)}, s^{(0)}) \leq (|P_3|+|P_5| +1) -2 = \frac{n}{2} -1 \qquad \text{for $i = 1,2$} \]

       %\begin{center}
       %  \includegraphics[scale=0.15]{diagramStartOnP5}
       %\end{center}

       %If the survivor stays on this path then it is eventually caught since play is restricted to half of $C_2$.

       We are left with two possible scenarios:

       \begin{enumerate}
        \item[I.] The zombies move in the same direction, or
        \item[II.] The zombies go in opposite directions
       \end{enumerate}

       \emph{Case I.} One way for the survivor to win is to force the zombies to
       spin in the same direction from the start.

       First we show by contradiction that the zombies cannot both go counterclockwise.
       Clearly $z_1$ cannot go counterclockwise if $s$ starts on $P_5$.
       It is impossible if the survivor starts on $C_1$, since all shortest zombie paths
       to $C_1$ must pass through $u$ or $v$ and the shortest paths to these vertices from $z_1^{(0)}$
       cannot include $P_4$ because

       \[ d(z_1^{(0)}, v) = |P_5| = k \leq \frac{n}{4} < \frac{n}{2} = |P_4| \qquad \forall n > 0\]
       \[ d(z_1^{(0)}, u) \leq |P_5| + 1 = k + 1 \leq \frac{n}{4} +1 \leq \frac{n}{4} + \frac{n}{2} = \frac{3n}{4} < n -k = |P_4| + |P_3| \qquad \forall n > 0, 0 \leq k \leq \frac{n}{4}, k \in \Z\]

       Suppose now that the survivor is on $P_3$ and forces the zombies to move counterclockwise.
       This implies that the shortest $z_1^0s^0$-path includes $P_4$ and at least 2 vertices of $P_3$,
       and so has length at least $n/2 + 2$. The other path which uses $P_5$,
       the chord and then part of $P_3$ has length at most
       \[k + 1 + (n/2 -k) -2 = n/2 -1\]
       This contradicts the choice of $|P_4|$ as shortest path for $z_1$.
       So under no circumstances can the survivor force the zombies to move counterclockwise
       at the beginning of the game.
       Let us now consider if it is possible to make the zombies move clockwise.

       \begin{center}
        \includegraphics[scale=0.15]{diagramCaseI_1}
       \end{center}

       Clearly it is not possible if the survivor starts on $P_3$.
       It is also impossible if the survivor starts on $P_5$ since assuming the survivor
       is on $P_5$ and $z_2$ moves clockwise gives

       \[ |P_4| + 2 = \frac{n}{2} +2 \leq |P_3| + 1 + |P_5| -2 = \left(\frac{n}{2} - k\right) + 1 + k-2 = \frac{n}{2} -1 \]

       A contradiction.
       So $z_2^{(0)}$ will not move clockwise if $s$ starts on $P_5$. The other
       possibility is that $s$ starts somewhere on $P_1$

       \begin{center}
        \includegraphics[scale=0.15]{diagramCaseI_2}
       \end{center}

       If we assume next that the survivor is on $P_1$ and that $z_2^{(0)}$ moves clockwise, then the shortest
       $z_2^{(0)}s^{(0)}$-path must pass through $v$ and so

       \[ |P_4| + |P_5| = n/2 + k < |P_3| + 1 = n/2 -k +1 \]

       or
       \[ n/2 +k \leq n/2 -k \]

       which is possible only when $k=0$. So $z_1^{(0)} = v$ and we have the following
       start situation:

       \begin{center}
        \includegraphics[scale=0.15]{diagramCaseI_3}
       \end{center}

       If $\ell = m/2$, $z_2^{(0)}$ may go in either direction.
       So we need $\ell \leq m/2 -1$ to force $z_2$ to move clockwise. We also need $\ell \geq 2$ else the survivor is captured by $z_1$
       on the next turn.

       For the next $m$ moves, $z_1$ will be forced to follow $s$ clockwise around $C_1$.
       The survivor must maintain distance at least 2 and so is forced to move around $C_1$. We can assume
       the initial distance $\ell$ is preserved since the survivor passing (or even reversing)
       is equivalent to choosing smaller initial distance of $\ell$.

       We look at the next event: when the next player attains the chord.
       Note that if $s$ and $z_2$ reach $u$ and $v$ on the same round, then $z_2$ captures
       the survivor on the next turn.

       So either
       \begin{enumerate}
        \item[(A)] $z_2$ attains the chorded vertex $v$ before $s$ reaches $u$; or, the reverse,
        \item[(B)] $s$ reaches the chord before $z_2$.
       \end{enumerate}

       \emph{Subcase I(A)} $z_2$ reaches $v$ before $s$ reaches $u$.

       Since $z_2$ was at a distance of $n/2$, this event must occur $n/2$ rounds later and $z_1$ will
       have pursued the survivor that length around $P_1$.

       \begin{center}
        \includegraphics[scale=0.15]{diagramCaseIA_1}
       \end{center}

       We have supposed here that $s$ hasn't yet reached the chord, so there exists a path of length
       \[m - \ell -n/2 \geq 1 \]
       between $s$ and $u$.

       On the following round, $z_2$ can either follow $z_1$ clockwise along a hull edge or go
       counterclockwise using the chord edge. But since

       \[ m - \ell - \frac{n}{2} +1 \leq m - 2 -\frac{n}{2}  +1 \leq n - 2 - \frac{n}{2} +1 = \frac{n}{2} -1 < \frac{n}{2} + \ell\]

       We see that the shortest $z_2s$-path cannot follow the hull edge.
       So $z_2$ takes the chord and moves counterclockwise.

       %The fact that we assumed $m \leq n$ implies that $\frac{m}{2} \leq \frac{n}{2}$, so if $s$ is still on $C_1$
       %after $n/2$ rounds, $z_1$ forces the survivor past the halfway point of $C_1$ from $u$ and so will
       %be encircled by $z_2$.

       This allows us to conlude that if the zombies start with $k=0$ and
       \[ m - \ell -\frac{n}{2}  \geq 0 \]
       then the survivor will lose.

       To avoid this scenario, the survivor must choose $\ell > m - \frac{n}{2}$,
       i.e. $\ell \geq m - \frac{n}{2} +1$
       while still respecting the restriction that $\ell \leq \frac{m}{2}-1$.

       In order to choose such $\ell$ we must have
       \[ m - \frac{n}{2} +1 \leq \ell \leq \frac{m}{2} -1 \]
       or, simply,
       \[ m + 4 \leq n \]

       Such choice for $\ell$ is impossible for the survivor whenever
       $m+3 \geq n$, so we have a simple winning zombie-strategy for these configurations:
       choose $k=0$.

       \emph{Subcase I(B)} $s$ reaches $v$ before $z_2$ reaches $u$.

       \begin{center}
        \includegraphics[scale=0.15]{diagramCaseIB_1}
       \end{center}

       It takes $m-\ell$ rounds for $s$ to complete its circuit around $C_1$ and reach $u$. So
       we must have $z_2$ at distance now $n/2 - (m-\ell)$ from $v$. This means we require

       \[ \frac{n}{2} - (m-\ell) \geq 1 \]

       This inequality allows us to bound $\ell$
       \[ m - \frac{n}{2} +1 \leq \ell \leq \frac{m-1}{2}\]
       which simplifies to
       \[ n \geq m+3 \]

       Notice that the survivor has won in this scenario since
       \[ d(s,z_1) = \ell \leq \frac{m}{2} -1 \leq \frac{n}{2} -1 \]
       and
       \[ d(s,z_2) = \frac{n}{2} - (m - \ell) + 1 \leq \frac{n}{2} - m + \left(\frac{m}{2} -1\right) +1 = \frac{n}{2} - \frac{m}{2} < \frac{n}{2} \]

       That is to say, the two zombies are now at distance less than $\frac{n}{2}$ from the survivor,
       so that the survivor now wins by looping around $C_2$.

       \emph{Case II.} From now on, we assume that the zombies go in different directions at the
       beginning of the game, so that we inevitably reach the first event of interest at round $k$:
       when $z_1$ reaches the chord.

       \begin{center}
        \includegraphics[scale=0.15]{diagram4}
       \end{center}

       Notice that when $k = n/4$ both zombies attain the chord at the same time. (Potentially important.
       If the zombies attain the chord at the same time the distance between them becomes 1,
       which is advantageous to the survivor trying to group them. Survivor must have zombies trailing within
       single file lemma).

       The survivor cannot be on $P_3''$, $P_4$ or $P_5$, as this would imply they
       somehow got around the zombies which are guarding these paths.

       Additionally, the survivor cannot be on $u$, since it is adjacent to $z_1^{(k)}$.
       So the survivor, if still alive, must be either on $P_1 \setminus N[v]$ or $P_3' \setminus \{ u \} $.

       If the survivor is on $P_3' \setminus \{ u \}$, then on the next turn $z_1^{(k)}$ moves to $u$
       and the survivor is surrounded on less than half of $C_2$ and hence loses.
       So we can assume further that the survivor must be on $P_1 \setminus N[v]$.
       Denote $P_1' : v = x_0 x_1 \dots x_\ell = s^{(k)}$, $P_1'' : s^{(k)} = y_0 y_1 \dots y_{m-\ell} = u$
       the subpaths formed by the survivor's position on $P_1$.

       \begin{center}
        \includegraphics[scale=0.15]{diagram5}
       \end{center}

       Now, either
       \begin{enumerate}
        \item[(A)] $\ell < \frac{m+1}{2}$ (or, equivalently, $\ell \leq \frac{m}{2}$),
              which forces $z_1$ to follow a hull edge onto $P_1$, or
              \begin{center}
               \includegraphics[scale=0.15]{diagram6}
              \end{center}
        \item[(B)] $\ell > \frac{m+1}{2}$ (or, equivalently, $\ell \geq \frac{m+2}{2}$),
              which forces $z_1$ take the chord edge to $u$.
              \begin{center}
               \includegraphics[scale=0.15]{diagram7}
              \end{center}
        \item[(C)] $l = \frac{m+1}{2}$, in which case the zombie can choose one or the other.
       \end{enumerate}

       In order to win, the survivor must either:

       \begin{enumerate}
        \item[1.] Force the two zombies to follow around $C_1$ in the same direction, or
        \item[2.] Be able to loop around $C_1$ with $z_1$ in pursuit before $z_2$ can reach the chord.
       \end{enumerate}

       We can assume that the survivor maintains its distance from $z_1$, since
       standing still or moving towards $z_1$ is the same as choosing a smaller (or larger, in case (B))
       initial value of $\ell$.
       We examine both possible survivor-winning scenarios for each possible $z_1$ decision.

       \emph{Case (A)}: We have $\ell < \frac{m+1}{2}$, so that $z_1$ follows a hull edge towards $s$.

       \emph{Subcase (A)1.} $z_2$ reaches the chord before the survivor.
       The survivor wins by forcing the zombies to follow in the same direction around $C_1$

       \begin{center}
        \includegraphics[scale=0.15]{diagramCaseA1_1}
       \end{center}

       We have assumed that $z_1$ is following $s$ in a clockwise direction. We must consider the distances at
       round $n/2 - k$, when $z_2$ attains the chord.

       \begin{center}
        \includegraphics[scale=0.15]{diagramCaseA1_2}
       \end{center}

       Here $z_1$ must continue in the same direction. In order for the survivor to win, we must have
       $z_2$ forced to take the chord on the next move and follow in clockwise direction. This implies that

       \begin{align*}
        1 + n/2-2k + \ell & < m - \ell - (n/2-2k)   \\
        2\ell             & < m -n +4k -1           \\
        2\ell             & \leq m - n +4k -2       \\
        \ell              & \leq \frac{m-n+4k-2}{2}
       \end{align*}

       Since we know $\ell \geq 2$, this allows us to bound $\ell$:

       \[ 2 \leq \ell \leq \frac{m-n+4k-2}{2} \]

       In order to be able to choose $\ell$, we must then have
       \[ 2 \leq \frac{m -n + 4k -2}{2} \]
       or
       \[ k \geq \frac{n-m+6}{4} \]

       \emph{Subcase (A)2.} The survivor is able to reach the chord before $z_2$ closes in.

       \begin{center}
        \includegraphics[scale=0.15]{diagramCaseA2_1}
       \end{center}

       In order for the survivor to win in this scenario, we must have $s$ able to
       reach the chord before $z_2$ gets to $u$'s neighbour on $P_2$. This implies that

       \begin{align*}
        \frac{n}{2} - 2k - (m-l) & \geq 2                        \\
        l                        & \geq m + 2k - \frac{n}{2} + 2
       \end{align*}

       Now since $l < \frac{m+1}{2}$, or $l \leq \frac{m}{2}$ we have

       \[ m+2k-\frac{n}{2} +2 \leq l \leq \frac{m}{2} \]

       So to be able to choose $\ell$ to make this strategy viable we require

       \[ m+2k-\frac{n}{2} +2 \leq \frac{m}{2} \]

       And solving for $k$ gives

       \[ k \leq \frac{n-m-4}{4} \]

       \emph{Case (B)}: We have $l > \frac{m+1}{2}$, so that $z_1$ follows the chord edge towards $s$.

       \emph{Subcase (B)1.} $z_2$ reaches the chord before $s$. The survivor wins by forcing
       the zombies to follow in the same (counterclockwise) direction around $C_1$.

       \begin{center}
        \includegraphics[scale=0.15]{diagramCaseB1_1}
       \end{center}

       We have assumed that $z_1$ is following $s$ in a counterclockwise direction around $C_1$. We again
       look at the next decision point: when $z_2$ attains the chord.

       \begin{center}
        \includegraphics[scale=0.15]{diagramCaseB1_2}
       \end{center}

       We have assumed that $k \leq \frac{n}{4}$. If we have equality, then the zombies reach
       the chord at the same time and the survivor has won since the zombies are on less than half
       of $C_1$ and will move in same direction.

       We can assume that the survivor preserves its distances of $m-l+1$ from $z_1$,
       since moving back or staying still is equivalent to choosing a larger initial value of $l$.
       In order for the survivor to win, we must have $z_2$ forced to follow in the same direction.
       This implies that

       \begin{align*}
        \frac{n}{2} -2k -1 + (m-l+1) & < 1 + 2k + l - \frac{n}{2} \\
        n+m                          & < 1 + 4k + 2l              \\
        2l                           & > n+m - 4k -1              \\
        2l                           & \geq n+m -4k               \\
        l                            & \geq \frac{n+m-4k}{2}
       \end{align*}

       Since $l \leq m -1$, we see that
       \[ \frac{n+m-4k}{2} \leq l \leq m-1 \]

       So in order to choose $\ell$ to enact this strategy we need

       \[ \frac{n+m-4k}{2} \leq m-1 \]

       Which allows us to conclude that
       \[ k \geq \frac{n-m+2}{4} \]

       \emph{Subcase (B)2.} $z_1$ follows the chord edge and $s$ reaches the chord before $z_2$

       We start with the same scenario as in (B)1; $z_1$ is forced to take the chord edge since
       $\ell > \frac{m+1}{2}$. Or, equivalently since $\ell \in \Z$, $\ell \geq \frac{m+2}{2}$.

       \begin{center}
        \includegraphics[scale=0.15]{diagramCaseB2_2}
       \end{center}

       $z_2$ was at a distance of $n/2-2k$ from the chorded vertex $u$ and $s$ requires
       $\ell$ turns in order to reach $v$. Thus, in order for the survivor to escape
       we must have

       \[ \frac{n}{2} -2k - \ell \geq 1 \]

       Solving for $\ell$ gives
       \[ \ell \leq \frac{n}{2} -2k -1 \]

       Combined with our lower bound for $\ell$ this gives

       \[ \frac{m+2}{2} \leq \ell \leq \frac{n}{2} -2k -1 \]

       So to be able to choose $\ell$ to make this strategy viable we need

       \[ \frac{m+2}{2} \leq \frac{n}{2} -2k -1 \]

       Solving for $k$ gives

       \[ k \leq \frac{n-m-4}{4} \]

       All together now, we have the following constraints for the different survivor-win scenarios:

       \begin{itemize}
        \item[II(A)1.] $k \geq \frac{n-m+6}{4}$
        \item[II(A)2.] $k \leq \frac{n-m-4}{4}$
        \item[II(B)1.] $k \geq \frac{n-m+2}{4}$
        \item[II(B)2.] $k \leq \frac{n-m-4}{4}$

       \end{itemize}

       If any of these conditions on $k$ are true, then the surivor has a winning strategy.
       So, to guarantee that none of these strategies will work, we must choose $k$ such that

       \begin{itemize}
        \item[II(A)1.] $k < \frac{n-m+6}{4}$
        \item[II(A)2.] $k > \frac{n-m-4}{4}$
        \item[II(B)1.] $k < \frac{n-m+2}{4}$
        \item[II(B)2.] $k > \frac{n-m-4}{4}$
       \end{itemize}

       Are all satisfied. Or, equivalently,

       \begin{itemize}
        \item[II(A)1.] $k \leq \frac{n-m+5}{4}$
        \item[II(A)2.] $k \geq \frac{n-m-3}{4}$
        \item[II(B)1.] $k \leq \frac{n-m+1}{4}$
        \item[II(B)2.] $k \geq \frac{n-m-3}{4}$
       \end{itemize}

       Now because

       \[ \frac{n-m-3}{4} < \frac{n-m+1}{4} < \frac{n-m+5}{4} \]

       We must choose $k \in [\frac{n-m-3}{4}, \frac{n-m+1}{4}]$. We know there exists
       such an integer $k$ since:

       \[ \frac{n-m+1}{4} - \frac{n-m-3}{4} = 1 \]



\end{enumerate}
\end{document}
