% Math and Comp Sci Student
% Homework template by Joel Faubert
% Ottawa University
% Winter 2018
%
% This template uses packages fancyhdr, multicol, esvect and amsmath
% http://ctan.cms.math.ca/tex-archive/macros/latex/contrib/esvect/esvect.pdf
% http://www.ams.org/publications/authors/tex/amslatex
%
% These packages are freely distribxted for use but not
% modification under the LaTeX Project Public License
% http://www.latex-project.org/lppl.txt

\documentclass[letterpaper, 10pt]{article}
% \usepackage[text={8in,10in},centering, margin=1in,headheight=28pt]{geometry}
\usepackage[margin=1in, centering, headheight=28pt]{geometry}
\usepackage{fancyhdr}
\usepackage{esvect}
\usepackage{amsmath}
\usepackage{bbold}
\usepackage{amsfonts}
\usepackage{amssymb}
\usepackage{amsthm}
\usepackage{mathrsfs}
\usepackage{mathtools}
\usepackage{multicol}
\usepackage{enumitem}
\usepackage{verbatimbox}
\usepackage{fancyvrb}
\usepackage{hyperref}
\usepackage[pdftex]{graphicx}
\usepackage{bm}
\usepackage{minted}

%\usepackage{sxbfigxre}

% Configure margins
\pagestyle{fancy}
% \hoffset -0.75pt
% \voffset -0.8pt
% \oddsidemargin 0pt
% \topmargin 0pt
% \headheight 25pt
% \headsep 20pt
% \textheight 8.25in
% \textwidth 6.25 in
% \marginparsep 5pt
% \marginparwidth 0.5in
% \footskip 10pt
% \marginparpush 0pt
\paperwidth 8.5in
\paperheight 11in

% Configxre headers and footers

\lhead{Prof. Jean-Lou De Carufel }
\rhead{Jo\"el Faubert \\ Student \# 2560106}
\chead{Zombies \& Survivors Working Document \\ 16-5-2018}
\rfoot{\today}
\fancyhfoffset[l]{40pt}
\fancyhfoffset[r]{40pt}
\renewcommand{\headrulewidth}{0.4pt}
\renewcommand{\footrulewidth}{0.4pt}
\setlength{\parskip}{10pt}
\setlist[enumerate]{parsep=10pt, itemsep=10pt}

% Define shortcuts

\newcommand{\floor}[1]{\lfloor #1 \rfloor}
\newcommand{\ceil}[1]{\lceil #1 \rcleil}

% matrices
%\newcommand{\bpm}{\begin{bmatrix}}
%\newcommand{\epm}{\end{bmatrix}}
%\newcommand{\vm}[3]{\begin{bmatrix}#1\\#2\\#3\end{bmatrix}}
%\newcommand{\Dmnt}[9]{\begin{vmatrix}#1 & #2 & #3 \\ #4 & #5 & #6 \\ #7 & #8 & #9 \end{vmatrix}}
%\newcommand{\dmnt}[4]{\begin{vmatrix}#1 & #2 \\ #3 & #4 \end{vmatrix}}
%\newcommand{\mat}[4]{\begin{bmatrix}#1 & #2\\#3 & #4\end{bmatrix}}

% common sets
\newcommand{\R}{\mathbb{R}}
\newcommand{\Qu}{\mathbb{Q}}
\newcommand{\Na}{\mathbb{N}}
\newcommand{\Z}{\mathbb{Z}}
\newcommand{\Rel}{\mathcal{R}}
\newcommand{\F}{\mathcal{F}}
\newcommand{\U}{\mathcal{U}}
\newcommand{\V}{\mathcal{V}}
\newcommand{\K}{\mathcal{K}}
\newcommand{\M}{\mathcal{M}}

% Power set
\newcommand{\PU}{\mathcal{P}(\mathcal{U})}

%norm shortcut
\DeclarePairedDelimiter{\norm}{\lVert}{\rVert}

% projection, vectors
\DeclareMathOperator{\proj}{Proj}
\newcommand{\vctproj}[2][]{\proj_{\vv{#1}}\vv{#2}}
\newcommand{\dotprod}[2]{\vv{#1}\cdot\vv{#2}}
\newcommand{\uvec}[1]{\boldsymbol{\hat{\textbf{#1}}}}

% derivative
\def\D{\mathrm{d}}

% big O
\newcommand{\bigO}{\mathcal{O}}

% probability
\newcommand{\Expected}{\mathrm{E}}
\newcommand{\Var}{\mathrm{Var}}
\newcommand{\Cov}{\mathrm{Cov}}
\newcommand{\Entropy}{\mathrm{H}}
\newcommand{\KL}{\mathrm{KL}}

\DeclareMathOperator*{\argmax}{arg\,max}
\DeclareMathOperator*{\argmin}{arg\,min}

\setlist[enumerate]{itemsep=10pt, partopsep=5pt, parsep=10pt}
\setlist[itemize]{itemsep=5pt, partopsep=5pt, parsep=10pt}

\begin{document}

\newtheorem{definition}{Definition}
\newtheorem{theorem}{Theorem}
\newtheorem{proposition}{Proposition}
\newtheorem{corollary}{Corollary}
\newtheorem{lemma}{Lemma}

\begin{definition}
We define a family of graphs we call \emph{bifurcated cycles} and denote as $Q_{m,n}$.
As the name suggests, bifurcated cycles are cycles of length $m+n$ with a single dividing chord such that the
chord is shared between two sub-cycles of lengths $m+1$ and $n+1$. The endpoints of the
chord form paths of lengths $m$ and $n$ by travelling around either side of the cycle.
\begin{center}
\includegraphics[scale=0.35]{Q_4,8}

$Q_{4,8}$
\end{center}
$Q_{m,n}$ can also be constructed by taking a cycle of length $m+1$, choosing one of its edges $uv$, and attaching
a path $P$ of length $n$ with $P : u = u_1, u_2, \dots, u_n = v$.
\end{definition}

It may be interesting to note that these graphs can be constructed iteratively by taking $Q_{m,n}$ and expanding
one of the cycle edges, giving either $Q_{m+1, n}$ or $Q_{m,n+1}$. Also note that $Q_{m,n} \sim Q_{n,m}$ by
an obvious isomorphism.

We seek to understand the game of Zombies and Survivors on this family.

Some basics

\begin{enumerate}
\item $Q_{2,2}$ is zombie-win since it has a dominating set of size 1: either of
the vertices attached to the chord.

\begin{center}
  \includegraphics[scale=0.35]{Q_2,2}
\end{center}

\item $Q_{2,3}$ has zombie number 2. A single zombie cannot win because it has subgraph $C_4$,
but two zombies suffice since it has dominating set of size 2:
one of the chorded vertices and one at distance 2 on $C_4$.

\begin{center}
  \includegraphics[scale=0.35]{Q_2,3}
\end{center}

\item $Q_{3,3}$ also has zombie number 2. Again, one zombie cannot win, but there is a dominating set of size 2

\begin{center}
  \includegraphics[scale=0.35]{Q_3,3}
\end{center}

\item $Q_{2,4}$ also has zombie number 2. Again, one zombie cannot win, but there is a dominating set of size 2

\begin{center}
  \includegraphics[scale=0.35]{Q_2,4}
\end{center}

\item $Q_{3,4}$ also has zombie number 2. It's not as easy as in the first few cases, but it is
fairly clear that the following zombie start leads to a loss for the survivor in round 2:

\begin{center}
  \includegraphics[scale=0.35]{Q_3,4}
\end{center}

There is only one vertex in $ V \setminus \left( N(z_1) \cup N(z_2) \right)$ that is
safe for the survivor to start. One round 1, he is immediately surrounded.

\item $Q_{4,4}$ also has zombie number 2 by an argument similar to the one above.
There are only 2 safe survivor starts, both of which are hopeless.

\begin{center}
  \includegraphics[scale=0.35]{Q_4,4}
\end{center}

\item $Q_{4,5}$ starts to get interesting. If we use a similar start strategy, the zombies
win depending on whether or not they choose the ``correct'' shortest path.

\begin{center}
  \includegraphics[scale=0.35]{Q_4,5}
\end{center}

Now there are 3 safe survivor starts. The one on the 4 side is no good.
However, the vertex at distance 3 from $z_1$ on the cycle $C_5$ could lead
to a survivor win if the zombies both attack from the same direction.

\end{enumerate}

It is intuitively clear that the chord is strategically important for the survivor:
it may allow for the manipulation of the zombies, changing their directions.

%\bibliographystyle{IEEEtran}
%\bibliography{references}

\end{document}
