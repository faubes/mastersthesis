\begin{frame}{Zombies on a planar graph}
\begin{figure}
  \includegraphics<1>[width=0.5\textwidth, height=0.8\textheight, keepaspectratio]{planar/Graph2.png}
  \includegraphics<2>[width=0.5\textwidth, height=0.8\textheight, keepaspectratio]{planar/Graph2Case1Round0.png}
  \includegraphics<3>[width=0.5\textwidth, height=0.8\textheight, keepaspectratio]{planar/Graph2Case1Round1.png}
  \includegraphics<4>[width=0.5\textwidth, height=0.8\textheight, keepaspectratio]{planar/Graph2Case1Round2.png}
  \includegraphics<5>[width=0.5\textwidth, height=0.8\textheight, keepaspectratio]{planar/Graph2Case1Round3.png}
  \caption{A planar graph where 3 zombies lose}
\end{figure}
\note[item]<1>{Not surprising that zombies not as effective as cops! Cannot apply strategy.}
\note[item]<1>{This is the smallest example we could find where 3 zombies always lose.}
\note[item]<2>{Shown by giving a winning strategy for every possible zombie start.}
\note[item]<2>{All possible zombie-starts divided into cases; proven individually.}
\note[item]<2>{3 zombies lose if all on interior since 5 zombies lose if all on interior.}
\end{frame}
