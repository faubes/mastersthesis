\begin{frame}{Cop-win Graphs and Dismantlings}
\begin{itemize}
  \item<1-> Study Cops and Robbers first attributed to Quilliot \cite{quilliot1978jeux}, and Nowakowski and Winkler \cite{nowakowski1983vertex}.

  \item<2-> Characterize graphs where the cop always wins, now known as \textit{cop-win} graphs.

 \item<3-> Exist ordering of the vertices called a \textit{dismantling}.

 \item<4-> Successive deletion of \textit{corners} resulting in a single vertex.
\end{itemize}
\end{frame}

\begin{frame}{Meyniel's Conjecture}
Upper bound on the cop-number \cite{frankl1987cops}
   \[ \bigO(\sqrt{\lvert V(G)\rvert }) \].

Incremental progress has been made on special classes of graphs as well as for graphs in general  \cite{gera2016graph}[p. 31].
\end{frame}

\begin{frame}{Cop-Number and Minimum Degree}
\end{frame}

\begin{frame}{Cop-Number and Genus of a Graph}
  In 2001, Schroeder conjectured \cite{bonato2017topological} that for a graph of genus $g$,
  the cop-number is at most $g+3$.

  \vspace{1cm}

  Currently, the best known bound \cite{schroder2001copnumber} for graph $G$ of genus $g$ is
  \[c(G) \leq \left\lfloor \frac{3}{2}g \right\rfloor +3\] .

\end{frame}
