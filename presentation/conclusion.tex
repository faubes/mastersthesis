\begin{frame}{Research Directions: Planar Zombies}
\begin{itemize}
  \item<1->Is there an upper bound on the zombie-number for planar graphs?

  \item<2->Is it possible to construct increasingly elaborate graphs (while still being planar) which would always provide the survivor with a winning strategy?
\end{itemize}
\note<1>[item]{In Chapter 2, we showed the existence of a graph for which 3 zombies always lose.}
\end{frame}

\begin{frame}{Research Directions: Zombies and Cycles}
\begin{itemize}
  \item Can this type of strategy be applied to any two cycles with a shared path?

  \item What about bowtie graphs?

  \item Can our results in $Q_{m,n}$ be used to determine the probabilistic zombie number?
\end{itemize}
\note[item]{In Chapter 3, we show how to win as two zombies on a cycle with one chord.}

\end{frame}


\begin{frame}{Research Directions: On Visibility Graphs}
\begin{itemize}
  \item Recently shown that the visibility graphs of simple polygons are cop-win.

  \item Are they also cop-win?

  \end{itemize}
  \begin{figure}
  \centering
  \includegraphics[height=0.45\textheight, keepaspectratio]{polygon/polygon_with_bfs_dismantling}
  \caption{A Polygon Inscribed with a BFS Cop-win Tree \label{fig:polygon_with_bfs_dismantling}}
\end{figure}
\end{frame}
