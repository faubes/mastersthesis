\begin{frame}{Games on Graphs}
  \centering
    \includegraphics<1>[width=0.95\textwidth, height=0.95\textheight, keepaspectratio]{intro/cops_and_robbers_1}
    \note<1>[item]{Graphs model networks, maps, relationships}

    \includegraphics<2>[width=0.95\textwidth, height=0.95\textheight, keepaspectratio]{intro/cops_and_robbers_2}
    \note<2>[item]{Vertex-pursuit games are perfect information adversarial games played on graphs.}
    \note<2>[item]{Set of nodes connected by edges or arcs}
    \note<2>[item]{Model movement, strategy, coordination of agents}
    \note<2>[item]{Variety of games: catching agents, containing fires/infection, intercepting messages, graph exploration}

    \includegraphics<3>[width=0.8\textwidth, height=0.8\textheight, keepaspectratio]{intro/cops_and_robbers_3}
    \note<3>[item]{Focus here is on Zombies and Survivors, a variety of Cops and Robbers.}
    \note<3>[item]{One team pursues the other team, who seek to evade.}
    \note<3>[item]{Perfect information}

    \includegraphics<4>[width=0.8\textwidth, height=0.8\textheight, keepaspectratio]{intro/cops_and_robbers_4}
    \note<4>[item]{Connected, finited and reflexive graphs.}
    \note<4>[item]{Reflexive: every vertex has a loop; players may pass/stay in place. }

    \includegraphics<5>[width=0.8\textwidth, height=0.8\textheight, keepaspectratio]{intro/cops_and_robbers_5}
    \note<5>[item]{Rules: Two players: one plays as cop(s), the other as robber(s).}
    \note<5>[item]{Cops pick starting vertices, followed by robbers.}
    \note<5>[item]{Players take turns; moving player tokens from vertex to neighbour.}

    \includegraphics<6>[width=0.8\textwidth, height=0.8\textheight, keepaspectratio]{intro/cops_and_robbers_6}

    \includegraphics<7>[width=0.8\textwidth, height=0.8\textheight, keepaspectratio]{intro/cops_and_robbers_7}

    \includegraphics<8>[width=0.8\textwidth, height=0.8\textheight, keepaspectratio]{intro/cops_and_robbers_8}

    \includegraphics<9>[width=0.8\textwidth, height=0.8\textheight, keepaspectratio]{intro/cops_and_robbers_9}

    \includegraphics<10>[width=0.8\textwidth, height=0.8\textheight, keepaspectratio]{intro/cops_and_robbers_10}
    \note[item]<10>{Cops win if robber(s) are captured.}
    \note[item]<10>{Robber(s) win if can evade indefinitely.}
    \note[item]<10>{Cops and Robbers was first introduced by Quilliot in late 70's.}
\end{frame}

\begin{frame}{Zombies and Survivors}
  \centering
  \includegraphics<1>[width=0.8\textwidth, height=0.8\textheight, keepaspectratio]{intro/zombies_and_survivors_1}
  \note[item]<1>{Twist on Cops and Robbers invented by \cite{fitzpatrick2016deterministic}.}
  \note[item]<1>{Same game, with zombies replacing cops. Only difference: zombies \textit{must} move closer to the survivor.}
  \includegraphics<2>[width=0.8\textwidth, height=0.8\textheight, keepaspectratio]{intro/zombies_and_survivors_2}
  \note[item]<2>{On their turn, zombies compute shortest path (geodesic) to survivor.}
  \includegraphics<3>[width=0.8\textwidth, height=0.8\textheight, keepaspectratio]{intro/zombies_and_survivors_3}
  \note[item]<3>{May be multiple such paths. Could be resolved by coin flip (probabilistic).}
  \note[item]<3>{Assume Zombies play optimally, as though they were coordinated and saw the future.}
  \note[item]<3>{Consider worst-case scenarios: if choice offered, and one of the possible outcomes leads to survivor capture, then that is considered a zombie win.}

\end{frame}

\begin{frame}{Cop-Number and Zombie-Number}
\begin{columns}
  \column{0.48\textwidth}
    \begin{figure}
      \centering
      \includegraphics[width=0.45\textwidth, keepaspectratio]{copwin_tree/copwin_but_notzombie_win}
      \caption{Cop-win but not zombie-win \cite{fitzpatrick2016deterministic})\label{copwin_but_not_zombiewin}}
  \end{figure}
  \column{0.48\textwidth}
\begin{itemize}
  \item<1-> Cop-Number $c(G)$ number of cops needed to guarantee a win on $G$.
  \item<2-> Zombie-Number $z(G)$ number of zombies needed to guarantee a win on $G$.
  \note[item]<2>{Under optimal play assumption. If winning sequence of moves (a winning zombie- or cop-play) exists.}
\end{itemize}
\begin{lemma}<3->
For any graph $G$, $c(G) \leq z(G)$.
\end{lemma}
\end{columns}
\end{frame}

\begin{frame}{Cop-Number and Zombie-Number}
  \centering
  \begin{table}
\begin{tabular}{c | c | c}
  $G$ & $c(G)$ & $z(G)$ \\
  \hline
  $T$ tree (acyclic) & 1 & 1 \\
  $C_3$ & 1 & 1 \\
  $C_n$ ($n \geq 4$) & 2 & 2 \\
  $K_n$ ($n \geq 1$) & 1 & 1 \\
  $K_{n,m}$ ($n,m \geq 2$) & 2 & 2 \\
  $G$ planar & 3 & \only<1>{?} \only<2>{at least 4} \\
  $G$ outerplanar & 2 & \only<1>{?} \only<2>{at least 3}
\end{tabular}
\caption{Cop and zombie number of a few graph families}
\end{table}
\end{frame}

\begin{frame}{Zombies vs. Cops on Planar Graphs}
A graph is planar if it can be \textit{embedded in the plane}.

\vspace{1cm}

Aigner and Fromme \cite{aigner1984game} described a winning 3 cop strategy: enclose the robber into a shrinking \textit{territory} using \textit{isometric paths}.

\vspace{1cm}

In next section, present a planar graph where 3 zombies lose.

\note[item]{Planar: Drawn on the plane such that edge crossings occur only at endpoints (vertices).}
\note[item]{For planar graphs, geodesics are isometric; allows guarding}
\note[item]{Constricting isometric paths strategy}
\note[item]{2 cops guard 2 isometric paths with same endpoints, forming cycle enclosing robber territory}
\note[item]{Third cop moves to guard a new path in the robber territory, halving it}
\note[item]{Not surprising zombies not as effective; cannot implement strategies.}
\end{frame}

\begin{frame}{On Outerplanar Graphs}
  \begin{columns}[onlytextwidth,T]
    \column{0.48\textwidth}
  A graph is \textit{outerplanar} if it can be \textit{embedded in a cycle on the plane} such that
  its vertices are all adjacent to the \textit{outer face}.

  \vspace{0.5cm}

  In \cite{clarke2002constrained}, Clarke showed that 2 cops suffice to win on outerplanar graphs.
  \column{0.48\textwidth}
  \begin{figure}
    \includegraphics[width=0.8\textwidth]{intro/outer_planar_1}
    \caption{An outerplanar graph}
  \end{figure}
  \end{columns}
  \note[item]{Clarke's proof considers two cases: blocks with and without cut vertices}
\end{frame}

\begin{frame}{Zombies vs Cops on Outerplanar Graphs}
  But 2 zombies lose on this outer-planar graph \ref{zombie_number_differ}.
  \begin{figure}
    \centering
    \includegraphics[width=0.35\textwidth, keepaspectratio]{intro/cop_and_zombie_number_fitzpatrick_fig_2}
    \caption{Two cops win but two zombies lose (Fig 2 of \cite{fitzpatrick2016deterministic})\label{zombie_number_differ}}
\end{figure}
\end{frame}
